
\textit{Adattamento di un discorso dato durante un ritiro condotto dal Ven. Tsoknyi Rinpoche tenuto presso la Wisdom House a Litchfield, Connecticut, nel Settembre del 1997. Una versione più estesa di questo discorso è stata pubblicata in ``Broad View, Boundless Heart'', con il titolo ``Ajahn Chah's View of the View''.}

\bigskip

\noindent\dropcaps{Q}{uando mi trovo di fronte} a insegnanti Dzogchen provo spesso la strana sensazione di sentire e vedere rievocate le immagini dei miei insegnanti, Ajahn Chah e Ajahn Sumedho; non si tratta semplicemente dei principi che mi sono familiari, ma addirittura dell'uso di identiche frasi e similitudini. Non appena mi accorsi di questa coincidenza, mi resi conto che da almeno metà della mia vita monastica, cioè da almeno il 1987, la mia pratica è stata vicina allo Dzogchen. Se avessi le sopracciglia, credo che le solleverei un po'.

Forse però, questa convergenza non è poi così sorprendente; dopo tutto, siamo tutti discepoli dello stesso insegnante: il Dharma viene dal Buddha e affonda le radici nella nostra stessa natura. Può anche darsi che ci siano 84.000 porte del Dharma, ma di fatto c'è un solo Dharma. 

Ci sono vari insegnamenti Tibetani che ho imparato ad apprezzare col tempo, soprattutto però quelli che descrivono l'anatomia e le sottili sfumature di \textit{rigpa}, anche detta la visione. La tradizione Thailandese della Foresta, il lignaggio in cui ho più praticato, dipende molto di più dall'eloquenza e dall'ispirazione cui ogni insegnante si affida improvvisando sui temi del Dharma. Se questo fa sì che gli insegnamenti siano sempre vivi e spontanei, d'altro canto spesso si hanno delle incongruenze in ciò che viene detto. Per questo ho avuto modo di imparare molto dagli insegnamenti Dzogchen, così strutturati e organizzati.

Gli insegnamenti di Ajahn Chah vertevano su molti argomenti, ma egli era particolarmente apprezzato per il modo aperto, schietto e sapiente in cui parlava della sfera della verità ultima. E questo avveniva con chiunque lui riteneva fosse in grado di comprendere, si trattasse di un laico o di un monaco. Il modo in cui parlava di questo regno, e della consapevolezza che lo conosce – la sua visione della visione – rispecchia molte similitudini con lo Dzogchen, per cui ritengo utile spiegarne alcune, così come vorrei illustrare alcuni metodi insegnati da Ajahn Sumedho, il suo discepolo anziano di origine occidentale. Cercherò anche di fornire alcune osservazioni dal punto di vista Theravāda che abbiano qualche attinenza con la nostra comprensione e la nostra pratica in questo ambito.

\section*{Quanto pi\`u ti affretti, tanto pi\`u vai piano}

È facile farsi prendere dalla vita spirituale, addirittura esserne dominati e ossessionati. Durante i primi 10 anni della mia vita monastica io divenni quasi un fanatico. Può sembrare un ossimoro, ma vi assicuro che non è impossibile. Cercavo di fare tutto al 120\%. Al mattino mi alzavo prestissimo per fare ogni tipo di pratica ascetica, tutte le \textit{pūjā} possibili e immaginabili. Non mi sdraiavo neppure per dormire, non mi sono sdraiato per circa tre anni. Alla fine mi resi conto che avevo troppe cose da fare; il giorno trascorreva senza che ci fosse un po' di spazio interiore.

Ero terribilmente occupato con la meditazione. In quel periodo la mia vita era strapiena di impegni; ero sempre infastidito e irascibile. Non riuscivo nemmeno ad attraversare il cortile o a mangiare senza che ciò fosse \textit{qualcosa}. Alla fine fui costretto a chiedermi: ``Perché faccio tutto questo? Si presume che questo tipo di vita sia per la pace, la realizzazione, la libertà; invece le mie giornate sono strapiene''.

Ce ne avevo messo di tempo per capire! Ero solito meditare seduto direttamente sul pavimento perché ai miei occhi lo \textit{zafu} era un segno di debolezza. Beh, una delle monache si era talmente stufata di vedermi ciondolare dal sonno durante le sedute che un giorno mi avvicinò e mi chiese: ``Posso offrirti un cuscino, Ajahn?''

``Grazie; sei molto gentile, ma non mi serve''.

Lei rispose: ``Io penso di sì''.

Alla fine andai da Ajahn Sumedho e gli dissi: ``Ho deciso di abbandonare tutte le mie pratiche ascetiche. Da adesso seguirò la routine generale e farò le cose in modo assolutamente normale.'' Fu la prima volta in cui lo vidi emozionarsi. ``Era ora!'' fu la risposta. Io invece ero convinto che avrebbe detto: ``Beh, se proprio non ce la fai.'' Aveva aspettato che mi rendessi conto che non era la quantità di cose che facevo, le ore che trascorrevo sul cuscino, il numero di \textit{mantra} che recitavo, o quanto rigidamente mi attenessi alle regole. Il punto era realizzare lo spirito del non divenire, non sforzarmi in tutto quello che facevo. Un barlume nella memoria mi riportò alla mente i tanti insegnamenti di Ajahn Sumedho sull'importanza di non sforzarsi; io, semplicemente, non li avevo ascoltati.

Ajahn Sumedho è solito stimolare la consapevolezza di ciò che chiamiamo ``la tendenza a divenire''. In pāli la parola usata è ``\textit{bhava}'', anche nella tradizione tibetana questo termine è usato con la stessa accezione. Significa il desiderio di diventare qualcosa. Si fa \textit{questo} per ottenere \textit{quello}. È un modo per darsi da fare, di impegnarsi, in cui ci si appropria di metodo, pratiche, regole e concetti per arrivare da qualche parte. Questa abitudine ci causa non pochi problemi. 

Affinché i semi germoglino c'è bisogno di terreno, concime, acqua e sole. Ma se ci dimentichiamo il sacco con i semi, allora abbiamo fallito nella pratica più importante. Mentre trasportiamo il concime e l'acqua ci sembra di stare facendo qualcosa. ``Sto proprio lavorando sodo alla mia pratica, adesso!'' Nel frattempo l'insegnante sta davanti al sacco con i semi per ricordarci. \textit{(Fa un gesto come se stesse indicando un sacco in un angolo)}.

Ajahn Sumedho parla spesso di \textit{essere} illuminati piuttosto che \textit{diventare} illuminati. Siate svegli adesso; siate risvegliati al momento presente. Non si tratta di fare qualcosa adesso per diventare illuminati in futuro. Questo modo di pensare è vincolato all'io e al tempo, e non produce frutti. Gli insegnamenti Dzogchen sono identici. Non si tratta di trovare rigpa in quanto oggetto, né di fare qualcosa adesso per ottenere rigpa in futuro; in realtà si tratta di essere rigpa adesso. Non appena cominciamo a farci qualche cosa, oppure a dire: ``Ehi, guarda, ce l'ho'' o ``Come posso trattenerlo?'' la mente si aggrappa a quel pensiero e abbandona rigpa -- a meno che il pensiero non sia contemplato come un'altra formazione trasparente dentro lo spazio di rigpa.

Nemmeno Ajahn Sumedho è stato sempre chiaro a questo proposito. Spesso racconta di come fosse ossessionato dal dover essere ``un meditante''. Il metodo insegnato da Ajahn Chah metteva molto l'accento sulla pratica di meditazione formale, ma egli stava anche estremamente attento a fare sì che la meditazione formale non fosse qualcosa di diverso dal resto della vita. Lui invitava a mantenere una continuità di pratica sia che si stesse in piedi, seduti o distesi, o che si stesse camminando. Lo stesso vale per il cibo, l'uso del bagno o il lavoro. Il punto era di mantenere sempre una continuità di consapevolezza. Egli era solito ripetere: ``Se la tua pace poggia sul cuscino di meditazione, quando lasci il cuscino la tua pace rimane lì''.

Una volta qualcuno donò ad Ajahn Chah un appezzamento di terra a boschi in cima a una collina nella sua provincia natale. Il generoso donatore gli disse: ``Se trovi il modo di aprire una strada fino alla cima della montagna, io ti costruirò un monastero''. Sempre pronto a raccogliere una sfida, Ajahn Chah trascorse un paio di settimane sulla montagna, fino a quando scoprì il modo di raggiungere la cima. A quel punto trasferì l'intero monastero lassù per costruire la strada.

Ajahn Sumedho era un monaco arrivato di recente. All'epoca si trovava lì da un anno o due ed era un praticante molto rigoroso. Non era stato propenso a lasciare la vita organizzata del monastero principale, il Wat Nong Pah Pong, ma si era unito agli altri, ed ora eccolo lì a spaccare pietre sotto al sole e spingere carriole piene di calcinacci, lavorando sodo con il resto della comunità. Dopo due o tre giorni si sentiva accaldato, sudato e dolorante. Alla fine della giornata, dopo un turno di 12 ore di lavoro, tutti si sedevano in meditazione e ciondolavano. Ajahn Sumedho pensava: ``Non serve a niente; sto sprecando il mio tempo. La mia meditazione è saltata completamente. Tutto questo non aiuta certo la vita spirituale''.

Così cercò di spiegare i suoi dubbi ad Ajahn Chah: ``Sento che tutto il lavoro che stiamo facendo è nocivo alla mia meditazione. Credo davvero che sarebbe meglio che io ne restassi fuori. Ho bisogno di fare più meditazione seduta e camminata, più pratica formale. Questo mi sarebbe di grande giovamento e credo che sarebbe anche la cosa migliore''.

Ajahn Chah disse: ``Ok, Sumedho. Puoi fare così. Sarà meglio però che informi il Sangha, così tutti sanno quello che sta succedendo''. Sapeva essere molto malizioso quando voleva.

Durante la riunione del Sangha disse: ``Voglio fare un annuncio a tutti. So che tutti noi siamo venuti quassù per costruire questa strada; so anche che stiamo tutti lavorando sodo per spaccare rocce e trasportare detriti. So che questo lavoro è molto importante per tutti noi, ma anche il lavoro di meditazione è altrettanto importante. Tan Sumedho mi ha chiesto di poter praticare la meditazione mentre noi costruiamo la strada e io gli ho risposto che per me non c'è nessun problema. Non voglio che nessuno di voi critichi questa scelta, perché per me va bene. Lui può rimanere da solo a meditare, mentre noi continueremo a costruire la strada''.

Ajahn Chah era sempre al lavoro dall'alba al tramonto. Quando non era fuori a lavorare riceveva gli ospiti o dava insegnamenti di Dharma. Era infaticabile. Nel frattempo Ajahn Sumedho se ne stava a meditare da solo. Questo lo fece stare abbastanza male il primo giorno, e ancora peggio il secondo. Il terzo giorno non ce la faceva più, stare lì da solo era una tortura. Così raggiunse il resto dei monaci a spaccare pietre e trasportare detriti, dedicandosi al lavoro con tutto se stesso. 

Ajahn Chah guardò il giovane monaco pieno di entusiasmo, che sorrideva tutto contento e gli chiese: ``Lavori volentieri, Sumedho?''

``Sì, Luang Por''.

``Non è strano che la tua mente sia più felice adesso, con il caldo e la polvere, di quanto non lo fosse mentre meditavi da solo?''

``Sì, Luang Por''.

La morale? Ajahn Sumedho aveva creato una falsa differenza fra ciò che è e ciò che non è la meditazione, quando di fatto, non c'è nessuna differenza. Quando siamo completamente devoti a ciò che facciamo, ciò che sperimentiamo, o ciò che accade intorno a noi, senza che preferenze e programmi prendano il sopravvento, lo spazio di rigpa, lo spazio della consapevolezza, è assolutamente lo stesso.

\section*{Il Buddha \`e consapevolezza}

Anche rispetto alla natura del Buddha gli insegnamenti di Ajahn Chah sono vicini a quelli Dzogchen. In sostanza la consapevolezza non è una \textit{cosa}. Eppure è un attributo della natura fondamentale della mente. Ajahn Chah era solito chiamare Buddha questa consapevolezza, questa `natura che conosce' della mente: ``Questo è il vero Buddha, colui che conosce (\textit{poo roo})''. Il modo in cui Ajahn Chah e altri maestri della tradizione della Foresta erano soliti parlare della consapevolezza era usare il termine ``Buddha'' in questo senso: la qualità perfettamente consapevole e sveglia della nostra mente. Questo è il Buddha.

Diceva cose del tipo: ``Il Buddha che è entrato nel \textit{parinibbāna} 2.500 anni fa non è il Buddha del rifugio''. A volte, quando voleva richiamare l'attenzione sui suoi insegnamenti, gli piaceva scandalizzare le persone. Quando diceva qualcosa di simile, le persone pensavano di trovarsi di fronte a un eretico. ``Come può quel Buddha essere un rifugio? Se ne è andato, veramente. Non può essere un rifugio. Un rifugio è un posto sicuro. Come può allora questo grande essere vissuto 2.500 anni fa fornire sicurezza? Pensare a lui può farci sentire bene, ma anche questa sensazione è transitoria. Dà un senso di ispirazione, che però può essere facilmente disturbata''.

Quando riposiamo nella conoscenza, nulla può ferire il cuore. È questo riposare nella conoscenza, che fa del Buddha un rifugio. La natura che conosce è invulnerabile e inviolabile. Quello che accade al corpo, emozioni e percezioni, è secondario, perché quella conoscenza va oltre il mondo fenomenico. Questo è il vero rifugio. Sia che sperimentiamo piacere o sofferenza, successo o fallimento, lodi o critiche, questa `natura che conosce' della mente è completamente serena. È indisturbata e incorruttibile. Così come uno specchio non è abbellito né contaminato dalle immagini che riflette, la conoscenza non può essere toccata da nessun senso di percezione, pensiero, emozione, stato d'animo o sensazione. È trascendente. Anche gli insegnamenti Dzogchen lo dicono: ``non c'è una punta di coinvolgimento degli oggetti mentali nella consapevolezza, nella natura della mente stessa''. Ecco perché la consapevolezza è un rifugio; la consapevolezza è il cuore stesso della nostra natura.

\section*{Qualcuno ha visto i miei occhi?}

Un'altra analogia tra lo Dzogchen e gli insegnamenti di Ajahn Chah si trova nell'avvertimento: non cercate l'incondizionato, o \textit{rigpa}, con la mente condizionata. Nei versi del Terzo Patriarca Zen si legge: ``Cercare la Mente con la mente discriminante è il più grave degli sbagli''. Ajahn Chah spiegava la futilità e assurdità di questa tendenza facendo l'esempio di qualcuno che, in sella a un cavallo, va in cerca del cavallo stesso. Noi cavalchiamo chiedendo: ``Qualcuno ha visto il mio cavallo?''. Tutti ci guardano come se fossimo pazzi. Cavalchiamo di villaggio in villaggio chiedendo la stessa cosa: ``Qualcuno ha visto il mio cavallo?''.

Ajahn Sumedho fa un esempio simile. Invece del cavallo lui usa l'immagine di qualcuno che cerca i propri occhi. L'organo stesso con cui vediamo è ciò che vede, eppure ce ne andiamo in giro a cercare: ``Qualcuno ha visto i miei occhi? Non vedo i miei occhi da nessuna parte. Devono essere qui da qualche parte, ma non riesco a trovarli''.

Non possiamo vedere i nostri occhi, ma possiamo vedere. Questo significa che la consapevolezza non può essere un oggetto. Però ci può essere consapevolezza. Ajahn Chah e altri maestri della Foresta usano l'espressione ``essere il conoscere''. È come \textit{essere} rigpa. In questo stato c'è la mente che conosce la propria natura, il Dharma che conosce la propria natura. È tutto qui. Il momento in cui cerchiamo di farne un oggetto, si è creata una struttura dualistica, un soggetto \textit{qui} che guarda un oggetto \textit{lì}. La soluzione è solo quando lasciamo andare il dualismo e abbandoniamo quel ``cercare''. Allora il cuore dimora solo nel conoscere. Però l'abitudine è quella di pensare: ``Non sto cercando abbastanza. Ancora non li ho trovati. Eppure i miei occhi devono essere da qualche parte. Dopo tutto posso vedere. Devo cercarli meglio''.

Vi siete mai trovati a fare un colloquio in un ritiro e mentre descrivete la vostra pratica di meditazione, l'insegnante vi guarda e dice: ``Devi sforzarti di più''? Voi pensate: ``Ma sto facendo il massimo!''. Dobbiamo fare uno sforzo, ma dobbiamo farlo in maniera saggia. Il tipo di sforzo cui dobbiamo arrivare comporta maggior chiarezza, ma fare di meno. Questa qualità di rilassamento non è considerata cruciale solo nello Dzogchen, ma anche nella pratica monastica Theravāda.

Il lato ironico è che questo rilassamento è la conseguenza di una moltitudine di pratiche preparatorie. Nel tirocinio Tibetano del \textit{ngondro }si fanno 100.000 prostrazioni, 100.000 visualizzazioni, 100.000 mantra, più anni di studio, attenersi completamente a \textit{sīla}, e così via. Anche nella tradizione Theravāda abbiamo sīla: la pratica delle virtù prevista sia per i laici sia per la comunità monastica; così come un rigoroso allenamento nella disciplina del Vinaya. Facciamo molti canti e pratiche devozionali, più un consistente allenamento nelle tecniche di meditazione, come la consapevolezza del respiro, consapevolezza del corpo, e così via. Poi c'è la pratica del vivere in comunità. Uno dei monaci anziani del mio Sangha una volta definì l'allenamento monastico comune come la pratica delle 100.000 \textit{frus}trazioni; non siamo promossi se non arriviamo a 100.000. C'è quindi bisogno di un immenso lavoro preparatorio perché quella rilassatezza sia effettiva.

Mi piace immaginare questa rilassatezza come un overdrive. Ingraniamo la quinta, stessa velocità, ma meno giri. Fino a quando non ho detto ad Ajahn Sumedho che avevo rinunciato alle mie pratiche ascetiche, stavo correndo in quarta. Il mio atteggiamento era un costante tentativo di andare al massimo. Quando mi decisi ad andare più piano, e non ero più così fanatico riguardo le regole e il dover fare tutto \textit{sempre} in modo perfetto, quel piccolo elemento di rilassamento fece scomparire il problema; semplicemente perché avevo lasciato andare lo stress, avevo smesso di spingere. Sembrava assurdo, ma ero ancora in grado di adempiere al 99,9\%\ dei miei doveri e pratiche spirituali, solo lo facevo in maniera non compulsiva. Possiamo rilassarci anche senza fermarci, così che possiamo goderci i frutti del nostro lavoro. È questo che intendiamo con lasciare andare il divenire e imparare ad \textit{essere}. Proprio come dei funamboli, se siamo troppo tesi e ansiosi di raggiungere l'altra sponda, siamo destinati a cadere dalla corda.

\section*{Realizzare la cessazione}

Un altro aspetto importante della visione è la sua risonanza con l'esperienza della cessazione, \textit{nirodha}. L'esperienza di rigpa è sinonimo dell'esperienza di \textit{dukkha-nirodha}, la cessazione della sofferenza.

Allettante, vero? Noi pratichiamo per porre fine alla sofferenza, eppure diventiamo così attaccati al nostro lavoro con gli oggetti della mente che quando dukkha finisce e il cuore si fa spazioso e vuoto, ci sentiamo persi. Non siamo capaci di non interferire con questa esperienza: ``Oh!, tutto è aperto, chiaro, spazioso\ldots{} e adesso che faccio?''. Il nostro condizionamento dice: ``Devo \textit{fare} qualcosa. Non è così che si avanza lungo il cammino''. Non sappiamo essere svegli e contemporaneamente lasciare in pace quella spaziosità.

Quando appare nella mente, questo spazio ci disorienta, oppure è probabile che lo ignoriamo. È come se ciascuno di noi fosse un ladro che penetra in una casa, si guarda intorno e dice: ``Non c'è molto da prendere qui, sarà meglio che me ne vada da qualche altra parte''. Ci sfugge il fatto che quando lasciamo andare, dukkha cessa. Invece ignoriamo questa qualità silenziosa, aperta e chiara per andare a cercare qualcos'altro, e qualcos'altro ancora, all'infinito. Potremmo dire che ``non assaporiamo il nettare'', il succo di rigpa, ma attraversiamo l'esperienza come razzi, pensando che non ci sia niente. Anzi, ci sembra addirittura noiosa: non c'è desiderio, né paura, né qualcosa da fare. Allora ci teniamo occupati con atteggiamenti tipo: ``Mi sto comportando da incosciente; dovrei concentrarmi su un oggetto; dovrei almeno contemplare l'impermanenza; non sto affrontando i miei problemi. Presto, bisogna che trovi qualcosa di difficile con cui lavorare.'' Mossi dalle migliori intenzioni, ci perdiamo il gusto del nettare che abbiamo davanti.

Quando smettiamo di aggrapparci, appare la verità ultima. È tutto qui.

\=Ananda e un altro monaco discutevano sulla natura dello stato senza morte e decisero di consultare il Buddha. Gli chiesero: ``Qual è la natura del senza morte?''. Si erano preparati a una spiegazione lunga e dettagliata, la risposta del Buddha fu invece breve e sintetica: ``Smettere di aggrapparsi è il senza morte''. Tutto qui.

A questo proposito gli insegnamenti Dzogchen e Theravāda sono identici. Quando finisce l'aggrapparsi c'è rigpa, c'è il senza morte, la fine della sofferenza, \textit{dukkha-nirodha}.

Il Buddha parlò espressamente di questo proprio nel suo primo insegnamento sulle Quattro Nobili Verità. Per ciascuna delle quattro verità, c'è un modo in cui deve essere praticata. La Prima Nobile Verità, dukkha, l'insoddisfazione, ``deve essere compresa''. Dobbiamo riconoscere che ``questo è dukkha. Questo non è rigpa. Questo è \textit{marigpa} (inconsapevolezza, ignoranza) ed è pertanto insoddisfacente''.

La Seconda Nobile Verità, la causa di dukkha, è il desiderio auto-centrato, la brama. Questo ``deve essere lasciato andare, abbandonato, vi si deve rinunciare''. 

La Quarta Nobile Verità, l'Ottuplice Sentiero, ``deve essere coltivato e sviluppato''.

Ma quello che più ci interessa in questo contesto è che la Terza Nobile Verità, Dukkha-nirodha, la fine di dukkha, ``deve essere realizzata''. Significa: quando dukkha cessa, \textit{notatelo}. Notate: ``Guarda, va tutto bene''. È qui che ingraniamo la quinta, quando semplicemente siamo, senza divenire.

``Ah!'' -- senti il gusto del nettare di rigpa -- Ah!, va bene''.

Nella tradizione Theravāda la realizzazione cosciente della cessazione di dukkha, della vacuità e dello spazio della mente, sono considerati elementi cruciali della retta pratica. Realizzare nirodha è, in qualche modo, l'aspetto più importante del lavoro con le Quattro Nobili Verità. Sembra insignificante, è il meno tangibile, eppure è quello che contiene la gemma, il seme dell'illuminazione.

Anche se l'esperienza di dukkha-nirodha non è una \textit{cosa}, non significa che non ci sia niente, che non abbia qualità. In realtà si tratta dell'esperienza della verità ultima. Se non ci precipitiamo a cercare l'oggetto successivo, ma prestiamo attenzione alla fine di dukkha, ci apriamo alla purezza, alla radiosità e alla pace. Permettendo al nostro cuore di gustare ciò che c'è, tutte le cosiddette esperienze ordinarie sbocciano e si schiudono, belle e ricche come una orchidea dorata, diventando più luminose e più chiare.

\section*{Non fatto di quello}

Tutti i praticanti buddhisti, a prescindere dalla tradizione, conoscono le tre caratteristiche dell'esistenza, anicca, dukkha e anattā (impermanenza, insoddisfazione e non-sè). Sono l'ABC del buddhismo. Nel Theravāda però si parla di altre tre caratteristiche dell'esistenza, a un livello più sottile: \textit{suññata, tathatā }e\textit{ atammayatā}. Suññata è la vacuità. Il termine deriva dal dire ``no'' al mondo fenomenico: ``Non credo a questo. Questo non è veramente reale''. Tathatā è la quiddità. Si tratta di una qualità molto simile a suññata, ma deriva dal dire ``sì'' all'universo. Non c'è nulla, eppure c'è \textit{qualcosa}. La qualità della quiddità è come la consistenza della realtà ultima. Suññata e tathatā – vacuità e quiddità, sono descritte in questo modo nelle scritture.

La terza qualità, atammayatā, non è molto conosciuta. Nel Theravāda atammayatā è definita come la concezione ultima. Alla lettera significa: ``non fatto di quello''; ma atammayatā può essere resa in molti modi, dando un'infinità di sfumature al suo significato. Bhikkhu Bodhi e Bhikkhu Ñaṇamoli (nella loro traduzione del \textit{Majjhima Nikāya}) usano l'espressione ``non identificazione'' – sottolineando il ``soggetto'' dell'equazione. Altri traduttori la definiscono ``non modellabilità'' o ``non mescolabilità'', focalizzandosi quindi sull ``oggetto''. In entrambi i casi, si riferiscono sostanzialmente alla qualità della consapevolezza precedente o priva di dualità soggetto-oggetto.

Sembra che le antiche origini indiane di questo termine si possano rintracciare in una teoria sulla percezione sensoriale secondo cui la mano che afferra fornisce l'analogia dominante: la mano assume la forma di ciò che contiene. Il processo del vedere, ad esempio, è spiegato in questo modo: l'occhio emette una specie di raggio che assume la forma di ciò che vede e torna indietro con questa. Lo stesso vale per il pensiero: l'energia mentale si adatta al suo oggetto (per esempio un pensiero) e quindi ritorna al soggetto. Questo concetto è sintetizzato nel termine ``tan-mayatā\textit{''}, ``consistente di quello''. L'energia mentale di colui che esperisce (soggetto) diventa consustanziale con la cosa (oggetto) compresa. 

La qualità opposta, atammayatā, indica uno stato in cui l'energia della mente non ``si reca'' dall'oggetto per occuparlo. Non crea né una ``cosa'' oggettiva, né un ``osservatore'' soggettivo che la conosce. Pertanto, la non identificazione si riferisce all'aspetto soggettivo mentre la non costruzione si riferisce all'aspetto oggettivo.

Il modo in cui si discute della vacuità in ambito Dzogchen rende molto chiaro che si tratta di una caratteristica della realtà ultima. Ma in altre circostanze l'accezione di vacuità e quiddità può comportare il senso di un agente (un soggetto), che è \textit{questo} che guarda \textit{quello}, là dove \textit{quello} è vuoto. Oppure \textit{quello} è così. Atammayatā è la realizzazione che, nella verità, non può esserci altro che la realtà ultima. Non esiste \textit{quello}. Nel lasciare andare, nell'abbandono completo di \textit{quello}, tutto il mondo relativo soggetto-oggetto, anche a livello più sottile, è frantumato e dissolto.

Mi piace molto la parola ``atammayatā'', proprio per il messaggio che trasmette. Tra le altre qualità, questo concetto si riferisce profondamente all'attitudine di chiedersi costantemente: ``Cosa c'è laggiù?'' come se ciò che si trova lì fosse sempre un po' più interessante di ciò che sta qui. Anche l'indefinibile senso di trascurare \textit{questo} per arrivare a \textit{quello}, di non essere soddisfatti con \textit{questo}, e voler diventare \textit{quello}, è un errore. Atammayatā è quella qualità in noi che sa che ``non c'è quello. C'è solo questo''. Allora anche il questo diventa privo di significato. Atammayatā aiuta il cuore a rompere con le impercettibili abitudini all'irrequietezza, così come a placare le riverberazioni del dualismo radicato di soggetto e oggetto. Questo abbandono porta il cuore a comprendere che c'è solo l'interezza del Dharma, spaziosità completa e appagamento. Gli apparenti dualismi di questo e quello, di soggetto e oggetto, sono visti come essenzialmente privi di significato. 

Un modo in cui possiamo usare tutto questo a livello pratico è attraverso una tecnica che Ajahn Sumedho ha consigliato spesso. Pensando che la mente sia nel corpo, diciamo: ``la mia mente'' $[$\textit{indica la propria testa$]$}, oppure ``la mia mente'' $[$\textit{si indica il petto$]$}. Giusto? ``È tutto nella mia mente''. Non abbiamo capito niente. Il nostro corpo è nella mente e non la mente nel corpo, giusto?

Cosa sappiamo del nostro corpo? Possiamo vederlo. Possiamo ascoltarlo. Possiamo odorarlo. Possiamo toccarlo. Dove ha luogo il vedere? Nella mente. Dove sperimentiamo il tatto? Nella mente. Dove sperimentiamo gli odori? Dove avviene? Nella mente.

Tutto ciò che sappiamo del corpo, adesso o in qualunque momento precedente, è stato conosciuto ad opera della nostra mente. Non abbiamo mai saputo nulla del nostro corpo tranne che per mezzo della mente. Tutta la nostra vita, fin dall'infanzia, tutto ciò che abbiamo mai saputo del nostro corpo e del mondo è successo nella nostra mente. Allora, dov'è il nostro corpo?

Non significa che non esiste un mondo fisico, quello che possiamo dire però è che l'esperienza del corpo, così come l'esperienza del mondo, avviene nella mente. Non avviene da qualche altra parte. Avviene tutta qui. E in questo essere qui, si smette di vedere il mondo come esterno e separato. Possiamo anche usare la parola ``cessazione'', (nirodha). Oltre alla accezione più conosciuta, questa parola significa anche ``contenere'', cioè può significare che la separatezza è cessata. Quando capiamo di contenere tutto il mondo dentro di noi, il suo essere cosa, il suo essere altro è stato superato. Siamo in condizioni migliori per riconoscerne la vera natura.

Questo cambio di prospettiva è un interessante piccolo strumento di meditazione che possiamo usare in qualsiasi momento, come è stato detto prima a proposito della meditazione camminata. È uno stratagemma molto utile perché ci conduce alla realtà della questione. Ogni volta che lo usiamo rovescia il mondo da dentro a fuori, perché siamo in grado di vedere che questo corpo non è altro che un insieme di percezioni. Non nega che noi funzioniamo liberamente, solo ridà il giusto senso alle cose. ``Tutto avviene nello spazio di rigpa, nello spazio della mente che conosce''. Se manteniamo le cose in questo modo, improvvisamente scopriamo che il nostro corpo, la mente e il mondo arrivano a una risoluzione, una strana realizzazione della perfezione. Avviene tutto qui. Questo metodo può apparire poco chiaro, ma a volte sono proprio gli strumenti più astrusi e poco comprensibili che ci permettono di realizzare i cambiamenti del cuore più radicali.

\section*{Indagine riflessiva}

L'indagine riflessiva era un altro metodo usato da Ajahn Chah per sostenere la visione o, per meglio dire, per sostenere la retta visione. Comporta l'uso deliberato del pensiero discorsivo per investigare gli insegnamenti, oppure specifici attaccamenti, paure e speranze, ma soprattutto il senso di identificazione stesso. Ne parlava quasi come se si trattasse di dialogo interiore.

Spesso in ambito buddhista il pensiero è visto come l'ospite indesiderato: ``Eh, la mia mente\ldots{} Se solo potessi smettere di pensare, sarei felice''. Eppure in realtà la mente pensante può dimostrarsi il più valido degli aiuti se usata nel modo giusto, soprattutto nell'investigare il senso dell'io. Trascurare l'uso del pensiero concettuale in questo modo significa perdere un'occasione. Quando state sperimentando, vedendo o facendo qualcosa, fatevi una domanda tipo: ``cos'è che è consapevole di questa sensazione? Chi possiede questo momento? Cos'è che conosce rigpa?''. 

L'uso deliberato del pensiero riflessivo, o indagine, può rivelare una serie di preconcetti, abitudini e compulsioni inconsci che abbiamo messo in moto. Fare questo può essere di grande aiuto e può produrre importanti intuizioni. Dopo esserci stabilizzati in una consapevolezza solida e aperta, ci chiediamo: ``Cos'è che conosce questo? Cos'è che è consapevole di questo momento? Chi è che prova dolore? Chi è che sta facendo questa fantasia? Chi è che sta pensando alla cena?''. In quel momento si crea una distanza. Una volta Milarepa disse qualcosa di questo genere: ``Quando il flusso del pensiero discorsivo è interrotto, si apre la porta della liberazione''. Esattamente allo stesso modo, quando poniamo una domanda simile è come se inserissimo un cuneo nell'intricato groviglio dell'identificazione, allentandolo. Rompe l'abitudine, lo schema del pensiero discorsivo. Quando domandiamo ``chi'' o ``cosa'', per un attimo la mente discorsiva inciampa nei propri piedi. Incespica. In quello spazio, prima che riesca a mettere insieme una risposta o un'identità, c'è infinita pace e libertà. Attraverso quello spazio pacificato appare la qualità innata della mente, l'essenza della mente. È solo quando frustriamo i nostri giudizi abituali, le realtà parziali che abbiamo posto in essere inconsciamente, che siamo forzati ad allentare la presa e lasciar andare il nostro insensato modo di pensare. 

\section*{Paura della libert\`a}

Il Buddha ha detto che lasciar andare il senso dell'io è la suprema felicità (es. in UD. 2.1 e 4.1). Nel corso degli anni però, ci siamo affezionati a questo personaggio, non è vero? Come disse una volta Ajahn Chah: ``È come avere un amico che conoscete da una vita. Siete inseparabili. Poi arriva il Buddha e dice che voi e il vostro amico dovete separarvi''. Vi si spezza il cuore. L'io rimane da solo. Provate un senso di abbattimento e di perdita. Poi piombate nella disperazione.

Per il senso dell'io, essere è sempre definito in termini di essere \textit{qualcosa}. Invece la pratica e gli insegnamenti enfatizzano sempre l'essere indefinito, una consapevolezza: illimitato, incolore, infinito, onnipresente, scegliete voi. Quando essere è definito in questo modo, l'io lo vede come la morte. E la morte è la cosa peggiore. Le abitudini fondate sull'io gridano vendetta, cercano qualcosa con cui colmare il vuoto. Va bene qualunque cosa: ``Presto, dammi un problema, una pratica di meditazione (\textit{questa }è lecita!). O almeno un ricordo, una speranza, una responsabilità che non mi sono preso, qualcosa con cui tormentarmi o per cui sentirmi colpevole; \textit{qualunque cosa}!''.

Ci sono passato molte volte. È come se in quella spaziosità ci fosse un cane famelico che cerca disperatamente di entrare: ``Dai, fammi entrare, fammi entrare''. Il cane famelico vuole sapere: ``Quand'è che questo tizio mi presterà un po' di attenzione? Sono ore che se ne sta lì seduto come un maledetto Buddha. Non lo sa che io sto soffrendo la fame qua fuori? Non lo sa che fa freddo e piove? Non gli importa niente di me?''.

``Tutti i \textit{sankara }sono impermanenti. Tutti i dharma sono così e vuoti. Non c'è nessun altro''\ldots{} $[$\textit{fa dei versi come quelli disperati di un cane affamato$]$}. 

Queste esperienze hanno determinato alcuni dei momenti più illuminanti della mia pratica e della mia ricerca. Contengono una tale ansia famelica di \textit{essere}. Qualunque cosa va bene, qualsiasi cosa, purché si possa essere qualcosa: un fallito, un eroe, un messia, una iattura per il mondo, un assassino di massa. ``Solo, fa che io sia qualcosa; ti prego, Dio, Buddha, chiunque''.

Al che la saggezza del Buddha risponde: ``No''.

Ci vogliono risorse interiori e una forza eccezionali per poter dire ``no'' in questo modo. Le suppliche patetiche dell'io diventano incredibilmente intense, viscerali. Può darsi che siamo presi da tremori e le gambe tremano dalla voglia di mettersi a correre. ``Fammi uscire da questo posto!''. Lo stimolo è così imperioso che forse i nostri piedi cominciano addirittura a muoversi per raggiungere la porta.

A questo punto, stiamo facendo risplendere la luce della saggezza dritto alla radice dell'esistenza separata. È una radice molto solida. Ci vuole un bel po' di lavoro per raggiungere quella radice ed eliminarla. Dobbiamo pertanto aspettarci non poche frizioni e difficoltà quando siamo impegnati in questo lavoro.

Sicuramente proverete una forte ansia. Non ve ne fate intimidire. Non cedete allo stimolo. È normale provare un forte senso di lutto e di perdita. C'è un piccolo essere che è appena morto. Il cuore è spazzato da un'onda di vuoto. Stateci, e lasciate che passi. La sensazione che ``qualcosa andrà perso se non seguo questo richiamo'' è il messaggio ingannevole del desiderio. Sia che si tratti di una impercettibile scintilla di irrequietezza, sia di una solenne dichiarazione: ``Morirò di crepacuore se non lo seguo!'', riconoscetele sempre come la seduzione ingannevole del desiderio.

C'è un bellissimo verso di Rumi che dice: ``quando mai il morire ti ha diminuito?''. Lasciate che quel richiamo dell'io nasca e lasciatelo morire. Allora, ecco, non solo il cuore non è diminuito, in realtà è più raggiante, vasto e gioioso che mai. C'è spaziosità, soddisfazione, e un benessere infinito che non può essere ottenuto se ci aggrappiamo o se ci identifichiamo con un qualunque attributo della vita.

A prescindere da quanto possano sembrare autentici i problemi, le responsabilità, le passioni o le esperienze, noi non dobbiamo necessariamente essere ciò. Non c'è nessuna identità che si \textit{debba} essere. Non c'è assolutamente nulla cui sia necessario aggrapparsi.

