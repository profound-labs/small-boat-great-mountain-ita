
\vspace*{-0.3em}
\dropcaps{U}{no dei temi} su cui Ajahn Chah era solito insistere era il principio del non dimorare. Durante i due brevi anni che trascorsi con lui in Thailandia ebbe modo di parlarne ripetutamente. In vari modi egli cercava di spiegare che il non dimorare è l'essenza del cammino, un fondamento per la pace e una porta di accesso al mondo della libertà.

\vspace*{-1.2em}
\section*{I limiti della mente condizionata}

Durante l'estate del 1981, Ajahn Chah diede ad Ajahn Sumedho un insegnamento fondamentale sulla qualità liberante del non dimorare. Ajahn Sumedho si trovava in Inghilterra da alcuni anni, quando ricevette una lettera dalla Thailandia. Sebbene Ajahn Chah sapesse leggere e scrivere, non lo faceva quasi mai. Di fatto non ha scritto quasi nulla, e non scriveva \textit{mai} lettere. Il messaggio cominciava con la nota di un altro monaco occidentale, che diceva: ``Ajahn Sumedho, tu non ci crederai, ma Luang Por ha deciso di inviarti una lettera e mi ha chiesto di scrivere quello che segue''. Il messaggio di Ajahn Chah era molto stringato e suonava così: ``Ogni volta che provi sentimenti di amore o di odio per qualunque cosa, questi saranno i tuoi aiutanti e soci per costruire le \textit{pāramī}. Il Buddha-Dharma non lo si trova procedendo, né retrocedendo, né stando fermi. Questo, Sumedho, è il tuo luogo del non dimorare''.

Ancora oggi mi fa venire la pelle d'oca.

Poche settimane più tardi, Ajahn Chah ebbe un ictus e non fu più in grado di parlare, né di camminare, né di muoversi. La sua carriera come insegnante orale si era interrotta. Quella lettera conteneva le sue ultime istruzioni. Ajahn Chah si rendeva ben conto di quanto fosse difficile e impegnativo avviare un monastero, proprio perché lui stesso l'aveva fatto molte volte. Si potrebbe pensare che quando dava consigli, lo facesse in termini di ``fa' questo, non fare quest'altro, ricordati sempre di\ldots{}''. Invece no, niente di tutto questo; non sarebbe stato da lui. Ajahn Chah disse semplicemente: ``Il Buddha-Dharma non lo si trova procedendo, né retrocedendo, né stando fermi''.

Nel suo monastero in Thailandia, Ajahn Chah era solito sedere su una panca di vimini posta all'aperto, sotto la sua capanna e ricevere visitatori dalle dieci di mattina sino a notte inoltrata. Ogni giorno. A volte anche sino alle due o le tre di notte.

Tra i vari modi in cui dava i suoi insegnamenti, a volte Ajahn Chah metteva alla prova i suoi visitatori, gli piaceva provocarli presentando loro degli enigmi, quesiti o rebus fatti apposta per frustrarli e spingerli a varcare i limiti della mente condizionata. Poneva domande tipo ``Questo bastone è lungo o corto?'', ``Da dove vieni e dove vai?'' oppure, come in questo caso, ``Non puoi andare avanti, non puoi andare indietro, non puoi rimanere fermo: dove vai?''. E quando rivolgeva domande simili, assumeva l'espressione di un cobra.

I più coraggiosi tentavano una risposta sensata: ``Vado di lato?''

``No, non ti puoi muovere neanche di lato''.

``Su o giù?''.

Continuava a provocare le persone che si sforzavano di trovare la risposta `giusta'. Quanto più tentavano una risposta intelligente o fantasiosa tanto più li bloccava con un ``No, no! Non è questo''.

Ajahn Chah cercava di spingere i suoi interlocutori fino ai limiti della mente condizionata affinché si potesse scorgere un barlume di incondizionato. Il principio del non dimorare è insopportabilmente frustrante per la mente pensante/concettuale, perché la mente ha costruito un edificio fatto di `me' e `te', di `qui' e `lì', di `passato' e `futuro' e di `questo' e `quello'.

Fin quando concepiamo la realtà in termini di sé e tempo, come `io' che sono da qualche parte e vado da qualche altra parte, ancora non abbiamo capito che andare avanti, andare indietro e stare fermi dipendono interamente dalle verità relative di io, spazio e tempo. In termini di realtà fisica, c'è un andare e venire; però c'è anche quel luogo di trascendenza dove non c'è né andare né venire. Pensateci. Dove possiamo andare veramente? Andiamo davvero da qualche parte? Dovunque andiamo noi siamo sempre `qui', non è vero? Per rispondere alla domanda ``Dove puoi andare?'' dobbiamo lasciar andare; lasciar andare l'io, il tempo e lo spazio. Abbandonando l'io, il tempo e lo spazio tutte le domande trovano risposta.

\section*{Antichi insegnamenti sul non dimorare}

Il principio del non dimorare è contenuto anche negli antichi insegnamenti Theravāda. Non si tratta solo di un'intuizione personale di Ajahn Chah o dell'eredità di un qualche lama Nyingmapa errante che vagando tra le montagne sia finito nel Nordest della Thailandia un centinaio di anni fa. Proprio nel Canone pāli il Buddha ne parla esplicitamente. Nell'\textit{Udāna} (la raccolta dei ``Versi ispirati'' del Buddha), si dice:

\begin{quote}
\itshape
``Esiste, o monaci, quello stato in cui non vi è terra, non vi è acqua, non vi è fuoco, non vi è aria, non vi è sfera dell'infinità dello spazio, non vi è sfera dell'infinità della coscienza, non vi è sfera della nullità, non vi è sfera della «né percezione né non percezione», né questo mondo né un altro mondo né entrambi, né il sole né la luna. Qui, monaci, io dico che non vi è giungere, non vi è andare e non vi è rimanere, non vi è crescita e non vi è decrescita. Esso non è fisso, non è mobile, non ha sostegno. Proprio questa è la fine della sofferenza''.\footnote{Ud VIII.1. Trad. di Francesco Sferra in Gnoli, R. (a cura di ), La rivelazione del Buddha, I Meridiani Mondadori, Milano 2001, p. 697.}
\end{quote}

\textit{Rigpa}, la consapevolezza non duale, è la conoscenza diretta di questo. È la qualità della mente che conosce, pur non dimorando da nessuna parte.

In un altro insegnamento dalla stessa raccolta si racconta di un eremita di nome Bāhiya che fermò il Buddha per la strada a Sāvatthí e gli disse: ``Venerabile signore, tu sei il Samaṇa Gotama. Il tuo Dharma è famoso in tutto il paese. Ti prego di darmi un insegnamento in modo che io possa comprendere la verità''.

Il Buddha rispose: ``Stiamo facendo la questua, Bāhiya. Non è questo il momento adatto''.

``La vita è incerta, venerabile signore. Non si sa mai quando moriremo; per favore, insegnami il Dharma''.

Questo dialogo si ripete tre volte. Per tre volte il Buddha dice la stessa cosa e Bāhiya risponde allo stesso modo. Alla fine il Buddha dice: ``Quando un \textit{Tathāgata }è pressato tre volte, deve rispondere. Ascolta attentamente, Bāhiya, e segui quello che ti dico: 

\begin{quote}
\itshape
``In ciò che è visto c'è solo ciò che è visto, in ciò che è udito c'è solo ciò che è udito, in ciò che è percepito c'è solo ciò che è percepito, in ciò che è conosciuto c'è solo ciò che è conosciuto. Pertanto, Bāhiya, dovrai vedere che in realtà non c'è nessuna cosa qui. Tu Bāhiya dovrai esercitarti così. Giacché, Bāhiya, in ciò che è visto c'è per te solo ciò che è visto, in ciò che è udito c'è solo ciò che è udito, in ciò che è percepito c'è solo ciò che è percepito, in ciò che è conosciuto c'è solo ciò che è conosciuto, tu vedrai che in realtà non c'è nessuna cosa lì . Quando, Bāhiya,vedrai che non c'è nessuna cosa lì, non ti ritroverai né nel mondo di qua, né nel mondo di là, né in nessun luogo tra i due. Solo questa è la fine della sofferenza''.\footnote{Ud I.10.}
\end{quote}

Al sentire queste parole Bāhiya fu immediatamente illuminato. Pochi minuti dopo egli venne ucciso da una vacca che correva. Aveva proprio ragione: la vita \textit{è} incerta. In seguito a Bāhiya venne conferito il titolo di `Il discepolo più veloce a comprendere l'insegnamento'. 

\section*{Non \`e corretto dire `dove'}

Cosa significa dire ``non c'è nulla \textit{lì}''? Significa parlare della sfera oggettiva; implica il riconoscimento che ``ciò che è visto è solo ciò che è visto''. Tutto qui. Ci sono forme, sagome, colori e così via, ma \textit{lì} non c'è nulla. Non c'è una vera sostanza, né solidità, né una realtà che esista autonomamente. Tutto ciò che c'è è la qualità dell'esperienza stessa. Niente di più e niente di meno. C'è soltanto il vedere, l'udire, il sentire, il percepire e il conoscere. Anche la mente che dà un nome a tutto questo è soltanto un'altra esperienza: ``lo spazio della sala di meditazione'', ``la voce di Ajahn Amaro'', ``ecco il pensiero `sto comprendendo questo?' ecco un altro pensiero `non sto comprendendo quest'altro?'\thinspace''.

C'è quello che è visto, udito, assaporato e così via, ma non c'è una quiddità, un'entità solida e indipendente cui questa esperienza si riferisca. 

Via via che questa profonda comprensione si fa più matura, non solo comprendiamo che non c'è nessuna cosa `là fuori', ma anche che non c'è un qualcosa di solido neanche `qui dentro', non c'è un'entità indipendente e fissa che sia colui che sperimenta. Questo significa parlare della sfera del soggettivo.

La pratica del non dimorare è un processo per cui la sfera dell'oggettivo e quella del soggettivo vengono completamente svuotate; per cui si vede che sia l'oggetto che il soggetto sono intrinsecamente vuoti. Se riusciamo a vedere che sia il soggettivo sia l'oggettivo sono vuoti, che non c'è un vero `qui dentro' o `là fuori', dove si trova il senso di `io', `mio' e `me'? Proprio come il Buddha aveva detto a Bāhiya: ``Non riuscirai a trovare il tuo io né nel mondo di questo $[$soggetto$]$, né nel mondo di quello $[$oggetto$]$, né in un qualunque punto fra i due''.

Nello \textit{Shuraṃgama Sūtra}, un testo molto citato nella tradizione Ch'an cinese, si legge un dialogo simile, ma molto più lungo, fra il Buddha e \=Ananda. Per molte pagine il Buddha chiede ad \=Ananda, in vari modi, se è in grado di definire esattamente dove sia la sua mente. Per quanto ci provi, \=Ananda però non lo può stabilire con certezza. Alla fine è obbligato a concludere così: ``Non riesco a trovare la mia mente da nessuna parte''. Ma il Buddha gli dice: ``Eppure la tua mente esiste, non è così?''.

Alla fine \=Ananda giunge alla conclusione che non è esatto dire `dove'.

Ah!

Questo è il punto in cui gli insegnamenti sul non dimorare cercano di portarci. Tutto il concetto, la costruzione del `dove', il concepire noi stessi come un'entità individuale che vive in questo specifico spazio-tempo è una supposizione. Soltanto frustrando i nostri giudizi abituali in questo modo siamo obbligati ad allentare la presa.

Questa visione delle cose stacca la spina, sottrae i puntelli e, soprattutto, scuote i nostri schemi di riferimento abituali. Questo è esattamente ciò che faceva Ajahn Chah quando chiedeva a qualcuno: ``Non puoi andare avanti, non puoi andare indietro, non puoi rimanere fermo, dove puoi andare?''. Stava indicando il luogo del non dimorare: la qualità senza tempo e priva di un io che è indipendente dal luogo. 

Sorprendentemente, recenti ricerche scientifiche hanno portato a una conclusione simile riguardo la natura fondamentale della materia. Nel mondo della fisica quantistica, adesso gli scienziati usano termini quali `la sorgente dell'essere' o `il mare della potenzialità' per riferirsi al livello primordiale di realtà fisica da cui si cristallizzano tutte le particelle e le energie, e in cui in seguito si dissolvono. Il principio dell'assenza di un luogo in questo ambito significa che il `luogo dove avviene qualcosa' non può essere definito veramente e che un singolo evento può avere effetti esattamente simultanei in luoghi (apparentemente) separati e lontani. Si può affermare con decisione che le particelle si disperdono nell'interezza del tempo e dello spazio.

Si è visto che termini quali `un unico luogo' e `luoghi separati' hanno un significato solo a certi livelli come convenzioni fittizie; al livello del campo ultimo, il mare della schiuma quantica, `luogo' non ha più alcun significato. Quando si raggiunge la sottile sfera subatomica, il concetto di un `dove' non ha più senso. Lì non c'è un lì. Sia che questo principio lo chiamiamo del non-dimorare o del non-luogo, è interessante e degno di nota che lo stesso principio vale sia per la sfera fisica sia per quella mentale. Gli intellettuali e i razionalisti si sentiranno confortati da questi parallelismi. 

Io mi avvicinai a questa investigazione meditativa durante un ritiro lungo nel nostro monastero nel quale facevo molta pratica da solo. Improvvisamente mi resi conto che anche se fossi riuscito a lasciar andare la sensazione dell'io, la sensazione di questo e quello e così via, qualunque fosse l'esperienza della realtà, essa era sempre `qui'. C'era ancora un `qui'. Per diverse settimane contemplai la domanda: ``Dove è il qui?''. Non usavo la domanda per ricavarne una risposta verbale, ma più per illuminare e favorire l'abbandono del mio attaccamento.

Quando si riconosce questo tipo di condizionamento si è già a metà strada, cioè riconoscere che non appena c'è un `qui', c'è un'impercettibile presenza di un `lì'. Così come creare un `questo' fa sorgere un `quello'. Non appena definiamo l'`interno' spunta l'`esterno'. É fondamentale riuscire a riconoscere queste sottili sensazioni di aggrapparsi; succede molto in fretta e a vari strati e livelli. 

Il semplice atto di comprendere questa esperienza illumina con la luce della saggezza ciò cui si sta aggrappando il cuore. Una volta che le contaminazioni si trovano sotto i riflettori, si sentono un po' nervose e a disagio. L'aggrapparsi opera meglio quando non guardiamo. Quando l'aggrapparsi è messo a fuoco dalla consapevolezza, non riesce a funzionare come si deve. In sostanza, l'aggrapparsi non si può aggrappare se c'è troppa saggezza in giro.

\section*{Acqua corrente immobile }

Ajahn Chah era solito fare la stessa domanda ``dove vai?'' per un po' di mesi. Quando ci si era abituati, cambiava domanda. Durante la sua carriera di insegnante ha posto una quantità di domande diverse. Prima che il suo stato di salute si aggravasse, prese a fare una serie di domande: ``Hai mai visto l'acqua immobile?''.

La persona assentiva ``certo, mi è capitato di vedere l'acqua immobile'', mentre probabilmente dentro di sé pensava ``che razza di domanda è questa!''. Ma apparentemente tutti erano molto rispettosi nei confronti di Ajahn Chah, dato che era uno dei più affermati insegnanti di meditazione thailandesi. 

Poi chiedeva: ``Bene, allora hai mai visto l'acqua corrente?''. Anche questa sembrava una cosa un po' strana da chiedere. La persona rispondeva: ``Sì, ho visto l'acqua corrente''. 

``Allora, hai mai visto l'acqua corrente immobile?''. In thailandese suonava \textit{nahm lai ning}. ``Hai mai visto \textit{nahm lai ning}?''. 

``No. Non l'ho mai vista''.

Gli piaceva provocare questo effetto di disorientamento.

Ajahn Chah allora spiegava che la natura della mente è immobile, eppure è corrente. É corrente, eppure è immobile. Usava la parola \textit{citta} per la mente che conosce, la mente della consapevolezza. Il citta di per sé è completamente immobile. Non fa movimenti, non è in relazione con ciò che sorge e cessa. É silenzioso e spazioso. Gli oggetti mentali, vista, suono, odore, sapore, tatto, pensieri ed emozioni, vi fluiscono attraverso. I problemi sorgono perché la chiarezza mentale è offuscata dalle impressioni sensoriali. Il cuore non allenato ricerca il piacevole, rifugge dallo spiacevole e, di conseguenza, si ritrova a combattere, alienato e sofferente. Contemplando la nostra esperienza possiamo fare una distinzione netta fra la mente che conosce (citta) e le impressioni sensoriali che vi fluiscono attraverso. Rifiutando di rimanere impastoiati nelle impressioni sensoriali, troviamo rifugio in quella qualità di immobilità, silenzio e spaziosità che è la natura stessa della mente. Questo atteggiamento di non interferenza lascia che tutto sia e non è disturbato da nulla. 

La capacità naturale di separare la mente (o essenza della mente, per usare un'espressione Dzogchen) dagli oggetti mentali si riflette chiaramente nella lingua pāli. Di fatto ci sono due verbi che entrambi significano `essere', che corrispondono al convenzionale o condizionato e all'incondizionato. Il verbo `\textit{hoti}' si riferisce a ciò che è condizionato e che scorre con il tempo. Si tratta delle normali attività e delle etichette che usiamo regolarmente per le varie impressioni sensoriali e, nella maggior parte dei casi, inconsapevolmente. Tutti siamo d'accordo, ad esempio, che l'acqua è bagnata, che il corpo è peso, che una settimana è fatta di sette giorni e che io sono un uomo.

Il secondo verbo, `\textit{atthi}', si riferisce alle qualità trascendentali dell'essere, dove `essere' non implica un divenire, il mondo del tempo e dell'identità, ma riflette l'incondizionato, la natura della mente immanifesta. Così, ad esempio, nel passaggio dell\textit{'Udāna} in cui si parla del non-nato, e di ``quella sfera dell'essere dove \ldots{} non c'è né il venire, né l'andare, né il rimanere fermi'', si usa sempre il verbo `atthi'. Esso indica un'essenza sovramondana e senza tempo. Il fatto che la distinzione fra mente (citta) e oggetti mentali faccia parte del linguaggio stesso, ci offre una riflessione sulla sua verità fondamentale, e serve a rammentarcene.

\section*{Non \`e corretto dire `chi' e `cosa' }

Per scoprire il luogo del non dimorare dobbiamo trovare un modo per lasciar andare il condizionato, il mondo del divenire. Abbiamo bisogno di riconoscere la forte identificazione con il nostro corpo e con la nostra personalità, con tutte le nostre credenze e con il modo in cui le consideriamo verità indiscutibili: ``Io sono Tizio; sono nato in questo luogo, ho una certa età, questo è il mio lavoro; io sono questo''.

Sembra così sensato ragionare in questo modo e, a un certo livello, è perfettamente logico; ma quando ci identifichiamo con questi concetti non c'è libertà. Non c'è spazio per la consapevolezza. Solo quando riconosciamo quanto seriamente e assolutamente assumiamo questa identità, ci apriamo alla possibilità della libertà. Sentiamo il sapore dell'io e ci accorgiamo che sa di sabbia e che ci appare molto reale. Riconoscendo questa sensazione, siamo in grado di sapere che ``questa è solo una sensazione''. La sensazione dell'io e del mio (\textit{ahaṃkāra} e \textit{mamankāra} in pāli) è trasparente come qualunque altra sensazione.

Quando la mente è calma e stabile, mi piace chiedermi ``chi sta osservando?'' o ``chi è consapevole?'' o ``chi è che conosce questo?''. Mi piace anche chiedermi ``cos'è che conosce?'', ``cos'è consapevole?'', ``cos'è che pratica la non-meditazione?''. Quando ci rivolgiamo domande di questo tipo il punto non è trovare una risposta; di fatto, se ci diamo una risposta verbale questa è sbagliata. Lo scopo di chiedersi `chi' o `cosa' è smontare ciò che diamo per scontato. Nella spaziosità della mente, le parole `chi' e `cosa' cominciano a suonare ridicole. Non c'è un vero `chi' o `cosa'. C'è solo la qualità del conoscere. Via via che continuiamo a lavorare in questo modo in maniera sempre più affinata, vediamo che quel senso dell'essere persona diventa sempre più trasparente; perde la sua solidità e il cuore è in grado di aprirsi e stabilizzarsi sempre di più. Sia la pratica della Vipassanā sia quella dello Dzogchen cercano di indicarci con molta chiarezza che noi rendiamo solido ciò che è intrinsecamente non solido. Questi metodi cercano di illuminare i modi sempre più sottili con cui ci aggrappiamo a ciò che creiamo attorno alle sensazioni di io, tempo, identità e luogo.

Inquadrando il nostro mondo in questo modo, inconsciamente lo rendiamo concreto. Domande come ``chi sei tu?'' implicano automaticamente la realtà dell'essere persona. Rispondere con il proprio nome è ragionevole a livello relativo. Ma il problema nasce quando permettiamo ciecamente al relativo di scivolare nell'assoluto. Crediamo che questo nome sia una cosa reale: ``Io sono una persona reale, io \textit{sono} Amaro''. Allo stesso modo, quando chiediamo ``che giorno è?'', questa domanda automaticamente implica la realtà del tempo. Se non c'è la presenza mentale, passiamo dal riconoscere una convenzione umana (provocata dal passaggio del nostro pianeta intorno al sole, da qualche parte nel mezzo di questa galassia) al creare una verità assoluta e universale.

Il corollario di questa non-creazione di solidità nella sfera delle percezioni e delle convenzioni (in caso si abbia paura di perdere tutte le forme di realtà) è che non dobbiamo creare o, in qualche modo, ottenere il Dharma per rimpiazzare le basi note che stiamo perdendo. Quando smettiamo di creare gli oscuramenti, il Dharma è sempre qui.

Non appena vediamo dove hanno luogo le forme sottili e quelle grossolane di attaccamento e quella stretta mortale si allenta, quando ci rammentiamo che ``eccola qua, ecco la presa, la contrazione di identità'', c'è apertura e spaziosità. Questa libertà del cuore viene dal riconoscere il modo in cui abitualmente creiamo cose e poi le accettiamo come reali. Quando ciò è veramente visto e conosciuto, la contrattura che mantiene la morsa non regge più e al suo posto si manifesta il Dharma.

\section*{Non \`e corretto dire `quando' }

Il tempo è un altro ambito in cui dovremmo notare un sottile attaccamento. Possiamo sperimentare una quiete nella consapevolezza e un conseguente senso di chiarezza e spaziosità, ma possiamo anche avere la sensazione netta che tutto questo sta avvenendo \textit{adesso}. In questo caso, senza accorgercene, abbiamo trasformato quell'`adesso' in una qualità solida.

Il processo del lasciar andare avviene strato dopo strato. Quando cade uno strato ci sentiamo elettrizzati e pensiamo: ``Stupendo, adesso sono libero. Questo spazio aperto è meraviglioso''. Ma appena cominciamo a comprendere che c'è qualcosa che non va, che c'è ancora viscosità nel sistema, notiamo la solidificazione del tempo e la limitazione del presente che abbiamo creato.

C'è una poesia sul tempo del Sesto Patriarca Zen che mi piace citare: 

\begin{quote}
\itshape
In questo momento non c'è nulla che viene in essere.

In questo momento non c'è nulla che cessa di essere.

Così, in questo momento, non ci sono né nascita né morte da far cessare.

Così, la pace assoluta è questo momento presente.

Anche se è solo questo momento, non c'è limite a questo momento,

E qui è la delizia eterna.
\end{quote}

La nascita e la morte dipendono dal tempo. Qualcosa apparentemente nato nel passato, che è vivo adesso, morirà in futuro. Una volta lasciato andare il tempo, e se lasciamo anche andare l'`essere una cosa', vediamo che non ci può essere una `cosa' reale che viene in essere o muore; c'è solo la quiddità del presente. In questo modo non c'è nascita o morte che debba essere fatta finire.

Ecco perché questo momento è assolutamente pieno di pace; è fuori del tempo, \textit{akāliko.}

Usiamo locuzioni tipo `questo momento', ma non sono esatte perché ci danno ancora l'impressione che il presente sia una minuscola frazione di tempo. Perché anche se è solo un momento, il presente è senza limiti. Lasciando andare la struttura di passato e futuro, comprendiamo che questo presente è un oceano infinito, e la conseguenza di questa comprensione è vivere nell'eterno, nel senza-tempo. Non abbiamo bisogno di solidificare e concepire il presente in una distinzione fra passato e futuro, è la sua stessa vastità che si auto-sostiene.

Stiamo parlando di abbandonare l'aggrapparsi a un livello molto sottile, una pratica che richiede un bel po' di duro lavoro spirituale, pronto e attento. Quando vediamo che la nostra mente è rimasta impigliata in qualcosa, possiamo applicare la classica tecnica vipassanā, colpiscila con l'impermanenza, il non-sé e la sofferenza, il vecchio un, due, tre. Se abbiamo un buon senso di \textit{anattā}, lo facciamo a pezzi con un `non io, non mio', e lo finiamo. È importante però ricordare che l'attaccamento è estremamente scaltro. Stiamo lì a goderci il successo e non ci accorgiamo che è come se stessimo giocando ad acchiapparella. C'è qualcuno che ci sta raggiungendo da dietro mentre guardiamo la persona che abbiamo appena abbattuto. Il compagno sta per colpirci. Abbiamo appena lasciato andare l'attaccamento al tempo quando l'attaccamento alle opinioni si scaglia all'impazzata. Lasciamo andare questo ed ecco il senso del qui; poi c'è il corpo\ldots{} l'aggrapparsi si presenta in molte forme, in molti modi e noi dobbiamo vederli tutti.

\section*{Olio e acqua}

Ajahn Chah diceva che fino a quando non conobbe il suo maestro Ajahn Mun, non aveva mai compreso veramente che la mente e i suoi oggetti esistono come qualità separate e che, confondendole e mischiandole, non riusciva mai a trovare la pace. Invece, ciò che ricevette da Ajahn Mun, nei tre brevi giorni che trascorse con lui, fu il senso chiaro che c'è la mente che conosce, \textit{poo roo}, c'è colui che conosce, e ci sono gli oggetti del conoscere. Sono come lo specchio e le immagini che vi si riflettono. Lo specchio non è né abbellito né corrotto dalla bellezza o dalla bruttezza degli oggetti che vi appaiono. E nemmeno si annoia. Anche quando non vi è riflesso nulla, è completamente equanime e sereno. Questa fu una grande intuizione per Ajahn Chah, e da quel momento divenne un punto centrale nella sua pratica e nell'insegnamento.

Era solito paragonare la mente e i suoi oggetti all'olio e all'acqua contenuti in una stessa bottiglia. La mente che conosce è come l'olio e le impressioni sensoriali sono come l'acqua. La nostra mente e la vita sono così agitate e turbolente che l'olio e l'acqua si mischiano, per questo sembra che la mente che conosce e i suoi oggetti siano una sostanza unica. Ma se lasciamo che il sistema si calmi, allora l'olio e l'acqua si separeranno, perché per loro stessa natura non si mescolano.

C'è la consapevolezza, la mente di Buddha e ci sono le impressioni dei pensieri, il mondo sensoriale, e tutti gli altri modelli della coscienza. La consapevolezza e le impressioni si separano naturalmente l'una dalle altre; non c'è bisogno che si faccia niente perché questo avvenga. Non sono intrinsecamente amalgamati e si separano se glielo permettiamo. 

A questo punto, possiamo veramente vedere che la mente è una cosa e gli oggetti mentali un'altra. Possiamo vedere la vera natura della mente, l'essenza della mente, che conosce l'esperienza e in cui si svolge tutta la vita; e possiamo vedere che quella qualità trascendente è priva di un rapporto con l'individualità, lo spazio, il tempo e il movimento. Tutti gli oggetti del mondo, le persone, le nostre abitudini e gli stati mentali, appaiono e scompaiono in quello spazio.

\section*{Respirare e camminare}

Lo sforzo di fare una netta distinzione fra la mente che conosce e gli oggetti mentali è pertanto molto importante per la nostra pratica. L'attenzione al respiro è un buon modo per lavorare con questa comprensione. Semplicemente notate la sensazione del respiro mentre lo percepite. Il respiro si muove, ma ciò che conosce il respiro non si muove. 

Forse possiamo cominciare concentrandoci sullo spazio alla fine dell'espirazione e poi alla fine dell'inspirazione. Notiamo che lì c'è una pausa, uno spazio. Ma, se estendiamo la visione, cominciamo a notare che quella spaziosità e quella calma in realtà sono sempre lì. Mentre l'aria entra ed esce, c'è un'eterna spaziosità della mente che non è mai ostruita dal movimento del respiro.

Possiamo anche estendere questa pratica alla meditazione camminata. Se stiamo fermi in piedi, con gli occhi aperti o chiusi, possiamo notare che tutte le sensazioni del corpo sono conosciute nella mente. La sensazione dei piedi sul pavimento, il corpo in piedi, la sensazione dell'aria e così via, sono tutte sostenute e conosciute nella mente. Può darsi che ci vogliano alcuni minuti per raggiungere questo punto, ma se facciamo uno sforzo, ben presto sperimenteremo il senso di una mente stabilizzata. A quel punto semplicemente lasciamo che il corpo inizi a camminare.

Generalmente quando camminiamo andiamo da qualche \textit{parte}; questo può complicare le cose. In realtà non c'è nessuna differenza sostanziale fra andare da qualche parte e non andare da nessuna parte. La meditazione camminata è molto utile in questo senso, semplifica di molto le cose. Sappiamo che non stiamo assolutamente andando da nessuna parte. È un esercizio deliberatamente e completamente inutile in termini di cercare di arrivare da qualche parte. 

Lavorare in meditazione con il corpo in movimento è un'opportunità per sperimentare il corpo che cammina senza andare da nessuna parte. Quando il corpo cammina a passo misurato, cominciamo a vedere che sebbene il corpo si stia muovendo, la mente che conosce il corpo non si muove. Il movimento non si applica alla consapevolezza. Ci sono i movimenti del corpo, ma la mente che conosce i movimenti non si sta muovendo. C'è quiete, eppure c'è flusso. Il corpo fluisce, le percezioni fluiscono, eppure c'è quiete. Non appena la mente vi si aggrappa e pensiamo di stare andando da qualche parte, l'olio e l'acqua si sono di nuovo mischiati. Ci sono `io' che sto andando `da qualche parte'. Ma in quel momento di riconoscimento ``guarda, la quiete della mente è assolutamente non toccata dal movimento del corpo'' conosciamo questa qualità di acqua corrente immobile.

C'è una comprensione della libertà. Ciò che si muove è non-sé. Ciò che si muove è l'aspetto del flusso e del cambiamento. E il cuore spontaneamente prende rifugio in questa qualità di spaziosità, di quiete e di apertura che conosce ma non è impigliata.

Personalmente trovo che la meditazione a occhi aperti sia molto utile a questo. Con gli occhi aperti siamo più stimolati a esercitare la stessa qualità che normalmente si ha solo nella meditazione camminata. Se teniamo gli occhi aperti e accogliamo lo spazio della sala, vediamo il via vai delle persone, i corpi che si cullano leggermente al vento, la luce che cambia, il calare del sole al pomeriggio.

Possiamo permettere a tutto questo di andare e venire e di essere contenuto in quello spazio di conoscenza dove c'è un'esperienza cosciente sia della verità convenzionale sia della verità ultima. C'è la visione ultima del `né persone, né tempo, né spazio', del conoscere senza tempo e della luminosità. Poi ci sono le convenzioni: io e te, qui e lì, seduta e camminata, andare e venire. Le due verità sono completamente fuse tra loro; una non ostruisce l'altra. Questo è un modo per comprendere direttamente che il non dimorare non è una qualche filosofia astrusa, ma qualcosa che possiamo gustare e valutare.

Nel momento in cui capiamo veramente questo principio il cuore comprende che ``il corpo si sta muovendo, il mondo va e viene, ma non sta assolutamente andando da nessuna parte''. La nascita e la morte finiscono qua.

E non abbiamo bisogno di sedere immobili o di camminare lentamente per risvegliarci a questa intuizione profonda. Possiamo correre, o addirittura giocare a tennis, e trovare la stessa qualità. È presente tanto mentre stiamo fisicamente immobili quanto mentre ci muoviamo in fretta, o magari stiamo sfrecciando sull'autostrada. 
