
\dropcaps{A}{GLI INIZI DEGLI ANNI} '80 ero un giovane monaco che viveva e praticava al Wat Suan Mokkh, un monastero della foresta nel sud della Thailandia fondato da Ajahn Buddhadāsa, uno dei più grandi studiosi e insegnanti di meditazione thailandesi degli ultimi 50 anni. Ho un profondo apprezzamento nei confronti del lignaggio Theravāda di questo paese, per la sua totale fedeltà agli insegnamenti originali del Buddha, così come ci sono stati tramandati attraverso il Canone pāli.

Mi trovavo a Suan Mokkh da poco tempo e una mattina, mentre facevamo colazione nel refettorio all'aperto, mi sorpresi nel vedere nelle vicinanze, in cima a un piedistallo alto quasi due metri, un busto di Avalokiteshvāra, il dio Mahāyāna della compassione. Mi chiesi cosa ci facesse una divinità Mahāyāna proprio lì, in un monastero Theravāda. Le due tradizioni si sono separate nell'India del nord 2000 anni fa. In quel momento credevo, e mi sbagliavo, che non si fossero mai più parlate, come una coppia astiosa senza figli dopo il divorzio. 

A quell'epoca i rapporti tra le varie scuole buddhiste non erano dei migliori. I maestri Zen raramente comunicavano con i lama tibetani, i monaci Theravāda thailandesi e birmani non avevano molti contatti con il resto del mondo. Quell'immagine di Avalokiteshvāra a Wat Suan Mokkh era ancora più misteriosa alla luce di così tanti secoli di separazione. Ancora più sconcertante fu la scoperta di un'intera costruzione, all'interno del monastero, chiamata il teatro spirituale (una reminiscenza del \textit{Lupo della steppa}, mi dissi) dove erano ospitati dipinti originali e copie dell'arte theravāda, zen, tibetana e addirittura occidentale. Tutto questo rifletteva l'apertura mentale di Ajahn Buddhadāsa, il cui apprezzamento per la verità era ancora più profondo della sua lealtà nei confronti di un qualunque lignaggio storico.

Quel busto di Avalokiteshvāra però, era un'altra questione. Qualcuno mi aveva detto che era stato ritrovato il secolo scorso sepolto nelle vicinanze di Chaiya, una città a pochi chilometri da Suan Mokkh e che risaliva al nono secolo. Era chiaro quindi che un migliaio di anni fa il buddhismo Mahāyāna era fiorito in questa regione. Infatti gli storici affermano che il Theravāda e il Mahāyāna avevano convissuto in Thailandia, così come il Vajrāyāna e l'induismo, fino al xiv secolo. A seguito di cambiamenti politici, il Theravāda divenne dominante, e così è rimasto fino ad oggi.

Non è quindi strano che nella attuale tradizione Thailandese della Foresta si riscontri una comprensione del Dharma che presenta forti parallelismi con i princìpi del Buddhismo Mahāyāna e Tibetano. Ad esempio, la dottrina Mahāyāna sulla natura di Buddha ci dice che la nostra vera essenza è la consapevolezza che non nasce e non muore. In insegnamenti più tardi della scuola Dzogchen si sono sviluppate tecniche di meditazione che permettono al praticante di riconoscere questa natura e dimorarvi. Ajahn Amaro (il cui nome significa ``il senza morte''), una volta ha detto che questo insegnamento può essere considerato l'inno della tradizione Thailandese della Foresta.

Ajahn Chah, un maestro thailandese considerato il capostipite del lignaggio cui appartiene Ajahn Amaro (nonché l'insegnante di Ajahn Sumedho e di Jack Kornfield), spesso si riferiva a ``Colui che conosce'' come a ciò che ci indica la saggezza inerente alla consapevolezza stessa. Ajahn Buddhadāsa dice che ``la vacuità e la presenza mentale sono la stessa cosa''. Ajahn Maha Boowa, un contemporaneo di Ajahn Chah con cui aveva condiviso lo stesso maestro, Ajahn Mun, dice dell'impermanenza: ``Questo svanisce, quest'altro svanisce, ma ciò che conosce lo svanire non svanisce\ldots{} Tutto ciò che resta è semplice consapevolezza, totalmente pura''.

Questo concetto della consapevolezza intrinseca, in quanto aspetto della natura che non muore, è generalmente considerato una innovazione Mahāyāna, eppure non è raro ritrovarlo anche nella tradizione della Foresta Thailandese. Risalendo alla sua genesi, si scoprono tracce di questa idea già nel Canone pāli, il testo fondamentale adottato dai Theravāda, anche se i riferimenti sono poco frequenti e ambigui. Uno dei pregi di \textit{Piccola barca, grande montagna} è che Ajahn Amaro vi cita molti di questi riferimenti, per i quali fornisce spiegazioni chiare e ineccepibili. Nei circoli ortodossi in Birmania e nello Sri Lanka, ad ogni modo, si tratta di un concetto assolutamente eretico, dato che la consapevolezza (o coscienza, \pali{viññāṇa}) è considerata impermanente. 

La questione è di particolare rilievo di questi tempi perché negli ultimi 10 anni molti insegnanti e studenti di \pali{vipassanā} occidentali hanno studiato gli insegnamenti di maestri Dzogchen. Tra gli insegnanti tibetani che sono stati di particolare aiuto ai praticanti di \pali{vipassanā}, citiamo il defunto Tulku Urgyen Rinpoche, suo figlio Tsoknyi Rinpoche e il defunto Nyoshul Khen Rinpoche. Ispirati dalle teorie e dalle tecniche di questo lignaggio, molti praticanti di \pali{vipassanā} cercano di conciliare i concetti Dzogchen con le proprie radici Theravāda. Poiché i discorsi di Ajahn Amaro pubblicati in questo testo rappresentano un contributo importante a questo dialogo, riteniamo interessante specificare in quali occasioni sono stati pronunciati.

Nell'autunno del 1997 gli insegnanti di Dharma allo Spirit Rock Meditation Center (tra i quali Ajahn Amaro) tennero una riunione sulla possibilità di invitare Tsoknyi Rinpoche a condurre un ritiro presso il loro centro. Quando invitiamo un insegnante di un'altra tradizione siamo soliti affiancarlo a un insegnante dello Spirit Rock, allo scopo di evitare che i nostri studenti rimangano confusi da termini e concetti diversi. Mentre si discuteva su quale insegnante potesse condurre il ritiro assieme a Tsoknyi Rinpoche, qualcuno fece il nome di Ajahn Amaro e un insegnante approvò la proposta con entusiasmo dicendo: ``Sì, il tulku e il bhikkhu!''. E così avvenne.

Il dialogo fra tradizioni spirituali diverse è irto di ostacoli, anche nell'ambito di uno stesso lignaggio. Durante le migliaia di anni dalla morte del Buddha, scuole diverse si sono sviluppate autonomamente e c'è il rischio di incomprensioni. Verso la fine degli anni '70 del secolo scorso si incontrarono un maestro Zen coreano e un apprezzato rinpoche tibetano. Naturalmente l'incontro era stato organizzato dai rispettivi studenti occidentali con il proposito di favorire uno scambio fra due tradizioni molto lontane una dall'altra. Il maestro Zen iniziò con una sfida tipicamente Zen; tenendo un'arancia tra le mani, con decisione chiese: ``Che cos'è questo?'' Il maestro Tibetano rimase in silenzio sgranando i grani della sua \textit{mala}. Il maestro Zen chiese di nuovo: ``Che cos'è questo?'' il rinpoche si rivolse al suo traduttore chiedendogli a bassa voce: ``Non conoscono le arance nel suo paese?''.

Ancora oggi le divisioni tra le varie scuole buddhiste non sono state ricomposte. Recentemente ho ascoltato la registrazione di una conversazione che verteva su alcuni importanti argomenti di Dharma fra un insegnante Theravāda occidentale e un maestro Dzogchen Tibetano, con l'ausilio di un ottimo interprete. Si aveva l'impressione che i due insegnanti provenissero da due pianeti diversi. Fui prima meravigliato, poi frustrato e alla fine divertito dalla loro incapacità di trovare un terreno comune nonostante l'evidente buona volontà di tutti e tre. Continuavano a non comunicare a causa della difficoltà della traduzione in termini di linguaggio, cultura e filosofia del Dharma.

Insomma, non era affatto certo che il ritiro con Ajahn Amaro e Tsoknyi Rinpoche sarebbe riuscito. Sono entrambi insegnanti carismatici e sicuri di sé, abituati a condurre ritiri autonomamente. Un accoppiamento di questo tipo non era mai stato tentato prima. Mi chiedevo se questa fosse la prima volta in cui un insegnante Theravāda e uno Vajrāyāna si sarebbero seduti sulla stessa pedana dai tempi dell'Università di Nālandā nell'India del Nord (che fu distrutta dagli invasori musulmani nel xii secolo).

C'erano da risolvere importanti questioni di gerarchia. Rinpoche in genere insegna seduto su un trono, un alto sedile riccamente ornato allo scopo di incutere il rispetto che chi ascolta dovrebbe provare nei confronti degli insegnamenti, che sono cosa diversa dall'insegnante. Ajahn Amaro si sarebbe sentito a suo agio seduto su quel palco decorato con i vivaci arazzi di seta tibetana? Oppure il monaco Theravāda si sarebbe limitato alla consueta pedana di legno? Questo però avrebbe creato un problema perché il Vinaya, il codice di disciplina dei monaci, proibisce espressamente che un monaco insegni se un laico è seduto più in alto di lui. Gli organizzatori si sentirono molto sollevati quando Ajahn Amaro spiegò che non era insolito che un monaco della tradizione della foresta insegnasse da un trono e che sarebbe stato lieto di farlo in quell'occasione. 

Contrariamente a tutti i nostri timori, il ritiro fu un grande successo. Partecipando a quel ritiro come studente, ebbi modo di apprezzare entrambi gli insegnanti. Rinpoche era molto bravo nel presentare gli insegnamenti Dzogchen in maniera comprensibile a degli occidentali, come si può capire leggendo il suo libro \textit{Carefree Dignity}. I discorsi serali di Ajahn Amaro, pubblicati in questo libro, rappresentavano un bellissimo complemento e chiarivano gli insegnamenti di Rinpoche a degli studenti di \pali{vipassanā}. Ogni sera io sedevo in ammirazione quando Ajahn Amaro teneva i suoi discorsi che vertevano sulla filosofia e gli aspetti tecnici della meditazione, nei quali citava a braccio lunghi passi dei discorsi del Buddha. Il suo stile fresco e quasi estemporaneo era una manifestazione di virtuosismo. Non meno ammirevole era il suo contegno. Molti di noi notarono il suo costante buon umore. Tsoknyi Rinpoche lo ricordò alla fine del ritiro, mentre esprimeva la sua gratitudine per il contributo di Ajahn Amaro: ``Non ho mai incontrato una persona simile. Il suo Vinaya è molto rigoroso. In genere, quando il Vinaya è rigoroso, dentro di sé il monaco è molto rigido. Invece lui è molto rilassato nel suo intimo e sempre felice''.

Nel lignaggio di Ajahn Chah un insegnante non dovrebbe preparare un discorso di Dharma; piuttosto, questi è stimolato a confidare sul proprio senso del momento presente e a intuire dall'atmosfera e dal pubblico quali sono le parole più appropriate. Credo che Ajahn Amaro abbia seguito questa regola durante il ritiro con Tsoknyi Rinpoche, e che siamo molto fortunati ad avere le registrazioni dei suoi straordinari discorsi che il contesto gli aveva ispirato. Per l'erudizione, l'umorismo e la profondità, essi rappresentano una testimonianza unica e accurata dell'atmosfera di quel ritiro. Possa il loro messaggio condurre tutti coloro che li leggono direttamente alla loro natura di Buddha e alla immensa libertà della Grande Perfezione Naturale.

\bigskip

{\raggedleft\par Guy Armstrong \\
SPIRIT ROCK MEDITATION CENTER \\
Luglio 2002 \par}
