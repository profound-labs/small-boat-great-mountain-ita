
\begingroup\setlength{\parindent}{0em}\setlength{\parskip}{0.8em}

\noindent\textit{I canti tibetani sono tratti dalle traduzioni di Tony Duff ed Erik Pema Kunsang}



\section{Prendere rifugio}
\vspace*{-1em}\noindent\dropcaps{T}{herav\=ada}

Namo tassa bhagavato arahato sammāsambuddhassa \\
Omaggio al Beato, Nobile e Perfettamente Risvegliato \\
(\textit{tre volte})

Buddhaṃ saranaṃ gacch\=ami \\
Dhammaṃ saranaṃ gacch\=ami \\
Saṅghaṃ saranaṃ gacch\=ami

Al Buddha vado per rifugio \\
Al Dhamma vado per rifugio \\
Al Sangha vado per rifugio \\
(\textit{tre volte})

\section{Prendere il rifugio in via ordinaria e suscitare il bodhicitta}
\vspace*{-1em}\noindent\dropcaps{T}{ibetano}

Fino a quando diverrò illuminato, prendo rifugio \\
Nel Buddha, nel Dharma e nell'Assemblea Suprema; \\
Possano i meriti derivanti dalla mia generosità e dalle altre virtù \\
Portare alla Buddhità a beneficio di tutti gli esseri. \\
(\textit{tre volte})

\section{Prendere il rifugio in via straordinaria}
\vspace*{-1em}\noindent\dropcaps{T}{ibetano}

Namo: l'essenza, vuota, il Dharmakāya; \\
la natura, la chiarezza, il Sāmbhogakāya; \\
La compassione, molteplice, il Nirmānakāya; \\
In questo io prendo rifugio fino all'illuminazione. \\
(\textit{tre volte})

\section{La dedicazione del merito di Nāgārjuna}
\vspace*{-1em}\noindent\dropcaps{T}{ibetano}

Che con questo merito si possa ottenere dunque l'onniscienza, \\
Avendo sconfitto i nemici, le azioni errate, \\
Che noi possiamo liberare gli esseri viventi dall'oceano dell'esistenza, \\
Con le sue tempestose onde di nascita, vecchiaia, malattia e morte.

\section{Riflessioni sulla condivisione del bene}
\vspace*{-1em}\noindent\dropcaps{T}{herav\=ada}

Grazie alla bontà che sorge dalla mia pratica, \\
possano i miei insegnanti spirituali e guide di grande virtù, \\
mia madre, mio padre e i miei famigliari, \\
il sole e la luna, \\
e tutti i capi virtuosi del mondo, \\
possano gli dèi supremi e le forze malvagie, \\
gli esseri celesti, gli spiriti guardiani della Terra \\
e il signore della morte; \\
possano quelli che ci sono amici, indifferenti o ostili, \\
possano tutti gli esseri ricevere il bene della mia vita, \\
possano ottenere presto la triplice benedizione \\
e realizzare il senza-morte. \\
Grazie alla bontà che sorge dalla mia pratica, \\
e grazie a questo atto di condivisione, \\
possano cessare rapidamente tutti i desideri e gli attaccamenti, \\
e tutti gli stati mentali nocivi. \\
Fino a quando realizzerò il Nibbāna, \\
in ogni tipo di nascita, \\
possa io avere una mente retta, \\
con consapevolezza e saggezza, austerità e vigore. \\
Possano le forze dell'illusione non attecchire, \\
né indebolire il mio proposito. \\
Il Buddha è il mio rifugio eccellente, \\
insuperabile è la protezione del Dhamma, \\
il Buddha solitario è il mio nobile signore, \\
il Sangha è il mio sostegno supremo. \\
Grazie al potere supremo di loro tutti, \\
possano oscurità e illusione dissiparsi. \\

\section{Le parole del Buddha sulla gentilezza amorevole}
\vspace*{-1em}\noindent\dropcaps{T}{herav\=ada}

Questo è ciò che va fatto \\
Da colui che ha praticato la bontà, \\
E che conosce il sentiero della pace: \\
che si possa essere capaci e retti, \\
schietti e gentili nel parlare. \\
Umili e non superbi, \\
Contenti e facilmente soddisfatti. \\
Non gravati da doveri e frugali nei modi, \\
pacifici e calmi, saggi e abili, \\
Non orgogliosi e con poche pretese. \\
Che non si commetta la minima azione \\
Che i saggi poi riproverebbero. \\
Augurando: in gioia e in sicurezza, \\
che tutti gli esseri siano felici. \\
Qualsiasi essere vivente, \\
debole o forte, nessuno escluso, \\
il grande o il potente, medio, corto o piccolo, \\
il visibile e l'invisibile, \\
quelli che vivono vicino e quelli che vivono lontano, \\
quelli nati e quelli che stanno per rinascere, \\
che tutti gli esseri siano in pace! \\
Che nessuno inganni l'altro, \\
o disprezzi qualunque essere in qualunque condizione. \\
Che nessuno, con rabbia o cattiveria, \\
Desideri nuocere l'altro. \\
Come una madre protegge con la sua vita \\
Il suo figlio, il suo unico figlio, \\
Così, con un cuore sconfinato, \\
si amino e rispettino tutti gli esseri viventi; \\
Irradiando gentilezza sul mondo intero: \\
Diffondendola in alto fino ai cieli \\
E in basso nelle profondità; \\
Ovunque e senza limiti, \\
Liberi dall'odio e dalla malevolenza. \\
Che si stia in piedi, camminando, seduti o sdraiati, \\
Liberi da torpore, \\
Si sostenga questo senso di raccoglimento. \\
Questa è chiamata la sublime dimora. \\
Non aggrappandosi a opinioni rigide, \\
colui che ha un cuore puro e una visione chiara, \\
liberato da tutti i desideri sensoriali, \\
non più nascerà in questo mondo. 

\section{Il mantra Vajrasatva delle cento sillabe}
\vspace*{-1em}\noindent\dropcaps{T}{ibetano}

OM VAJRASATVA SAMAYAM ANUPALAYA \\
VAJRASATVA TVENOPATISHTHA DRDHO ME BHAVA \\
SUTOSHYO ME BHAVA \\
SUPOSHYO ME BHAVA \\
ANURAKTO ME BHAVA \\
SARVASIDDHI MEM PRAYACCHA \\
SARVA KARMASU CA ME \\
CITTAṂ SHREYAM KURU HŪṂ \\
HA HA HA HA HOH \\
BHAGAVAN SARVATATHĀGATA VAJRA MĀ ME MUÑCA \\
VAJRĪ BHAVA MAHĀSAMAYASATVA ĀH

\section{Il canto Vajra del primo Tsoknyi Rinpoche}
\vspace*{-1em}\noindent\dropcaps{T}{ibetano}

Non divagare, non divagare, metti in guardia la consapevolezza; \\
Sulla strada della distrazione è in agguato Mara. \\
Mara è la mente, l'attaccamento a ciò che piace e a ciò che non piace, \\
Perciò guarda nell'essenza di questo incantesimo, libero dalla fissazione dualistica. \\
Comprendi che la tua mente è purezza primordiale non costruita; \\
Non c'è nessun Buddha in nessun luogo, guarda il tuo vero volto; \\
Non c'è nient'altro da cercare, riposa nel tuo luogo; \\
La non-meditazione è la perfezione spontanea, quindi prenditi il trono reale.

\newpage

\section{Emanazione delle dimore sublimi}
\vspace*{-1em}\noindent\dropcaps{T}{herav\=ada}

Dimorerò pervadendo un quarto con un cuore soffuso di gentilezza amorevole- \\
Così il secondo quarto, così il terzo, così il quarto- \\
Così di sopra e di sotto, intorno e dappertutto, \\
e a tutti come a me stesso. \\
Dimorerò pervadendo tutto quanto il mondo \\
Con un cuore soffuso di gentilezza amorevole; \\
abbondante, eccelsa, incommensurabile, \\
senza ostilità e senza malevolenza.

Dimorerò pervadendo un quarto con un cuore imbevuto di compassione- \\
Così il secondo quarto, così il terzo, così il quarto- \\
Così di sopra e di sotto, intorno e dappertutto, \\
e a tutti come a me stesso. \\
Dimorerò pervadendo tutto quanto il mondo \\
Con un cuore soffuso di compassione; \\
abbondante, eccelsa, incommensurabile, \\
senza ostilità e senza malevolenza.

Dimorerò pervadendo un quarto con un cuore imbevuto di gioia- \\
Così il secondo quarto, così il terzo, così il quarto- \\
Così di sopra e di sotto, intorno e dappertutto, \\
e a tutti come a me stesso. \\
Dimorerò pervadendo tutto quanto il mondo \\
Con un cuore soffuso di gioia; \\
abbondante, eccelsa, incommensurabile, \\
senza ostilità e senza malevolenza.

Dimorerò pervadendo un quarto con un cuore imbevuto di equanimità- \\
Così il secondo quarto, così il terzo, così il quarto- \\
Così di sopra e di sotto, intorno e dappertutto, \\
e a tutti come a me stesso. \\
Dimorerò pervadendo tutto quanto il mondo \\
Con un cuore soffuso di equanimità; \\
abbondante, eccelsa, incommensurabile, \\
senza ostilità e senza malevolenza.

\section{Aspirazione al Bodhicitta}
\vspace*{-1em}\noindent\dropcaps{T}{ibetano}

Possa sorgere il prezioso Bodhicitta \\
Dove non è sorto, \\
E dove è sorto possa non diminuire \\
Ma aumentare sempre di più.

\section{Evocazione straordinaria del Bodhicitta}
\vspace*{-1em}\noindent\dropcaps{T}{ibetano}

Ho: perché si collochino tutti gli esseri viventi, ovunque nello spazio \\
Sul gradino di un Buddha, \\
Userò l'\textit{upadesha} della Grande Perfezione \\
Per realizzare il rigpa Dharmakāya che sorge spontaneamente.

\section{Riflessioni sul benessere universale}
\vspace*{-1em}\noindent\dropcaps{T}{herav\=ada}

Che io dimori nel benessere \\
Libero da afflizione, \\
Libero da ostilità \\
Libero da malevolenza, \\
Libero dall'ansia, \\
e che io possa conservare il benessere in me.

Che tutti dimorino nel benessere, \\
Liberi da ostilità, \\
Liberi da malevolenza, \\
Liberi dall'ansia, \\
E che possano conservare il benessere in loro.

Che tutti gli esseri siano liberati dalla sofferenza \\
E possano non essere privati \\
Della buona fortuna che hanno ottenuto.

Tutti gli esseri sono possessori delle loro azioni, \\
E ne ereditano gli effetti. \\
Il loro futuro nasce da tali azioni, \\
Si accompagna a tali azioni, \\
E gli effetti saranno la loro casa. \\
Tutti i tipi di azione, \\
Utili o dannose, \\
Di questi atti \\
Saranno gli eredi.

\section{L'origine dipendente e la cessazione}
\vspace*{-1em}\noindent\dropcaps{T}{herav\=ada}

Con la condizione dell'ignoranza sorgono le formazioni.

Con la condizione delle formazioni sorge la coscienza.

Con la condizione della coscienza sorge materia/mente.

Con la condizione di materia/mente sorgono i sei sensi.

Con la condizione dei sei sensi sorge il contatto.

Con la condizione del contatto sorge la sensazione.

Con la condizione della sensazione sorge la brama.

Con la condizione della brama sorge l'attaccamento.

Con la condizione dell'attaccamento sorge il divenire.

Con la condizione del divenire sorge la nascita.

Con la condizione della nascita, sorgono dunque vecchiaia e morte, dispiacere, lamento, dolore, cordoglio e disperazione.

Così è l'origine di tutta questa massa di sofferenza.

A questo punto, con il dissolvimento senza residuo, con la cessazione e con l'assenza proprio dell'ignoranza, vi è la cessazione, il non-sorgere delle formazioni.

Con la cessazione, il non-sorgere delle formazioni vi è la cessazione, il non-sorgere della coscienza.

Con la cessazione, il non-sorgere della coscienza vi è la cessazione, il non-sorgere di materia/mente.

Con la cessazione, il non-sorgere di materia/mente vi è la cessazione, il non-sorgere dei sei sensi.

Con la cessazione, il non-sorgere dei sei sensi vi è la cessazione, il non-sorgere del contatto.

Con la cessazione, il non-sorgere del contatto vi è la cessazione, il non-sorgere della sensazione.

Con la cessazione, il non-sorgere della sensazione vi è la cessazione, il non-sorgere della brama.

Con la cessazione, il non-sorgere della brama vi è la cessazione, il non-sorgere dell'attaccamento.

Con la cessazione, il non-sorgere dell'attaccamento vi è la cessazione, il non-sorgere del divenire.

Con la cessazione, il non-sorgere del divenire vi è la cessazione, il non-sorgere della nascita.

Con la cessazione, il non-sorgere della nascita, allora smettono di sorgere, cessano vecchiaia e morte, dispiacere, lamento, dolore, cordoglio e disperazione.

Così è la cessazione, il non-sorgere di tutta questa massa di sofferenza.

\section{I cinque precetti}
\vspace*{-1em}\noindent\dropcaps{T}{herav\=ada}

\begin{enumerate}

\item \textit{Pāṇātipāta veramaṇī sikkhāpadaṃ samādiyāmi.}

Prendo il precetto di astenermi dal prendere la vita di qualsiasi creatura vivente.

\item \textit{Adinnādānā veramaṇī sikkhāpadaṃ samādiyāmi.}

Prendo il precetto di astenermi dal prendere ciò che non mi è stato dato.

\item \textit{Kāmesu micchācārā veramaṇī sikkhāpadaṃ samādiyāmi.}

Prendo il precetto di astenermi da una cattiva condotta sessuale.

\item \textit{Musavada veramaṇī sikkhāpadaṃ samādiyāmi.}

Prendo il precetto di astenermi da discorsi falsi e nocivi.

\item \textit{Surāmeraya-majja-pamādatthānā veramaṇī sikkhāpadaṃ samādiyāmi}

Prendo il precetto di astenermi dall'assumere bevande e droghe intossicanti, che portano alla mancanza di attenzione.
\end{enumerate}

Precettore:

\textit{Imāni pañca sikkhāpadāni} \\
\textit{Silena sugatiṃ yanti} \\
\textit{Sīlena bhogasampadā} \\
\textit{Sīlena nibbutiṃ yanti} \\
\textit{Tasmā sīlam visodhaye.}

Questi sono i cinque precetti: \\
sīla è la fonte della felicità, \\
sīla è la fonte della vera ricchezza, \\
sīla è l'origine della pace, \\
pertanto, si purifichi sīla.

\endgroup
