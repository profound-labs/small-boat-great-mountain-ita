
\dropcaps{A}{jahn Amaro \`e un autentico discepolo} del Buddha e un continuatore del lignaggio di insegnamenti della tradizione Theravāda.

I primi anni della sua vita sono stati quelli di una persona qualsiasi, anche se fin da giovane il suo interesse per la spiritualità lo spinse a intraprendere un viaggio in Thailandia dove, grazie ai suoi meriti karmici, entrò ben presto in contatto con un insegnante buddhista. Per molti anni ha praticato la meditazione ed è stato discepolo devoto di insegnanti buddhisti nella tradizione dei Monaci della Foresta. La sua pratica era incentrata sulla ``ruota della rinuncia nella pratica di meditazione''.

La pratica principale di Amaro è stata l'applicazione diretta delle Quattro Nobili Verità: conoscere la sofferenza, eliminare l'origine della sofferenza, realizzare la cessazione della sofferenza e seguire il cammino che vi conduce. Queste quattro verità racchiudono gli insegnamenti principali del Buddha e, di questi -- sofferenza, origine, cessazione e cammino -- nella contemplazione dei dodici anelli dell'origine dipendente, l'obiettivo principale è l'eliminazione dell'origine della sofferenza.

Io ho instaurato un legame karmico con Ajahn Amaro durante le mie visite negli Stati Uniti, avendo avuto modo di incontrarci in varie occasioni e di insegnare insieme allo Spirit Rock Center in California. Sono certo che si tratta di una persona che ha studiato e praticato il cammino Theravāda in modo molto profondo; ha anche conosciuto vari maestri Vajrāyāna, compreso Dudjom Rinpoche, dimostrando un apprezzamento scevro di pregiudizi verso gli insegnamenti Vajrāyāna.

Dal mio punto di vista, il Buddha ha insegnato i cosiddetti Tre Veicoli, ciascuno dei quali contiene un percorso completo affinché tutti gli esseri senzienti possano eliminare le emozioni negative, ovvero attaccamento, avversione, ignoranza, orgoglio e invidia, con tutte le loro 84.000 proliferazioni e variazioni. È pertanto assolutamente possibile che chiunque percorra uno di questi tre cammini senza pigrizia o indolenza, possa raggiungere lo stesso livello del Buddha Sakyamuni.

Ritengo addirittura possibile che una persona possa praticare i tre veicoli contemporaneamente senza che sorgano conflitti. Questo avviene spesso nella tradizione buddhista Tibetana in cui molti praticanti seguono i tre veicoli sia separatamente che unificati in un solo metodo.

Di questi tempi, in cui assistiamo a un crescente interesse per la pratica buddhista in tutto il mondo, ritengo importante che le persone arrivino a comprendere le qualità e gli aspetti specifici di ognuno di questi tre veicoli. Ciascun individuo quindi, libero da pregiudizi e con chiarezza, sarà libero di adottare quello che più si avvicina alla propria sensibilità -- sia che si tratti di un solo veicolo, sia di una combinazione di tutti e tre. È proprio per questo che mi sento di invitare tutti a studiare gli aspetti fondamentali dei tre veicoli del Buddha.

Fra i tanti insegnanti di Dharma, ciò che più mi fa apprezzare Ajahn Amaro è il suo rispetto per questo principio non settario e perché ne incarna la comprensione.

\bigskip

{\raggedleft\par Drubwang Tsoknyi Rinpoche \\
PUTUO SHAN ISLAND \\
Ottobre 2002 \par}