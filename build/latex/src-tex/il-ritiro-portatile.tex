
\dropcaps{L}{a fine di un ritiro} è un momento speciale. In questi dieci giorni abbiamo creato un rifugio sicuro osservando i precetti e rispettando reciprocamente gli spazi personali. Questo rifugio ci ha permesso di praticare il Dharma in maniera intensiva e di conoscere direttamente le leggi della natura. Ma, come tutto nella vita, questo prezioso periodo trascorso insieme sta per finire. Comincerà il processo delicato e inevitabile per cui ci disperderemo nelle realtà più diverse. Qualcuno cerca di prolungare gli ultimi momenti di silenzio e riflessione, mentre altri probabilmente non vedono l'ora di schizzare fuori. Qualsiasi cosa la vostra mente stia facendo, non interferite. Cercate di rilassarvi e di godere di questi ultimi momenti insieme.

A differenza di altre tradizioni, compresa la Dzogchen, conosciute per i loro complessi rituali, la scuola Theravāda prevede rituali molto semplici. Non si distribuiscono vino e cibo, né ci sono balli in costume e la banda. Ciò nonostante, i nostri rituali sono molto belli nella loro semplicità e profondità. L'usanza di chiudere un ritiro con una cerimonia di benedizione e di commiato ci aiuta nella transizione fra la vita durante il ritiro e la vita di lavoro, famiglia, piatti sporchi, la macchina e così via. Noi condividiamo un diverso livello di esperienza che è fatto di partecipazione, di contatto ed è denso di significato. Il rituale di cantare i rifugi e i precetti aiuta a chiudere il cerchio; li abbiamo cantati il primo giorno e li cantiamo adesso prima di andarcene. Adesso però, forse siamo tutti consapevoli che non c'è veramente un inizio e una fine. L'integrità e la completezza sono sempre qui.

\section*{Anatomia di un rituale}

\vspace*{-0.8em}
Questa cerimonia prevede alcuni simboli e principi base. I simboli comprendono un lungo filo di cotone, una ciotola d'acqua, alcune gocce di olio profumato, una candela di cera d'api e una immagine del Buddha. Il Buddha rappresenta il principio spirituale primordiale, mentre il filo rappresenta la qualità della purezza, quel principio spirituale fondamentale che si manifesta nel mondo della forma. La cerimonia inizia avvolgendo tre volte il filo intorno all'immagine del Buddha e poi tre volte intorno alla ciotola d'acqua.

Poi il filo passa fra le mani del Sangha ordinato e percorre tutta la sala, passando per le mani di tutti i presenti. Da un punto di vista fisico unisce e simbolicamente rappresenta la purezza intrinseca che ci unisce in quanto esseri umani e lega il nostro cuore al principio del Buddha; ci rammenta che l'essenza del nostro cuore \textit{è} il principio del Buddha.

La cerimonia si incentra nella ciotola d'acqua. Dei quattro elementi, terra, acqua, fuoco e aria, l'acqua rappresenta la coesione e simboleggia compassione. É il simbolo di ciò che tiene unito tutto. Nell'acqua si spruzza un olio profumato speciale, con sandalo e altre sostanze preziose.

Al momento dei canti, accendo la candela sopra la ciotola. La cera sgocciola nell'acqua e, quando la cera tocca l'acqua, i quattro elementi sono riuniti. L'elemento terra è rappresentato dalla cera della candela; il fuoco è la fiamma; l'aria è ciò che alimenta il fuoco e la cera che sgocciola; da ultimo c'è l'acqua. I quattro elementi sono inoltre fusi con la presenza degli insegnamenti del Buddha, incarnati nel suono dei canti e con l'elemento finale della coscienza, l'attenzione che prestiamo a tutta la rappresentazione.

Per lo meno a livello simbolico i quattro elementi della nostra natura fisica sono permeati dalla presenza del Buddha-Dharma, cui tutti i partecipanti sono collegati attraverso il filo.

La prima parte dei canti è una invocazione. Evoca i brahmā, i devatā, gli spiriti della terra, i nāgā (i draghi celesti) e tutti gli esseri dell'universo che potrebbero essere ben disposti verso questa assemblea. È un invito a che vengano a benedirci con la loro presenza, a renderci artefici della nostra vita e a rafforzarci nel lavoro spirituale.

Possiamo considerare questo aspetto della cerimonia in vari modi. Alcune persone non accettano l'idea degli esseri invisibili come gli angeli, i brahmā, i Buddha e i bodhisattva. Se è così, sentitevi liberi di considerarlo come un appello a tutte le forze benigne interiori che sono intrinseche alla nostra natura fondamentale, tutti gli stati luminosi e salutari del nostro essere. Potremmo pertanto dire che il primo livello dell'invocazione coinvolge gli esseri esterni radiosi e benigni, mentre il secondo livello gli `esseri' interiori, cioè quegli stati della mente che sono parimenti risplendenti, nobili e potenti. C'è poi un terzo livello dell'invocazione in cui il Dharma è semplicemente benedetto dalla sua stessa natura; qui si ha il grande gesto spontaneo il cui il cuore realizza la propria essenza sacra.

Comunque la vediamo, ricordate che interno ed esterno sono convenzioni fittizie, non realtà ultime. Quindi, quando invochiamo le forze benedicenti dall'interno o dall'esterno, o semplicemente le realizziamo, in sostanza stiamo invitando quelle qualità raggianti ad affacciarsi alla coscienza.

I canti sono la recitazione degli insegnamenti del Buddha, versi di lode nei suoi confronti e versi recitati da lui per la protezione, la guarigione e per disperdere le forze malefiche. Io li canterò in pāli mentre voi, l'assemblea del Dharma, ascoltate in silenzio. Anche se non riuscite a comprendere molte delle parole, non lasciate che la mente rimanga impastoiata nella ricerca di un significato, ma rilassatevi. Lo scopo è di dare una opportunità, attraverso l'azione del suono e del rituale, di ricevere le benedizioni e che queste si manifestino e fioriscano.

In questa cerimonia, la qualità dell'atteggiamento corretto è cruciale, sia da parte delle persone che guidano il rituale, sia di quelle che ricevono le benedizioni. Se desideriamo onestamente di essere benedetti, dobbiamo `accalappiare' ciò che è benefico e aiutarlo a crescere nel giardino del Dharma nel nostro cuore. Anche se là fuori ci sono una miriade di esseri benevolenti o di virtù raggianti, l'apertura del cuore, proprio come un campo fertile, è necessaria per `acchiappare' le benedizioni.

Se il cuore si mantiene aperto le vere benedizioni ci pioveranno addosso, emergeranno da dentro di noi e si manifesteranno nel nostro cuore. Sedete quindi e apritevi. Non cercate di fare nulla (ovviamente). Lasciate semplicemente che il suono fluisca liberamente attraverso la vostra consapevolezza.

Una volta terminati i canti riavvolgeremo il filo. Poi si taglia il filo, secondo la tradizione tibetana, e a ciascuno ne viene dato un pezzetto per legarlo al polso in una sorta di `manetta del Dharma', un promemoria visibile di questo ritiro, della nostra connessione primordiale e del nostro impegno nel Buddha Dharma.

Una volta mi capitò di guidare questa cerimonia durante una benedizione per un matrimonio nella Città dei Diecimila Buddha, un importante monastero di tradizione cinese vicino a Ukiah, in California. Uno dei monaci chiese di assistere alla cerimonia dato che non aveva mai visto una benedizione né nella tradizione Settentrionale né in quella Meridionale.

Alla fine della cerimonia, disse: ``Credevo che i Theravādin non avessero questa roba! Ti rendi conto che è Tantra puro?''

La risonanza dei quattro elementi esternamente (nella ciotola) e internamente (nel nostro corpo), assieme alla trasformazione dell'energia, è qualcosa che la cerimonia può realizzare se è presente la qualità della retta attitudine.

\section*{Rifugi e precetti}

\vspace*{-0.8em}
Se c'è una domanda prevedibile che le persone fanno alla fine di un ritiro, questa è: ``Come posso mantenere la pratica una volta finito il ritiro e tornato nel mondo di violenza, avidità, bambini che muoiono di fame, cattiva assistenza sanitaria e mai un parcheggio che si trovi?''. Questa è una domanda stupenda perché ci fa portare l'attenzione al saggio desiderio di integrare la nostra pratica spirituale con la vita quotidiana. Perché si sviluppi la vita spirituale non c'è bisogno di trasformarsi in un `ritiro-dipendente' che malvolentieri tollera i periodi in cui deve convivere con la famiglia e il lavoro solo per conquistarsi il tempo e le risorse per poter andare a un altro ritiro. Probabilmente è più utile impegnarsi a costruire le basi per una condotta etica nella vita di tutti i giorni. Sulla base di questo, i cinque precetti non sono offerti tanto come rigidi comandamenti quanto come linee guida per la vita. In pratica sono strategie che ci aiutano a esistere e funzionare in modo più armonioso.

Potrei consigliarvi molti modi per sostenere lo spirito del ritiro dopo che siete partiti, che vanno dalla pratica quotidiana al creare uno spazio sacro in un angolo di casa vostra. Ma il suggerimento in assoluto più importante che vi posso dare è questo: oltre a interiorizzare i tre rifugi (come detto nel Capitolo Sei) vorrei invitarvi ad apprezzare profondamente i cinque precetti. Osservateli come una pratica devozionale, come una pratica di presenza mentale, come una pratica di concentrazione, come una pratica di condotta. Tutti questi elementi sono contenuti in questi semplici principi. Prendere i precetti è un atto che rinnova l'intenzione di essere in sintonia con voi stessi e di essere il più possibile gentili con voi e con il mondo che vi circonda. 

Quando i visitatori si recano nei monasteri thailandesi, c'è l'usanza di prendere i rifugi e i precetti come un semplice e consueto promemoria. In Occidente li prendiamo all'inizio e alla fine dei ritiri, negli intensivi di un giorno e anche prima delle riunioni del Sangha. La presa dei rifugi non è definitiva, come se una volta presi questi ci trasformassero per sempre; niente affatto. Si tratta di principi che necessitano di essere ripetuti, coltivati ed esplorati costantemente. La saggezza può svilupparsi solo in una mente che è costantemente ri-orientata verso la verità e radicata nel non-io.

\section*{Due tipi di persone}

\vspace*{-0.8em}
Nelle scritture Theravāda si parla di due tipi di persone: \textit{puggala} e \textit{manussa}. Essere un puggala significa che pur essendo dotati di un corpo umano potreste non sentirvi completamente umani, potreste funzionare interiormente come un animale o come uno spirito famelico. Se siete un manussa, siete veramente umani. Nella cosmologia buddhista le sfere dell'esistenza sono divise in paradisi (deva), gli dèi gelosi (\textit{asura}), gli animali, gli spiriti famelici (\textit{peta}), gli abitanti dell'inferno (\textit{niraya}) e gli esseri umani. Essere nati nella cerchia degli umani delle sei sfere significa che siete un manussa. Un manussa è uno che vive per lo meno secondo i cinque precetti. Vale a dire che la caratteristica saliente di uno che è completamente umano è la qualità della virtù, della condotta impeccabile.

Trovo che questa sia una riflessione utile che possiamo verificare da soli: quando ci comportiamo in modo sgradevole, egoista, crudele o avido, come ci sentiamo? In quelle circostanze siamo sub-umani; non siamo in armonia con la vita; stiamo male con noi stessi. C'è uno squilibrio. Il cuore non può aprirsi in mezzo al caos.

Possiamo anche vedere da soli cosa succede quando ci comportiamo in modo saggio e gentile. Come ci si sente \textit{allora}? Stiamo bene con noi stessi, e c'è un senso di armonia con tutte le cose. Il cuore è aperto e ricettivo all'intera panoplia della vita. Possiamo essere ignoranti in molti modi, e soggetti a tutti i tipi di sofferenza, ma avere questa sensibilità e questa nobiltà di condotta essenziali è sinonimo di autentica umanità.

\section*{Una legge naturale}

\vspace*{-0.8em}
I cinque precetti non sono un'invenzione del Buddha. Fanno parte dell'ordine naturale. Non vengono imposti come un'idea buddhista, né sono specifici della tradizione buddhista. Ogni paese al mondo ha leggi che consentono agli esseri umani di funzionare in modo libero e armonioso. Sono le leggi che riguardano il rispetto della vita umana, la proprietà, un corretto uso della sessualità e l'onestà. Il Buddha insisteva sul fatto che sono aspetti intrinseci alla condizione umana. Se uccidiamo, se ci appropriamo indebitamente di qualcosa, se ci approfittiamo del prossimo (attraverso la sessualità o una condotta amorale), se inganniamo o siamo aggressivi, se feriamo con le parole, ne conseguirà inevitabilmente dolore. I versi iniziali del \textit{Dhammapada }recitano: ``Se parli o agisci con mente corrotta, ne seguirà dolore come le ruote del carro seguono il bue che lo tira''. Il Buddha chiamava questi precetti \textit{pakati-sīla}, la virtù naturale o autentica. Sono contrapposti a \textit{panatti- sīla}, i comportamenti etici prescritti che sono il prodotto delle usanze locali e delle religioni e norme specifiche di alcune professioni.

Mi piace paragonare i cinque precetti al manuale per l'utente di una nuova auto: ``Complimenti! Lei è adesso il fortunato proprietario di una vita umana. Le voglio presentare la sua nuova autovettura''. Magari non somigliano tanto a un manuale, sono piuttosto come dei segnali stradali, tipo: CURVA PERICOLOSA, o VIETATO L'ACCESSO – SENSO UNICO o RALLENTARE. Cercate di comprendere i precetti in questo modo. Sono i segnali stradali della nostra vita di esseri umani. Ci aiutano a vedere che ``la vita è veramente in \textit{questa} direzione, non in \textit{quella}''. 

Questi segnali ci proteggono dai pericoli. Ci avvisano quando ci sono ostacoli e aiutano il cuore a non perdere la strada. Avrete notato che se non seguite le indicazioni, tendete a perdervi, i problemi si moltiplicano e vi ritrovate afflitti da frustrazione e tensione. Quando però prestate attenzione e seguite la legge e i segnali stradali, c'è flessibilità, c'è attenzione al tempo e al luogo e in genere siamo in grado di arrivare a destinazione. I precetti dovrebbero essere intesi proprio allo stesso modo. Li prendiamo e li usiamo come utili indicazioni nelle regioni della vita dove più facilmente ci perdiamo, dove c'è una maggiore carica emotiva: questioni che riguardano la vita e la morte, il possesso e la proprietà, la sessualità, l'onestà e l'inganno, la parola e la comunicazione. 

\section*{Il quinto precetto}

\vspace*{-0.8em}
È interessante che quando il Buddha descrive i precetti morali, spesso non fa menzione del quinto. Il Buddha non sempre etichettava il precetto contro l'uso di intossicanti come intrinsecamente morale. Quando dico questo, c'è sempre qualcuno che mostra un vivo interesse! Però il punto è che quando la mente è in uno stato di non attenzione è molto più facile cadere a testa in giù nelle prime quattro zone a rischio di quando la mente è sveglia, equilibrata e non intossicata. Per continuare con l'analogia del guidatore: pensate solo al numero di incidenti provocati da persone sotto l'influenza dell'alcool o di altri intossicanti. Può darsi quindi che non sperimenteremmo l'inevitabile risultato karmico negativo come accadrebbe invece, per esempio, nel caso di una bugia volontaria, ma il precetto contro l'uso di intossicanti è compreso in tutti e cinque perché fa da perno a tutti gli altri; quando si allenta, le ruote cominciano a oscillare.

Da parte mia mi piace incoraggiare la comprensione che il quinto precetto: ``Prendo il precetto di trattenermi dal consumare bevande e sostanze intossicanti che portano alla disattenzione'' invita ad astenersi dal consumare le sostanze, non solo dall'intossicarsi. In genere si preferisce pensare: ``Una birra di quando in quando, o un bicchiere di vino a cena non va contro i precetti, vero?'' Ad essere sincero, direi che non è così.

Osservare l'astinenza è un atto di grande gentilezza verso voi stessi e verso gli altri cui diamo l'esempio. Non vi sto chiedendo di essere intransigenti o di diventare dei fanatici, ma ci può essere di grande aiuto prendere un impegno inequivocabile. È come dire: ``La presenza mentale è un bene prezioso e fragile, perché metterla a rischio o indebolirla?'' Personalmente cerco di incoraggiare una osservanza rigorosa dei precetti, compreso il trattenersi dall'uso di intossicanti. L'unica ragione è il mio amore per voi e per tutti gli esseri. Vi accorgerete che osservare i precetti in questo modo è il sostegno più solido per tutti gli ambiti della pratica buddhista.

\vspace*{-0.8em}
\section*{S\={\i}la \`e un altro termine per felicit\`a}

\vspace*{-0.8em}
I cinque precetti non riguardano solo la moralità. Sono anche gli strumenti per una notevole presenza mentale. C'è forse un segnale che ci indica quando cominciamo a deviare da rigpa verso marigpa, dalla chiara consapevolezza verso la non attenzione? Non è come se ci fosse una piccola spia sul cruscotto che si accende in caso di contaminazione o di confusione. Non è come quando create un documento al computer e la macchina vi indica il nome del file e il percorso, la data in cui l'avete scritto e così via. ``Questo è uno stato di avidità di terzo grado, generato alle h. 15.41 del 1.6.02'' ``Questa è condizione di auto-inganno\ldots{}'' non sono etichettati in questo modo.

Ma quando diamo il cuore ai precetti e li rispettiamo davvero, \textit{loro} ci informano, ci mettono in guardia. Quando il cuore stoltamente scivola nell'inconsapevolezza, nell'attrazione illusoria e nell'avversione, c'è un campanello d'allarme nel sistema che permette al cuore di svegliarsi prima che si perda di vista la purezza innata, prima che gli stati negativi si siano materializzati, e prima che ci ficchiamo nei pasticci. Per ritornare all'analogia del guidatore, sono come le bande sonore ai lati dell'autostrada che fanno vibrare le ruote quando ci avviciniamo troppo al guardrail. ``Ops. Un colpo di sonno. Come è successo? Meglio che mi svegli o mi metterò nei guai e non ce la farò''.

Dopo aver recitato i precetti, la persona che li imparte canta: 

\vspace*{-0.8em}
\begin{quote}
\itshape
Questi sono i cinque precetti: \\
sīla è la fonte della felicità, \\
sīla è la fonte della vera ricchezza, \\
sīla è l'origine della pace, \\
pertanto, si purifichi sīla.
\end{quote}

\vspace*{-0.8em}
Questo è tutto ciò che serve per essere felici. Prendiamo questi princìpi di gentilezza e virtuosità verso il cuore e lasciamo che ci guidino. La culla del Dharma è dentro di noi.

È un ritiro portatile.
