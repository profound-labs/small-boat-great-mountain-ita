
\dropcaps{L}{a traduzione di alcuni termini} può essere molto interessante, specialmente al limite dove le parole si estinguono. Ricordo che alcuni anni fa stavo guardando un glossario di una raccolta di insegnamenti Vedānta. Dove in sanscrito c'era una parola sola, la spiegazione in inglese era lunga un paragrafo. Per quanto riguarda le aree sottili della coscienza, l'inglese è piuttosto scarso. La nostra lingua va bene per le emozioni, abbiamo un mucchio di parole per ogni sfumatura delle emozioni, ma per i dettagli degli aspetti più articolati della coscienza è difficile trovare parole che veramente rendano l'idea in maniera accurata e completa, senza disorientarci. 

\section*{Prestare attenzione al senza-morte}

Negli insegnamenti Theravāda uno dei modi in cui il Buddha parlava della liberazione è molto simile al principio centrale dello Dzogchen. Per quanto ne sappia, entrambe le tradizioni insistono sul fatto che a un certo punto dobbiamo lasciar andare tutto e risvegliarci alla presenza del Dharma. Non bisogna attaccarsi neppure agli stati più salutari. Questo principio viene tradotto in vari modi, ma quello che sembra essere il più preciso è `prestare attenzione al senza-morte'. In pāli, l'ultima parola è `\textit{amatadhātu}'.

Un importante brano dei Sutta (A 3.128) presenta una conversazione fra due monaci anziani del Buddha. Il Venerabile Sariputta era il primo discepolo del Buddha, il più autorevole per saggezza e per i risultati ottenuti nella meditazione. Sebbene non fosse dotato di poteri psichici, era il grande maestro dei meditanti. L'altro discepolo anziano del Buddha, il Venerabile Anuruddha, era dotato di sorprendenti poteri psichici. Era quello che più di tutti aveva il dono dell'`occhio divino'; poteva vedere tutte le diverse dimensioni.

I due discepoli formavano una coppia interessante. La debolezza di Sariputta era il grande dono di Anuruddha. Ad ogni modo, poco prima dell'illuminazione, Anuruddha andò da Sariputta e disse: ``Con l'occhio divino purificato e perfetto io sono in grado di vedere tutte le 10.000 forme del sistema universo. La mia meditazione è assolutamente stabile, la mia consapevolezza è solida come una roccia. La mia energia è incessante e il mio corpo è completamente rilassato e calmo. Eppure il mio cuore non è completamente libero da efflussi e confusione. Dov'è che sbaglio?''.

Sariputta rispose: ``Amico mio, la tua capacità di vedere nelle 10.000 forme del sistema universo è collegata alla tua presunzione. La tua energia costante, la tua acuta consapevolezza, la tua calma fisica e la focalizzazione della tua mente hanno a che fare con la tua irrequietezza. Inoltre il fatto che ancora non hai liberato il cuore dagli \textit{āsava} e dalle contaminazioni dipende dalla tua ansia. Sarebbe bene, amico mio, se invece di occuparti di queste cose, tu volgessi la tua attenzione verso l'elemento del senza-morte''. (Sia detto per inciso, il Canone pāli abbonda di umorismo di questo tipo, anche se è molto simile allo humour inglese e a volte è facile che ci sfugga). Così, naturalmente, Anuruddha disse ``ti ringrazio molto'' e se ne andò. Poco tempo dopo, ottenne la completa illuminazione. Si tratta di un umorismo molto sottile.

Eppure il punto di questa conversazione è piuttosto serio. Fino a quando diciamo ``guarda come sono complicati i miei problemi'' oppure ``guarda la mia forza di concentrazione'', rimaniamo impastoiati nel \textit{saṃsāra}.\textit{ }In sostanza Sariputta\textit{ }aveva detto al suo compagno: ``Sei così occupato con tutto il daffare e i suoi effetti, così occupato con tutte queste proliferazioni che non sarai mai libero. Stai guardando nella direzione sbagliata. Ti stai rivolgendo verso \textit{l'esterno}, stai osservando gli oggetti di meditazione \textit{là fuori}, il sistema universo dalle 10.000 forme che è \textit{là fuori}. Sposta il tuo sguardo al contesto dell'esperienza e prenditi cura piuttosto dell'elemento del senza-morte''. 

Anuruddha aveva bisogno soltanto di spostare leggermente la propria attenzione per giungere a questa comprensione: ``La questione non sono soltanto tutti gli oggetti affascinanti o tutte le cose nobili di cui mi sono occupato, tutto questo è condizionato, nato, composto e destinato a morire. Quello che manca è il Dharma senza tempo. Guarda dentro, espandi il tuo sguardo. Prenditi cura del senza-morte''.

Ci sono anche altri punti nei Sutta (ad es. M 64.9 e 9.36) in cui il Buddha parla dello stesso processo riguardo allo sviluppo della concentrazione e dell'assorbimento meditativo. Egli addirittura spiega che quando la mente è nel primo \textit{jhāna}, nel secondo jhāna, nel terzo jhāna, via via fino ai jhāna superiori senza forma, possiamo osservare quegli stati e riconoscere che sono tutti condizionati e dipendenti. Questo, dice, è il vero sviluppo della saggezza: la consapevolezza di riconoscere la natura condizionata di uno stato, allontanarsi da esso, e prestare attenzione al senza-morte, anche quando questo stato è ancora presente. Quando la mente è concentrata e molto pura e luminosa, possiamo riconoscere che questo stato è condizionato, dipendente, estraneo, o qualcosa che è vuoto, vacuo. C'è la presenza mentale per riflettere sulla verità che tutto questo è condizionato e quindi grossolano, ma c'è l'elemento del senza-morte. E nel rivolgersi verso l'elemento del senza morte, il cuore è liberato.

In un certo senso è come guardare un quadro. In genere l'attenzione è rivolta alla figura e non allo sfondo. Oppure immaginate di essere in una stanza con qualcuno seduto su una sedia. Se vi trovate dall'altra parte della stanza, il vostro sguardo non si rivolgerà verso lo spazio davanti o di fianco a quella persona, ma la vostra attenzione sarà per la figura sulla sedia, giusto? Allo stesso modo, se avete mai disegnato una figura su una parete, in genere c'è un punto con un difetto o una sbavatura. E dove va l'occhio quando guardiamo la parete? Si dirige direttamente verso la pecca. Esattamente allo stesso modo, i nostri organi di percezione sono programmati per focalizzarsi sulla figura piuttosto che sullo sfondo. Anche se un oggetto è simile allo sfondo, come ad esempio la luce illimitata, abbiamo tuttavia bisogno di sapere come si fa a distoglierci da quell'oggetto.

Sia detto per inciso, questo spiega perché nei circoli di meditazione buddhista si è spesso messi in guardia contro gli stati profondi di assorbimento. Quando ci si trova in uno di questi stati, può risultare molto difficile sviluppare la visione profonda, e lo è ancor di più quando la mente è meno concentrata. Lo stato di assorbimento è una copia così verosimile della liberazione che sembra essere autentica. Quindi pensiamo: ``È qui, perché cercare più oltre? Questo va bene''. Ci lasciamo imbrogliare e, di conseguenza, sprechiamo l'opportunità di prenderci cura del senza-morte.

In termini cosmologici, il luogo migliore per la liberazione è nella sfera umana, dove troviamo un buon mix di sofferenza e beatitudine, di felicità e infelicità. Se invece ci troviamo nella sfera dei deva, è difficile ottenere la liberazione, perché è come stare a un'interminabile festa dove non dobbiamo nemmeno pulire dopo. Semplicemente ci crogioliamo nel boschetto di Nandanā, i deva lasciano cadere chicchi d'uva nella nostra bocca, mentre noi ci muoviamo leggeri circondati dai nostri esseri adoranti preferiti che ci fluttuano intorno. E, naturalmente, non c'è molto agonismo; voi lì siete sempre la star dello spettacolo. Nei regni di Brahmā è ancora peggio. Chi è che ha voglia di ritornare alla vecchia spregevole terra e avere a che fare con il rimborso delle tasse e i permessi edilizi?

Questa cosmologia è un riflesso del nostro mondo interiore. Quindi i regni di Brahmā sono l'equivalente degli stati di assorbimento senza forma. Uno dei maggiori maestri di meditazione thailandesi, il Venerabile Ajahn Tate, era così dotato per la concentrazione che, non appena si sedeva in meditazione, entrava direttamente negli \textit{arūpa-jhāna}, gli stati di assorbimento senza forma. Gli ci vollero dodici anni dopo aver incontrato il suo insegnante, il Venerabile Ajahn Mun, per allenarsi a \textit{non} farlo e per mantenere la concentrazione a un livello in cui potesse sviluppare la visione profonda. In quegli stati senza forma si sta proprio bene. È facile che ci si chieda: ``A che serve coltivare la saggia riflessione o investigare la natura dell'esperienza? L'esperienza stessa è assolutamente deliziosa, perché preoccuparsi?''. Il motivo per cui ci preoccupiamo è che si tratta di stati su cui non ci si può contare. Sono inaffidabili e non sono nostri. Probabilmente non sono molti quelli che hanno il problema di rimanere bloccati negli arūpa-jhāna, ciò nonostante, è utile comprendere perché si discute di questi principi e gli si dà importanza.

L'atto di prestare attenzione al senza-morte è pertanto una pratica spirituale nodale, ma non complicata. Semplicemente distogliamo la nostra attenzione dagli oggetti mentali per portarla sul senza-morte, il non-nato. Non si tratta di un programma imponente di ricostruzione. Non è che si debba \textit{fare} grandi cose. È molto semplice e naturale. Ci rilassiamo e notiamo qualcosa che è sempre stato qui, come notare lo spazio in una stanza. Non notiamo lo spazio perché non attira la nostra attenzione, perché non è affascinante.

Allo stesso modo, siccome il \textit{nibbāna} non ha una caratteristica, né colore, né sapore, né forma, noi non ci accorgiamo che è proprio qui. Gli organi di percezione e l'attività della mente che attribuisce nomi operano sulle forme; è lì che vanno prima. Pertanto tendiamo a perderci ciò che è sempre qui. Di fatto, proprio perché non è dotato di vita, lo spazio è a un tempo il peggiore e il miglior esempio, ma a volte è utile usarlo.

\section*{Coscienza non sostenuta}

Secondo gli insegnamenti Theravāda, il Buddha parlava di questa qualità anche in termini di `coscienza non sostenuta'. Significa che c'è la cognizione, c'è il conoscere, che però non si poggia in nessun luogo, non dimora da nessuna parte. `Prestare attenzione al senza-morte' e la `coscienza non sostenuta' sono in un certo senso sinonimi. Sono come due descrizioni dello stesso albero visto da due prospettive diverse.

Nel descrivere la coscienza non sostenuta, il Buddha insegnava che ``dovunque ci sia qualcosa di intenzionale, qualcosa su cui si agisce, o qualcosa che rimane quiescente, quella cosa diventa la base su cui si appoggia la coscienza. E dove si poggia la coscienza, quella è la causa della confusione, dell'attaccamento, del divenire e della rinascita, e così via.

Ma se non c'è nulla di intenzionale, nulla su cui si agisce, o che rimane latente, allora la coscienza non ha una base su cui poggiare. E, non avendo una base su cui poggiare, la coscienza è liberata. Si riconosce che `la coscienza, che non trova fondamento, è liberata'. Grazie al suo essere stabile, il cuore è contento. Grazie alla contentezza, non è agitato. Colui che non è agitato realizza interiormente il nibbāna assoluto e perfetto''. (S 12.38 e S 22.53). 

Il Buddha utilizzava un'intera galassia di immagini, metafore e forme simili a questa perché si rivolgevano a persone diverse in modi diversi. In un altro passaggio il Buddha chiede ai suoi discepoli: ``Se ci fosse una casa con un muro rivolto verso est, e in quel muro ci fosse una finestra, quando il sole sorge al mattino, dove andrebbe a cadere il raggio di luce solare?''. 

Uno dei monaci risponde: ``Sulla parete occidentale''. Allora il Buddha chiede: ``E se non ci fosse una parete occidentale, dove andrebbe a posarsi la luce?''.

Il monaco risponde: ``Sul pavimento''. Allora il Buddha insiste: ``E se non ci fosse il pavimento, dove andrebbe a posarsi?''. Il monaco risponde: ``Sull'acqua''. 

Il Buddha si spinge un po' oltre e chiede: ``E se non ci fosse acqua, dove si poserebbe?''. Il monaco risponde correttamente quando dice: ``Se non c'è acqua, allora non si posa''. Il Buddha pone termine alla discussione dicendo: ``Esattamente. Quando il cuore è liberato dall'aggrapparsi ai cosiddetti quattro nutrimenti – cibo fisico, contatto sensoriale (vista, suono, odore, gusto e tatto), intenzione e coscienza – allora la coscienza non si posa da nessuna parte. Questo stato, ti dico, è senza sofferenza, afflizione o disperazione''. (S 12.64). 

\section*{La coscienza: invisibile, radiosa, illimitata}

In molti casi il linguaggio della tradizione Dzogchen è sorprendentemente simile al Theravāda. Nello Dzogchen la descrizione corrente delle qualità di rigpa, la consapevolezza non dualistica, è ``essenzialmente vuota, conoscitiva per natura, e sconfinata per capacità''. Un'altra traduzione di queste tre qualità è ``vacuità, conoscenza e lucidità o chiarezza''. Nelle scritture pāli (D 11.85 e M 49.25) il Buddha parla della mente di \textit{arahant} come di ``coscienza non manifesta, priva di segni, infinita e radiosa in tutte le direzioni''. I termini pāli sono \textit{viññāṇam} (coscienza), \textit{aniddassanam }(vuoto, invisibile o privo di segni, che non si manifesta), \textit{anantam }(illimitato, sconfinato, infinito) e \textit{sabbato pabham} (radioso in tutte le direzioni, accessibile da tutti i lati).

Uno dei casi in cui il Buddha usa questa descrizione è alla fine di un lungo racconto esplicativo. Un monaco aveva chiesto: ``Quando è che la terra, l'acqua, il fuoco e il vento svaniscono e cessano senza residui?''. Al che il Buddha risponde che il monaco ha posto la domanda sbagliata. Quello che \textit{avrebbe dovuto} chiedere è: ``Dove è che la terra, l'acqua, il fuoco e il vento non trovano una base?''. Il Buddha stesso risponde a questa domanda, dicendo che è ``nella coscienza che è invisibile, illimitata e radiosa in tutte le direzioni'' che i quattro grandi elementi ``lunghi e corti, grossolani e sottili, puri e impuri non trovano una base. È lì che \textit{nāma-rūpa} (corpo e mente, nome e forma, soggetto e oggetto) giungono a termine. Con questa fine, questa cessazione della coscienza, tutte le cose qui sono condotte al loro termine''. 

Questa coscienza non sostenuta e in-sostenibile non è un principio astratto. Infatti, è stata la base dell'illuminazione del Buddha. Mentre il Buddha stava seduto ai piedi dell'albero della bodhi, fu attaccato dalle orde di Mara. Interi eserciti si lanciavano contro il Buddha, eppure nulla poteva penetrare lo spazio sotto l'albero. Tutte le armi e le lance che gli scagliavano si trasformavano in raggi di luce; le frecce che tiravano si trasformavano in fiori che andavano a posarsi attorno al Buddha. Nulla che potesse danneggiare il Buddha poteva penetrare quello spazio. Non c'era un punto su cui potesse poggiarsi. La vista, il suono, l'odore, il gusto e il tatto, lungo e corto, grossolano e sottile, puro e impuro, sono tutti aspetti di corpo e mente. Rappresentano gli attributi di tutti i fenomeni. Ciò nonostante, nessuno di questi riusciva a trovare una base. Il Buddha era in una sfera dove nulla aderiva. Tutto ciò che gli si avvicinava cadeva nel vuoto. Nulla rimaneva attaccato, nulla poteva penetrare e ferire il Buddha in nessun modo. Per farsi un'idea più chiara di questa qualità di coscienza non sostenuta, è opportuno riflettere su quest'immagine. Sono anche molto utili le frasi alla fine del passo che ho appena citato, soprattutto quando il Buddha dice: ``Quando la coscienza cessa, tutte le cose qui sono condotte al loro termine''. 

\section*{Anatomia della cessazione}

Il concetto di cessazione è molto conosciuto nella tradizione Theravāda. Sebbene si presume che sia un sinonimo di nibbāna, a volte viene presentato come un evento che tutti ricerchiamo, dove tutte le esperienze svaniranno e noi staremo finalmente bene: ``Un grande dio arriverà dal cielo, spazzerà via ogni cosa e farà in modo che tutti si sentano esultanti''. Non voglio lasciarmi ossessionare dalle parole, ma noi soffriamo molto, o ci sentiamo confusi, a causa di fraintendimenti come questo. Quando parliamo di far cessare la coscienza, pensate che questo significhi ``perdiamo tutti la coscienza''? Non può essere questo, no? Il Buddha non esaltava le virtù dell'assenza di coscienza, oppure tranquillanti e barbiturici sarebbero la soluzione: ``Dammi un anestetico e siamo sulla strada per il nibbāna''. Ovviamente non è così. Comprendere cosa si intende per interruzione o cessazione è pertanto di estrema importanza in questo contesto.

Ho conosciuto persone, specialmente fra quanti hanno praticato nella tradizione Theravāda, cui è stato insegnato e sono stati addestrati nell'idea che lo scopo della meditazione sia raggiungere un luogo di cessazione. Potremmo arrivare a un punto dove non vediamo né proviamo nulla; c'è la consapevolezza, ma è scomparso tutto. Un'assenza di vista, suoni, odori, sapori, tatto, il corpo, svanisce tutto. Poi a questi praticanti si dice: ``Questa è la cosa più grande. È a questo che bisogna mirare''. L'insegnante li incoraggia a impegnarsi diligentemente per lunghe ore nella meditazione. Quando una praticante disse al suo insegnante di aver raggiunto quello stato, questi si emozionò molto. Quindi le chiese ``allora, com'era?'' e lei rispose: ``Era come bere un bicchiere di acqua fredda, ma senza l'acqua e senza il bicchiere''. In un'altra occasione disse: ``Era come essere chiusa dentro un frigorifero''.

Questo non è l'unico modo per comprendere la cessazione. La radice della parola `nirodha' è \textit{rudh}, che significa ``non sorgere, terminare, controllare o trattenere'', come trattenere un cavallo con le redini. Quindi nirodha significa anche trattenere ogni cosa in tutta la sua ampiezza. ``Interruzione della coscienza'' può pertanto implicare che in qualche modo tutto viene tenuto sotto controllo piuttosto che semplicemente svanisca. È tracciare di nuovo la propria mappa interiore.

Un aneddoto dei tempi del Buddha potrà aiutarci a comprendere più estesamente cosa significhi tutto questo. Una notte, mentre stava meditando, apparve al Buddha un bellissimo e luminoso devata, di nome Rohitassa. Questi disse al Buddha: ``Quando ero un essere umano, ero un cercatore spirituale dai grandi poteri psichici, un camminatore del cielo. Sebbene abbia camminato per cento anni con grande determinazione e risoluzione per raggiungere la fine del mondo, non sono riuscito a raggiungere la fine del mondo. Sono morto durante il viaggio, prima di trovarla. Mi puoi dire allora se è possibile viaggiare sino alla fine del mondo?''.

Al che il Buddha rispose: ``Non è possibile raggiungere la fine del mondo camminando, però ti dico anche che sino a quando non raggiungi la fine del mondo, non raggiungerai la fine della sofferenza''. Rohitassa era un po' confuso e disse: ``Ti prego, venerabile, spiegami cosa vuoi dire''. Il Buddha rispose: ``Proprio in questo corpo, alto due braccia, vi è il mondo, l'origine del mondo, la cessazione del mondo e la via che conduce alla cessazione del mondo''. (A 4.45, S 2.26).

In questo caso il Buddha ha adottato l'identica formulazione delle Quattro Nobili Verità. Il mondo, \textit{loka}, in questo caso significa il mondo dell'esperienza. E quasi sempre il Buddha usa il termine `il mondo' in questo senso. Egli si riferisce al mondo \textit{così come lo sperimentiamo}. Il che comprende solo vista, suono, odorato, gusto, tatto, pensiero, emozione e sensazione. È tutto qui. Ecco cos'è `il mondo': il mio mondo, il tuo mondo. Non è il pianeta geografico astratto, il mondo che corrisponde all'universo. È l'esperienza diretta del pianeta, delle persone e del cosmo. Qui è l'origine del mondo, la cessazione del mondo e la via che conduce alla cessazione del mondo.

Egli diceva che fintanto che creiamo ``io e la mia esperienza'', ``io qui'' e ``il mondo là fuori'', siamo bloccati nel mondo di soggetto e oggetto. Allora c'è dukkha. E la via che conduce alla cessazione di questo dualismo è la via che conduce alla cessazione della sofferenza. Da un punto di vista geografico è impossibile viaggiare sino alla fine del mondo. È solo quando arriviamo alla fine del mondo, che alla lettera significa la cessazione della sua alterità, la sua quiddità, che raggiungeremo la fine di dukkha, dell'insoddisfazione. Quando smettiamo di creare gli oggetti sensoriali come realtà assolute e smettiamo di vedere i pensieri e le sensazioni come cose solide, allora c'è la cessazione. 

Vedere che il mondo è dentro la nostra mente è un modo per lavorare con questi principi. L'intero universo è compreso quando ci rendiamo conto che ha luogo dentro la nostra mente; e in quel momento, quando riconosciamo che ha luogo tutto \textit{qui}, cessa. La sua quiddità cessa. La sua alterità cessa. Cessa la sua sostanzialità.

Questo è solo un modo di parlarne e di concepirlo, però ritengo che ci avvicini molto alla verità, perché in questo modo è tenuto sotto controllo. È conosciuto. Ma c'è anche la qualità della sua vacuità. La sua in-sostanzialità è conosciuta. Non gli imputiamo una solidità, una realtà che non gli appartiene. Ci limitiamo a guardare il mondo direttamente, conoscendolo completamente e interamente.

Allora, cosa accade quando il mondo cessa? Ricordo che una volta Ajahn Sumedho stava tenendo un discorso proprio su questo tema e disse: ``Adesso farò sparire completamente il mondo. Farò terminare il mondo''. Rimase lì seduto e disse: ``Ok, siete pronti?... Il mondo è appena terminato\ldots{} Volete che lo riporti in essere? Ok\ldots{} Bentornato di nuovo''.

Dall'esterno non si vedeva nulla. Tutto ha luogo dentro di noi. Quando smettiamo di creare il mondo smettiamo di crearci a vicenda. Smettiamo di imputare il senso di solidità che crea un senso di separatezza eppure non escludiamo i sensi in nessun modo. In realtà, facciamo luce sull'apparenza, sugli strati di confusione, di opinioni, di giudizio, del nostro condizionamento, in modo da riuscire a vedere le cose così come sono veramente. In questo momento, dukkha cessa. Questo è ciò che possiamo chiamare l'esperienza di rigpa. C'è il conoscere, c'è la liberazione e la libertà. Non c'è dukkha.

\vspace*{-1.2em}
\section*{Ti danno fastidio i rumori?}

\vspace*{-0.8em}
A questo proposito, ancora una volta sono colpito da quanto il linguaggio adottato negli insegnamenti Dzogchen corrisponda alle espressioni usate dai maestri della foresta thailandesi. Infatti essi, e in particolare il mio insegnante, Ajahn Chah, fanno un uso ripetuto e abbondante proprio di questo tipo di insegnamenti. 

Se le persone cercavano di meditare escludendo il mondo, egli era solito rendergli la vita molto difficile. Se si imbatteva in un monaco o una monaca che aveva sprangato la finestra del suo cuore nel tentativo di bloccare tutto ciò che arrivava dall'esterno, era capace di fargli passare un brutto quarto d'ora. Nominò un monaco di questo tipo come suo assistente per un periodo e non lo lasciava mai stare seduto a meditare. Non appena vedeva il monaco chiudere gli occhi `per mettersi a meditare' immediatamente lo spediva a fare qualche commissione. Ajahn Chah sapeva che l'isolamento non era il modo per trovare l'autentica pace interiore. Questo perché lui stesso aveva trascorso anni cercando di escludere il mondo affinché lo lasciasse in pace. Fallì miseramente e alla fine fu in grado di vedere che non è in questo modo che si trovano la compiutezza e la risoluzione. 

Tanti anni fa lui era un monaco errante che viveva da solo su una montagna ai cui piedi c'era un villaggio, e si atteneva a un rigido programma di meditazione. In Thailandia le persone adorano guardare film all'aperto fino a tarda notte, perché le notti sono fresche a confronto del caldo torrido del giorno. Se c'era una festa, in genere andava avanti tutta la notte. Circa cinquant'anni fa, in Thailandia si cominciava appena a usare l'amplificazione e qualsiasi evento pubblico degno di questo nome doveva avere un altoparlante che sbraitasse al massimo del volume per tutta la notte. Una volta Ajahn Chah se ne stava tranquillo a meditare sulla montagna mentre giù al villaggio si faceva festa. Le canzoni del posto e la musica pop risuonavano amplificate per tutta la zona. Ajahn Chah se ne stava seduto, fremendo e pensando: ``Non si rendono conto di tutto il cattivo karma che producono disturbando la mia meditazione? Lo sanno che sono quassù. Dopo tutto, sono il loro insegnante. Non hanno imparato proprio niente? Nemmeno i cinque precetti? Scommetto che si stanno ubriacando senza misura'' e così via.

Ajahn Chah però era un tipo piuttosto in gamba. Si ascoltò lamentarsi e immediatamente comprese: ``Beh, si stanno solo divertendo laggiù. Io invece qui mi sto infliggendo del dolore. Per quanto mi possa arrabbiare, la rabbia dentro di me fa solo ancora più chiasso''. Quindi ebbe questa intuizione: ``Ah, il rumore è solo rumore. Sono io che vado a dargli fastidio. Se lascio in pace il rumore, non mi darà fastidio. Fa semplicemente quello che deve fare. Questo è ciò che fa il rumore. Fa rumore. È il suo lavoro. Quindi, se non vado fuori a dare fastidio al rumore, il rumore non darà fastidio a me. Ah!''.

Come poi si dimostrò, questa intuizione ebbe un effetto così profondo da diventare un principio che da quel momento in poi egli cominciò a esporre. Se qualcuno dei monaci mostrava l'impulso di evitare le persone, gli stimoli, il mondo delle cose e le responsabilità, allora egli tendeva a sbattercelo dentro. Incaricava quel monaco di controllare il lavoro degli operai, oppure se lo portava dietro ogni volta che c'era da condurre una cerimonia di benedizione di una casa. Faceva in modo che il monaco fosse coinvolto nelle faccende perché stava cercando di insegnargli a lasciar andare l'idea che la meditazione avesse bisogno di condizioni sterili, ma a vedere di fatto che la maggior parte della saggezza sorge dal gestire abilmente il logorio provocato dal mondo.

Ajahn Chah stava trasmettendo un'intuizione importante. È inutile cercare di trovare la pace annullando o cancellando il mondo sensoriale. La pace è possibile solo se non si attribuisce al mondo più sostanzialità o più realtà di quanta ne abbia veramente.

\section*{Toccare la terra}

A volte, quando faccio l'esempio del Buddha seduto ai piedi dell'albero della bodhi, le persone ritengono che si tratti della negazione del mondo sensoriale. C'è una sfumatura di condiscendenza, di alterigia. Ci spaventiamo quando sentiamo parlare di distacco dal mondo sensoriale, perché potrebbe offendere le nostre abitudini di affermazione della vita. 

L'equilibrio, e si tratta di qualcosa che possiamo sperimentare da soli, non è nella negazione, ma si raggiunge quando smettiamo di crearci a vicenda e ci permettiamo di rilassarci nella pura qualità del conoscere. Non fabbricando il mondo, noi stessi o le nostre storie, si genera un lieve rilassamento e, paradossalmente, ci ritroviamo più in sintonia che mai con la vita. Questo non può avvenire quando siamo impegnati a portare in giro ``io e te'', ``è la mia vita'', ``il mio passato'' e ``il mio futuro'' e tutto il resto del mondo con i suoi problemi. In realtà, la conseguenza di questo lasciar andare non è una sorta di ottundimento o di presa di distanza, ma una sorprendente sintonia.

La cosmologia buddhista e le storie dei Sutta contengono sempre un elemento storico, mitico e psicologico. Quando parliamo del Buddha ai piedi dell'albero della bodhi a volte ci chiediamo: ``Si trattava davvero di \textit{quell'}albero? Siamo sicuri che fosse \textit{veramente }seduto presso il fiume Neranjara, vicino a Bodh-gaya? Come si fa a sapere che fosse veramente lì?''. La storia vuole che forse il Buddha stava seduto sotto un albero, o che un principe nepalese fosse seduto sotto un albero, e qualcosa accadde (o smise di accadere) da qualche parte in India circa un paio di migliaia di anni fa. In altre parole, questo racconto contiene aspetti sia storici sia mitologici; ma l'elemento cruciale è che tutto questo corrisponde alla nostra psicologia. In che modo simboleggia la nostra esperienza?

Il canovaccio della storia è che, anche se il Buddha ha penetrato completamente i cicli dell'origine dipendente e il suo cuore è totalmente libero, l'esercito di Mara non si arrende. Mara ha inviato l'orrore, ha inviato le sue splendide figlie, ha addirittura inviato il fattore ricatto genitoriale: ``Figlio mio, avresti potuto fare grandi cose; sei un leader nato, saresti stato un grande re. Adesso è rimasto solo il tuo fratellastro, Nanda, che è uno smidollato, un inetto sul campo di battaglia. Se proprio vuoi continuare con questa faccenda del monachesimo, il regno cadrà in rovina. Ma non importa, va bene così. Devi fare quello che ti senti di fare. Sappi solo che stai rovinando la mia vita; ma non preoccuparti, va tutto bene''.

Le forze della seduzione, della paura e della responsabilità sono tutte lì. Eppure il Buddha non si limita a chiudere gli occhi e rifugiarsi in un assorbimento beato. Quando gli eserciti di Mara lo raggiungono, egli li affronta direttamente e dice: ``Ti conosco, Mara. So di cosa si tratta''. Il Buddha non si mette a discutere con Mara, non fa sorgere avversione nei confronti di Mara. Non si lascia illudere, non reagisce contro ciò che sta avvenendo in quel momento. Qualunque cosa facciano gli eserciti di Mara, nessuno può penetrare quello spazio sotto l'albero della bodhi. Tutte le loro armi si trasformano in fiori e incenso e raggi di luce che illuminano il trono del \textit{vajra}.

Eppure, anche quando il cuore del Buddha è totalmente liberato, Mara ancora non arretra. Egli dice al Buddha: ``Che diritto hai di pretendere il seggio reale nel sito inamovibile? \textit{Sono io} il re di questo mondo. \textit{Sono io} che dovrei sedere lì. \textit{Sono io} che comando qui. \textit{Sono io} che merito di stare lì, non è vero?''. E quindi si rivolge verso la sua orda di 700.000 soldati e questi in coro dicono: ``Sì, è così, sire!''. ``Vedi'', dice Mara, ``sono tutti d'accordo. Questo posto appartiene a me, non a te. Sono io il potente''.

A questo punto succede che non appena Mara invoca i suoi testimoni per sostenerlo, il Buddha invoca la dea madre, Maer Toranee, come sua testimone. Il Buddha si protende verso il suolo, tocca la terra e fa appello alla madre terra, la quale compare e dice: ``Questo è il mio vero figlio. Ha tutto il diritto di reclamare il trono del vajra nel sito inamovibile. Egli ha sviluppato tutte le virtù necessarie a reclamare la sovranità dell'illuminazione perfetta e completa. Questo posto non ti appartiene, Mara''. A questo punto dai capelli della dea madre esce un fiume che spazza via gli eserciti di Mara. Più tardi questi ritornano contriti, offrendo doni e fiori e chiedendo perdono: ``Sono veramente dispiaciuto, Madre, non era davvero mia intenzione''.

È molto interessante il fatto che non è diventato un Buddha insegnante completamente illuminato senza l'aiuto della dea madre e, in seguito, quello del dio padre. Fu Brahmā Sahampati, il dio creatore, l'amministratore delegato dell'universo, che venne a chiedere al Buddha di insegnare. Senza queste due figure, non avrebbe mai lasciato il sito inamovibile e non avrebbe iniziato a insegnare. Così, stando alla mitologia, il racconto presenta alcune bizzarrie.

Il Buddha che delicatamente tocca la terra è una metafora meravigliosa. Ci insegna che anche se possediamo uno spazio interno illuminato e libero, questo spazio deve essere connesso al mondo fenomenico. Altrimenti non c'è compiutezza. Per questo motivo meditare con gli occhi aperti è, in un certo senso, un utile ponte. Noi coltiviamo un vasto spazio interiore, che però è necessariamente collegato al mondo fenomenico. Se c'è solo un'esperienza interiore soggettiva dell'illuminazione, siamo ancora intrappolati. L'esercito di Mara non si arrende. Le contrarietà sono ovunque, la dichiarazione dei redditi, la burocrazia, le invidie. Vediamo che sono vuote, ma continuano ad arrivarci da ogni parte.

Però, nel protendersi a toccare la terra, il Buddha riconosce che sì, c'è il trascendente e l'incondizionato. Ma anche l'umiltà richiede non solo di afferrare l'incondizionato e il trascendente. Il Buddha ha riconosciuto e affermato che: ``C'è il condizionato. C'è il mondo sensoriale. C'è la terra che costituisce il mio corpo e il mio respiro e il cibo che mangio''.

L'atto di protendersi dal trascendente esprime: ``Come potrebbe il pieno coinvolgimento nel mondo sensoriale corrompere la libertà innata del cuore? Tale libertà è inarrestabile, incorruttibile e inconfondibile con nessuna esperienza sensoriale. Allora perché non lasciare entrare tutto? Riconoscendo apertamente, liberamente il limitato (il bisogno di invocare la dea madre a testimone, per esempio) l'illimitato manifesta tutto il suo potenziale. L'esitazione, la cautela di tenere lontano il condizionato, tradisce una fondamentale mancanza di fede nell'inviolabilità naturale dell'incondizionato.

Un'altra frase che esprime lo stesso concetto è ``\textit{cittaṃ pabhassaraṃ, ākandukehi kilesehi}'', che significa ``la natura del cuore è intrinsecamente radiosa, le contaminazioni sono solo visitatori'' (A 1.61). Indica che la natura del cuore è intrinsecamente pura e perfetta. Le cose che sembrano contaminare questa purezza sono solo visitatori di passaggio, viandanti che si trovano lì per caso. La natura del cuore non può essere veramente corrotta da nulla di tutto ciò.

