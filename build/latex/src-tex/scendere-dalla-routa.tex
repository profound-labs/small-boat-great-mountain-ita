
\dropcaps{N}{ella tradizione buddhista therav\=ada}, il \textit{Sutta sulla gentilezza amorevole} è uno dei più conosciuti, più amati e più recitati fra i discorsi del Buddha. Conoscevo la versione pāli già da molto tempo, ma quando nella mia comunità fu tradotto in inglese e cominciammo a recitarlo regolarmente, mi colpì il fatto che solo il novanta per cento degli insegnamenti è sulla gentilezza nei confronti di tutti gli esseri.

La prima parte è una chiara descrizione di come inviare questa qualità di amore puro. Ad esempio:

\begin{quote}
\itshape
Augurando: in gioia e in sicurezza, \\
che tutti gli esseri siano felici. \\
Qualsiasi essere vivente, \\
debole o forte, nessuno escluso, \\
il grande o il potente, medio, corto o piccolo, \\
il visibile e l'invisibile, \\
quelli che vivono vicino e quelli che vivono lontano, \\
quelli nati e quelli che stanno per rinascere, \\
che tutti gli esseri siano in pace!
\end{quote}

Però poi le ultime quattro righe del sutta presentano un messaggio molto diverso:

\begin{quote}
\itshape
Non aggrappandosi a opinioni rigide, \\
colui che ha il cuore puro e visione chiara, \\
liberato da tutti i desideri sensoriali, \\
non più nascerà in questo mondo.\footnote{Mettasutta, SN 145-152.}
\end{quote}

Fino a queste ultime quattro righe si ha un flusso ininterrotto di idee, un sentimento di profonda ispirazione che il Buddha incoraggia in ciascuno di noi. Diventa sempre più esaltante e luminoso, ma alla fine arriva la battuta chiave: ``Non rinato in questo mondo''. Si tratta di un cambiamento non da poco che ci coglie di sorpresa. Cosa è successo all'amore per tutti gli esseri? Qualcos'altro è entrato in gioco, ma cos'è?

Queste ultime righe ci rammentano qualcosa che molti di noi preferirebbero dimenticare: il concetto di non rinascita. Questo concetto non ha veramente messo radici come ideale culturale in occidente. Al contrario, sono molto diffuse cose come dei rassicuranti fondi pensione e una buona assistenza sanitaria. Non c'è nulla di male, ma non rappresentano lo scopo della nostra vita.

Specialmente nel mondo buddhista occidentale, non pensiamo veramente in termini di nascita e di morte. Possiamo avere una vaga idea che dopo la morte qualcosa potrebbe accadere, ma non siamo sicuri di cosa e non sembra che la maggior parte di noi se ne curi un granché. La nostra principale preoccupazione è andare avanti con la pratica, il che va benissimo, ma neppure questo obiettivo importantissimo è il culmine. Sarà quindi utile fare un passo indietro e considerare il nostro condizionamento culturale e come abbia influenzato la nostra comprensione di ciò che significa `non rinascere'.

\section*{Affermazione e negazione della vita}

Quella occidentale è una cultura che dà molto valore alla vita. Teniamo in considerazione le nostre vite e il mondo e vogliamo rispettare tutti gli esseri viventi, vogliamo sviluppare un senso di amore, di unità ed essere tutt'uno con la natura. Vogliamo provare compassione e gentilezza per il nostro corpo ed essere sani e felici. Avere una `vita piena' significa fare un sacco di cose, divertirci e sentirci soddisfatti. Sto generalizzando, naturalmente. Ciò nonostante, queste cose riflettono la cultura che ci circonda. Così, quando si allude al fatto che il completamento del cammino è il non rinascere, le persone spesso dicono: ``Va bene, ma veramente preferirei ritornare. Adesso che ho capito e mi sono abituato a questa faccenda dell'essere umano, perché dovrei rinunciarvi?''. Con questo tipo di approccio mentale, si può rimanere delusi e risentiti per il fatto che questi insegnamenti vanno contro l'idea che la vita sia buona. Alcune persone protestano dicendo: ``C'è da occuparsi del movimento ecologista, l'aspirazione ad amare questo pianeta e prendersi cura della natura. Perché dovrei desiderare di lasciarlo per sempre?''.

Queste reazioni possono essere in parte giustificate. Alcune traduzioni degli insegnamenti Theravāda, specialmente i commentari, danno l'impressione che tutto il mondo materiale sia un esperimento malriuscito. Se mettete insieme la vostra roba e vi tirate fuori da questo posto, non vi guardate indietro: ``Addio, ci siamo visti. Buona fortuna a tutti voi che siete rimasti indietro''. È un modo un po' grossolano per dirlo, ma c'è anche questo aspetto. Quando lo poniamo in questi termini, le persone dicono: ``Però è strano. Che significa tutto questo?''.

Generazioni di praticanti hanno seguito gli insegnamenti Theravāda e si sono innamorati della meditazione e di molti dei principi, eppure, quando sentono parlare della non rinascita molti diventano dubbiosi. Le cose assumono una sfumatura negativa: ``Mah, non ne sono poi così sicuro. Forse il Buddha sotto sotto non era altro che un tipo incavolato che non vedeva l'ora di tirarsi fuori da tutto questo, che non gli andava di preoccuparsi''.

La critica che il buddhismo sia nichilista non è solo di questi tempi. Il Buddha veniva regolarmente accusato di essere un nichilista, di avere una visione negativa della vita e del mondo. Egli rispondeva spiegando che questo era un fraintendimento degli insegnamenti: ``Coloro che dicono che io insegno l'annichilimento di un essere vivente mi travisano, mi equivocano. Essi non insegnano quello che io insegno. Essi non dicono quello che io dico''. (M 22.37).

La metafora del serpente e della corda, usata dal Buddha, è molto appropriata qui. Supponete di stare camminando in mezzo all'erba quando improvvisamente vedete una forma circolare sul sentiero. Sembra un serpente e rimanete paralizzati dalla paura. Ma quando guardate meglio vedete che non si tratta affatto di un serpente, è soltanto una corda arrotolata. Vi sentite sollevati in tutto il corpo e la paura scompare. Tutto è di nuovo tranquillo. Dopo aver illustrato questa metafora il Buddha faceva questa domanda: ``Cosa ne è stato del serpente dopo che la corda arrotolata è stata riconosciuta?''. La risposta è: nulla. Non è successo nulla al serpente perché non ce n'è mai stato uno. 

Così, allo stesso modo, quando le persone chiedono ``cosa ne è dell'io quando il corpo muore?'', la risposta è sostanzialmente la stessa. Tutto il concetto di io si basa su un equivoco, per cui la domanda non è pertinente. Il modo in cui vediamo il nostro `io' è un equivoco di fondo che deve essere corretto. La pratica ci insegna a vedere con chiarezza, a risvegliarci a ciò che c'è veramente. 

C'è un personaggio, di nome Vacchagotta, ben noto negli insegnamenti Theravāda. Egli era stato dapprima membro di un'altra scuola, ma era uno di quegli yogi erranti che spesso si presentavano agli insegnamenti del Buddha. In varie occasioni arrivava e faceva domande al Buddha. Una volta si avvicinò al Buddha e gli chiese: ``Cosa succede agli esseri illuminati dopo che il loro corpo muore?''.

Vacchagotta sondò quella che pensava potesse essere la risposta: ``Ricompaiono in un'altra sfera?''. Il Buddha disse: ``Non è corretto dire `ricompaiono'\thinspace''.

Vacchagotta tentò di nuovo e chiese: ``Bene, \textit{non }ricompaiono da qualche parte?''. Il Buddha disse: ``Non è corretto dire `non ricompaiono'\thinspace''.

Vacchagotta, che non si arrendeva, tentò di nuovo: ``Allora \textit{sia} ricompaiono \textit{sia} non ricompaiono?''. Il Buddha disse: ``Non è corretto dire `sia ricompaiono sia non ricompaiono'\thinspace''.

Vacchagotta chiese ancora: ``\textit{Né} ricompaiono e \textit{né }non ricompaiono?''. E il Buddha disse: ``Non è corretto dire `né ricompaiono e né non ricompaiono'\thinspace''. Questa volta Vacchagotta si arrese: ``Sono veramente confuso. Come può essere? Per lo meno uno di questi quattro casi deve andare bene''.

``Non è così, Vacchagotta'', disse il Buddha, ``lascia che ti faccia un esempio. Supponiamo che qui stia ardendo un fuocherello fatto di paglia e sterpi e io ti chiedessi: `C'è qui un fuocherello?'. Tu risponderesti: `Sì, qui c'è un fuoco'. Se spegnessimo il fuoco e poi io ti chiedessi: `Dove è andato il fuoco? È andato a nord, a sud, a est o a ovest?', cosa risponderesti?''. Vacchagotta disse: ``Ma la domanda non è corretta. Non è andato da nessuna\textit{ parte}, si è semplicemente spento''. E il Buddha disse: ``Esattamente, Vacchagotta. Il modo in cui hai posto la domanda presume una realtà che non esiste. Quindi, così come è stata posta, la tua domanda non ha una risposta. Non è corretto dire `ricompaiono'; non è corretto dire `non ricompaiono' \ldots{}'' (M 72.16-20).

Questo è un aspetto su cui il Buddha fu eccezionalmente chiaro. Per quanto gli si facesse pressione, o a prescindere dalla situazione in cui si trovava, non cercò mai di descrivere com'è. Le persone gli chiedevano: ``Allora, fai tutta questa pratica, lavorando sodo per anni e anni, facendo meditazione seduta e camminata, osservando tutte queste regole. Alla fine scendi dalla ruota e poi, e poi\ldots{}''. E il Buddha rispondeva: ``Di nulla si è parlato; a questo non ci si è riferiti'' (ad es. in M 63, s 44 e A 10.95). A quel punto era solito incoraggiare le persone perché continuassero nella pratica di meditazione. Non ottenere una risposta alla domanda di cosa succede può essere molto frustante e può far arrovellare la mente.

\section*{Il non rivelato}

Quando mi avvicinai per la prima volta a questi insegnamenti in Thailandia, credevo che ci fosse una qualche trasmissione orale segreta che era stata tramandata durante i secoli. Magari Ajahn Chah era uno di quelli che aveva ricevuto questo insegnamento segreto e lo teneva nascosto. Naturalmente era una cosa non adatta ad essere divulgata, ma diretta solo ai meditanti veramente impegnati ed evoluti. Pensavo che se fossi rimasto abbastanza a lungo forse avrei fatto lo scoop. Come potete vedere, avevo le idee piuttosto confuse su quello che insegnava il Buddha. 

Poiché il Buddha fu sempre risoluto nel rifiutarsi di rispondere a certe domande, molti pensavano che in realtà egli non sapesse le risposte. Di nuovo, sembra che gli insegnamenti contengano una sfumatura di nichilismo: l'estinzione dell'io, la non rinascita e neppure l'ombra di beatitudine eterna. Potremmo reagire a questa faccenda dello scendere dalla ruota con un po' di disappunto. A questo proposito i maestri Theravāda si sono pronunciati molte volte. Alle persone piace l'idea di andare in paradiso, ma hanno paura di meditare troppo perché si potrebbero ritrovare negli elevati regni di Brahmā e sentire la mancanza della famiglia, della moglie, del marito e dei figli. E chissà cosa succede quando si realizza il nibbāna? Allora sei veramente andato. Per lo meno, nei regni dei deva ritrovi la famiglia e gli amici e te ne puoi stare in beatitudine per qualche migliaio di eoni. Il nibbāna però, è una cosa seria. È come se suonasse la campanella e tu dicessi addio a tutto.

Anche nelle culture asiatiche è una cosa molto diffusa, spesso le persone hanno paura del nibbāna. Sappiamo di discussioni esilaranti fra i praticanti e i loro maestri: ``Pensate che il Buddha ti avrebbe promesso qualcosa o ti avrebbe incoraggiato a fare qualcosa che fosse veramente orrendo?''. E i praticanti rispondono: ``No, certo che non l'avrebbe fatto''. ``Allora, volete raggiungere il nibbāna?'' e la risposta è: ``No. Preferiamo prima andare in paradiso, grazie''. È come se morissero dalla voglia di trascorrere un po' di tempo in uno di questi ospizi di lusso dove poter giocar a golf, nuotare in un lago, passeggiare fra giardini di aiole bellissime, fra bella gente e bei panorami. Poi, una volta che si siano stancati di tutto questo, forse il nibbāna sembrerà più attraente. 

Di recente ho finito di rivedere un romanzo buddhista scritto nel 1906. A quell'epoca, le istituzioni cristiane erano molto critiche nei confronti del buddhismo. Era visto come un insegnamento negativo, nichilista, e il nibbāna era considerato come una specie di estinzione gloriosa. L'autore fa un buon lavoro nel correggere questo malinteso. 

Uno dei personaggi principali della storia è l'eroe Kāmanīta che, senza saperlo, incontra il Buddha. Ancora prima di incontrarlo, Kāmanīta ha preso la risoluzione di diventare un discepolo del Buddha. Sapendo che è il più grande maestro in circolazione e certo che il Buddha ha promesso una vita eterna di beatitudine dopo la morte, Kāmanīta si mette in viaggio alla sua ricerca. Dopo alcuni anni di ricerca, l'eroe trascorre una notte in un piccolo \textit{dharmasālā}, un ricovero per pellegrini, nella casa di un vasaio. Si dà il caso che il Buddha stia viaggiando nella stessa città e che si fermi presso lo stesso ricovero. 

Il suo compagno di stanza inizia a parlare e il Buddha gli chiede: ``Chi sei e da dove vieni?''.

Il giovanotto risponde: ``Il mio nome è Kāmanīta. Sono un discepolo del Buddha Gotama''.

``Ah'', dice il Buddha, ``l'hai mai incontrato?''.

``No'', dice Kāmanīta, ``sfortunatamente non l'ho mai incontrato''.

``Se lo incontrassi, lo riconosceresti?''.

``Non credo''.

``Allora perché sei un discepolo del Buddha?''.

``Egli insegna la beatitudine all'inizio, la beatitudine nel mezzo e la beatitudine alla fine. E dopo una vita di sincera devozione, si può aspirare a una vita beata ed eterna dopo la morte''.

``Davvero'', dice il Buddha e, con il suo senso dell'umorismo, il Buddha non si rivela. Lascia che Kāmanīta continui a parlare. Alla fine il Buddha dice: ``Beh, se vuoi posso riferirti gli insegnamenti del Buddha''.

``Davvero? Sarebbe stupendo'', dice Kāmanīta entusiasta. (Questa prima parte della storia è tratta dal Sutta m 140).

Prima il Buddha dà un insegnamento sulle Quattro Nobili Verità, poi insegna le tre caratteristiche: anicca, dukkha e anattā. Alla fine, insegna l'origine dipendente. 

Nel frattempo Kāmanīta si fa irrequieto e pensa: ``Non mi sembra proprio che abbia ragione. Tutto questo non mi piace per niente''. Continua a insistere con il Buddha: ``Cosa succede dopo la morte? Magari la vita è anche insoddisfacente; ma che accade \textit{dopo} la morte?''. 

Il Buddha semplicemente risponde: ``Il Maestro non ha rivelato nulla a questo proposito''. 

``Ma è impossibile'', dice Kāmanīta, ``deve avere detto qualcosa. Come puoi andare avanti se non puoi aspirare a una vita di beatitudine eterna?''.

Il Buddha racconta una metafora per avvalorare la sua tesi: se una casa andasse a fuoco e il servo accorresse per avvisare il padrone che devono tutti allontanarsi immediatamente, sarebbe ridicolo se il padrone reagisse all'emergenza chiedendo, ``c'è una tempesta fuori o è una bella notte di luna piena?'', intendendo che, in caso di tempesta, non sarebbe disposto a uscire. Si dovrebbe dedurre che il padrone di casa non si rende conto della gravità della situazione. Altrimenti non darebbe una risposta così insensata; non ignorerebbe la realtà. A questo proposito il Buddha dice a Kāmanīta: ``Allo stesso modo anche tu dovresti comportarti come se il tuo capo fosse circondato dalle fiamme, come se la tua casa stesse bruciando. Quale fuoco? Il mondo! E quale fiamma ha appiccato il fuoco? La fiamma del desiderio, la fiamma dell'odio, la fiamma dell'illusione''. 

Inoltre spiegò che se il Tathāgata avesse parlato in termini di ``una vita eterna e beata'', molti suoi discepoli sarebbero stati conquistati dall'idea, ma avrebbero avuto la tentazione di aggrapparvisi con un desiderio appassionato che sarebbe stato d'ostacolo alla vera pace e alla libertà, e si sarebbero pertanto impantanati nella rete vischiosa della brama di esistenza. E aggrappati all'idea di un aldilà, per il quale necessariamente avrebbero dovuto prendere a prestito le immagini da questa vita, non si sarebbero afferrati forse ancora più saldamente al presente, quanto più cercassero di rincorrere quell'aldilà immaginario?

Kāmanīta si risentì di quello che gli veniva insegnato e si rifiutò di credervi. Si convinse che i \textit{veri} insegnamenti del Buddha portassero a una vita eterna, piena di gioia suprema. È una storia molto lunga, ma la sostanza è che Kāmanīta fa la stessa esperienza comune a molte altre persone. Egli si imbatte in questa presunta espressione negativa e non vuole rinunciare all'idea di beatitudine eterna. Solo molto più avanti nel libro, Kāmanīta si rende conto che di fatto aveva condiviso la stanza con il Buddha e aveva ricevuto direttamente da lui i veri insegnamenti.

Discussioni di questo tipo e altre simili mi avevano lasciato perplesso per alcuni anni. Perché il Buddha non ha detto qualcosa? Qualunque tradizione religiosa ha un modo per esprimere qual è la destinazione del cammino. Ciò nonostante, a noi buddhisti Theravāda tutto quello che ci viene dato è che dobbiamo capire cosa significhi rinascere e smettere di farlo.

\section*{L'incomprensibile}

In realtà, in un altro paio di occasioni, al Buddha vengono rivolte domande simili riguardo la natura di un essere illuminato dopo la morte ed egli ci dà qualcosa in più con cui divertirci. Dice: ``Un tale essere, dopo la disgregazione del corpo che segue la morte, oltrepassa l'ambito della conoscenza di dei e umani'' (D i.3.73). Questo ci dà qualche informazione. In un'altra occasione, un giovane di nome Upasiva gli chiede: ``Coloro che hanno raggiunto la fine, non esistono più? O sono resi immortali, perfettamente liberi?''.

Il Buddha risponde: ``Coloro che hanno raggiunto la fine esulano da qualunque criterio con cui possano essere misurati. Colui di cui fosse possibile parlare non è più. Non si può dire `essi non esistono'. Ma quando tutte le modalità dell'essere, tutti i fenomeni, sono stati rimossi, anche tutti i modi per parlarne si sono esauriti''. (SN 1076).

Questo è ciò cui noi ci riferiamo come \textit{parinibbāna}; è quando si esauriscono le parole e i pensieri. È la dimensione del rigpa. Ogni linguaggio si basa su concezioni dualistiche. Per questo il Buddha è molto fermo nel dire: ``Anche tutti i modi per parlarne si sono esauriti''. Oltre quel punto gli autobus non vanno. È la fine, il limite. Il linguaggio e i concetti possono essere applicati fino a questo punto, ma non oltre. 

Dopo aver contemplato questo principio per anni, un po' alla volta ha cominciato a sembrarmi sempre più chiaro. Adesso provo un profondo rispetto per il fatto che il Buddha non ha detto assolutamente niente. Credo anche che il suo essere così risoluto quando diceva ``no, nulla può essere detto. Qualunque immagine, qualunque forma, qualunque descrizione va oltre la parola. Non può rappresentare la realtà'' fosse dovuto ad una autentica fedeltà alla realtà.

In un'altra circostanza, a uno dei discepoli del Buddha fu posta questa domanda: ``Cosa dice il tuo insegnante a proposito di cosa accade a un essere illuminato quando muore?''. Il discepolo risponde: ``Il Tathāgata non ha rivelato nulla a questo proposito''. Allora i suoi interlocutori dicono: ``Devi essere stato ordinato da poco o, se sei uno degli anziani, devi essere uno sciocco incompetente''; e con questo si alzano e se ne vanno. Il discepolo ritorna dal Buddha per vedere se ha dato la risposta giusta e il Buddha allora gli fa un sacco di domande.

Gli chiede: ``Qui e ora, Anurādha, con il Tathāgata seduto di fronte a te, puoi dire veramente che il Tathāgata \textit{è }i cinque khandha (corpo, sensazioni, percezioni, formazioni mentali e coscienza)?''.

``No, veramente questo non è possibile dirlo''.

``Puoi dire che il Tathāgata è \textit{nei} cinque khandha?''.

``No, veramente questo non è possibile dirlo''.

``Puoi dire che il Tathāgata è \textit{separato} dai cinque khandha?''.

``No, veramente questo non è possibile dirlo''.

``Puoi dire che egli \textit{possiede} i cinque khandha?''.

``No, questo non è possibile dirlo''.

``Puoi dire che egli \textit{non possiede }i cinque khandha? Che egli non ha i cinque khandha?''.

``No, neanche questo è possibile dirlo''.

Il Buddha allora dice: ``Se il Tathāgata non è veramente comprensibile quando è seduto di fronte a te qui e ora, allora, dopo la disgregazione del corpo che segue la morte, a maggior ragione, come può dirsi alcunché?'' (S 22.86).

Il Buddha cerca di scoraggiare l'abitudine di riempire i vuoti con idee o con qualche tipo di credenza o di forma. Invece incoraggia a una comprensione diretta della verità, così che ognuno sappia in prima persona qual è quella qualità trascendente.

Egli ci incoraggia a rendere solida quella qualità del conoscere, del rigpa. Piuttosto che creare un'idea su un qualcosa, o una sua immagine o un ricordo, oppure un piano per ottenerla, dobbiamo continuamente risvegliarci ad essa, ritornare sempre ad essa. Ed essa è di per sé indescrivibile. Possiamo parlare di temi come il conoscere, la vacuità, la lucidità, la chiarezza e così via, ma quando la mente è completamente sveglia alla sua stessa natura, le parole vengono a mancare. Si tratta dell'effetto `parinibbāna'; è un evento che marca il confine con ciò che sta al di là della sfera delle parole. 

Quando la mente è veramente sveglia, evaporiamo? Ci congeliamo? No. Di fatto siamo più vivi di quanto non siamo mai stati prima. C'è una qualità dell'essere totalmente vivi. Eppure, c'è anche una completa assenza di definizione. In quel momento non siamo maschi, non siamo femmine, non siamo vecchi, non siamo giovani, non siamo in alcun luogo, non c'è il tempo. È `essere', ma senza proprietario, né tempo. Sperimentando questo, ci si sente bene? A me fa sentire bene, molto bene. Questo è il risultato della fine delle rinascite.

\section*{Il processo della rinascita}

Quando parliamo della rinascita, parliamo di quel momento in cui l'attaccamento colpisce e il cuore viene afferrato e trascinato via. Il verso che chiude il \textit{Metta Sutta} ci incoraggia a lasciar andare l'attaccamento così da non rinascere. Non rinascere è come la realizzazione dell'amore puro, o rigpa. Non ci identifichiamo con nessun aspetto interno, esterno, psicologico, né con i mondi materiali del nostro corpo, pensieri, sensazioni, emozioni, sfere del Buddha, o qualsiasi altra cosa. Non appena avviene quella formulazione, quella cristallizzazione, c'è la nascita.

I quattro tipi di attaccamento sono:

\begin{quote}
\textit{kāmupādāna}:\\ attaccamento ai piaceri sensoriali;

\textit{diṭṭhupādāna}:\\ attaccamento alle opinioni e ai punti di vista;

\textit{sīlabbatupādāna}:\\ attaccamento alle convenzioni, ai guru, alle tecniche di meditazione, a un'etica, a determinate forme religiose; e

\textit{attavādupādāna}:\\ attaccamento all'idea dell'io.
\end{quote}

Gli ultimi quattro versi del \textit{Metta Sutta} riguardano la fine dell'attaccamento:

\begin{quote}
\textit{Non aggrappandosi a opinioni rigide} (diṭṭhupādāna),

\textit{colui che ha il cuore puro} (sīlabbatupādāna, attaccamento alle virtù, all'etica, alle regole, alle forme),

\textit{e visione chiara} (questo ha a che fare con l'attaccamento all'io, attavādupādāna),

\textit{liberato da tutti i desideri sensoriali} (kāmupādāna),

\textit{non più nascerà in questo mondo} (come cessa l'attaccamento, così cessa il rinascere).
\end{quote}

Non c'è perdita di rigpa. Se avijjā, l'ignoranza, non sorge, non c'è ignoranza. Non appena avijjā si intrufola, si innesca il processo della rinascita. La rinascita non può avvenire senza l'ignoranza e l'attaccamento.

Il Buddha scoprì un modo per individuare la via verso la libertà. Questa intuizione, che ebbe al momento dell'illuminazione, fu formulata in termini del processo dell'origine dipendente. Durante la prima settimana dopo l'illuminazione, egli trascorse tutto il tempo seduto sotto l'albero della bodhi studiando la modalità con cui sorge dukkha, la sofferenza.

All'inizio, quando c'è avijjā, i saṅkhārā (le formazioni mentali della volontà, la separatezza) vengono in essere. I saṅkhārā condizionano la coscienza. La coscienza condiziona nāma-rūpa (corpo-mente, nome-forma, soggetto-oggetto). Nāma-rūpa condiziona i sei sensi.

Quando ci sono un corpo e una mente sono presenti i sei sensi.

A causa dei sei sensi, c'è il contatto sensoriale: udito, sensazione, odorato, gusto, e tatto.

A causa del contatto sensoriale, sorge la sensazione: sensazioni piacevoli, spiacevoli e neutre.

E quando si intrufola una sensazione, se c'è abbastanza ignoranza in questo cocktail, la sensazione condiziona la brama: ``Questo mi piace, questo è proprio carino, di quello ne voglio di più''. Oppure: ``Quello non mi piace, come si è permesso?''. Le sensazioni di malessere condizionano l'avversione.

La brama (\textit{taṇhā}) condiziona l'attaccamento (upādāna).

L'attaccamento condiziona il divenire. C'è un'esperienza sensoriale: ``Cos'è questo? È buono. Chissà a chi appartiene? Lo posso tenere?''. Proviamo un'attrazione verso un determinato oggetto. Poi ne veniamo risucchiati: ``Oh, è proprio buono. Bene, ne devo avere di più. Chissà se ce n'è in frigorifero. Vado a prendermene un po''\thinspace'. Questo è divenire. 

Contatto, sensazione, brama, attaccamento, divenire. Quindi il divenire porta alla nascita.

Al momento della nascita, non si torna indietro. Il neonato non può tornare nel ventre materno. Nel momento del divenire è già quasi troppo tardi per spezzare il ciclo. Una volta che c'è la nascita è fatta, ne consegue inevitabilmente il corso della vita, e durante questo corso della vita ci saranno dispiaceri, pianto, dolore, perdita, disperazione, invecchiamento, malattia e morte. Dukkha. Seppure ci sarà sicuramente anche qualcosa di piacevole.

Una volta che c'è la nascita, una volta che a tentoni abbiamo acchiappato la condizione, una volta che abbiamo seguito qualcosa, abbiamo ceduto i nostri cuori, siamo presi all'amo. Il momento del divenire è quello della massima gratificazione; è quando otteniamo quello che vogliamo, la dolcezza dell'esca. Tutta la cultura consumistica si basa sul successo di bhava, del divenire, su quel momento positivo: ``Sì, mi piace, ce l'ho ed è mio''. Tutto è finalizzato a quel piacere, quella scarica di adrenalina. Una volta che si è avviato questo processo, ci sono anche le scariche sulla carta di credito: ``Ohi, ohi, come ho fatto a spendere tanto? Chi me l'ha manomessa? Chi mi ha preso il bancomat?''.

Il Buddha trascorse tutta la prima settimana osservando il modo in cui funziona il processo di rinascita. Trascorse la seconda settimana osservando il modo in cui il processo \textit{non} si innesca. Se la mente dimora in vijjā, nel conoscere, in rigpa, allora non c'è ignoranza. Se non c'è ignoranza, non c'è saṅkhārā. Se non c'è il saṅkhārā, non c'è nāma-rūpa, e così via. Il processo non si mette in moto. 

Egli trascorse tutta la seconda settimana esplorando la modalità con cui il cuore si libera da questo ciclo. Contemplò che quando non c'è ignoranza, non c'è problema. Non sorge l'alienazione, la separazione, la confusione del sentirsi strattonati dai pensieri, dalle sensazioni o dagli oggetti sensoriali. Nella terza settimana trascorsa sotto l'albero della bodhi, investigò contemporaneamente sia il sorgere sia la cessazione di dukkha, contemplando sia il processo ``con la corrente'' (\textit{anuloma}), sia ``contro corrente'' (\textit{patiloma}).

Così egli trascorse tre settimane stando semplicemente seduto e riflettendo su questo processo. Il nocciolo delle Quattro Nobili Verità è descritto qui: come sorge dukkha, come cessa, e ciò che dobbiamo fare perché cessi. In termini di Buddha-Dharma, questo è il seme, il cuore del seme, il nocciolo, la quintessenza.

In termini molto pratici, il problema non è la natura della realtà ultima, che si sostiene da sé, il problema è che il cuore perde di vista la realtà a causa della dipendenza dall'attaccamento.

Dopo l'illuminazione il Buddha diede il suo primo insegnamento a un asceta incontrato per strada. Il Buddha era particolarmente alto, incredibilmente bello e aveva il portamento di un nobile principe guerriero quale era stato. Inoltre era radioso, risplendente della sua recente esperienza dell'illuminazione. Era sicuramente una figura che colpiva e, mentre stava camminando, questo personaggio di nome Upaka lo fermò dicendo: ``Perdonami, sei così raggiante e luminoso. Il tuo viso è così trasparente. Di certo devi avere avuto una qualche realizzazione meravigliosa. Che tipo di pratica fai? Come sei arrivato a questo? Chi sei? Chi è il tuo maestro?''.

Il Buddha rispose: ``Non ho un maestro. Sono completamente auto-illuminato. In effetti, io sono l'unico essere illuminato in tutto il mondo. Non c'è nessuno che io possa considerare mio maestro o mio superiore. Soltanto io sono pienamente risvegliato''.

Upaka disse: ``Buon per te, amico mio''. Poi, scuotendo la testa, si allontanò rapidamente in un'altra direzione. (M 1.6).

Il Buddha si rese conto che questo tipo di risposta non funzionava. Se, incontrando qualcuno che gli chiedeva ``sembri felice, come ci sei arrivato?'' avesse risposto ``io sono la realtà ultima'', ovviamente non sarebbe stato il modo migliore per comunicare la propria scoperta.

Così il Buddha cambiò il proprio modo di porsi. Comprese che ``forse esordire bruscamente con la realtà ultima non funziona. Forse, se cominciassi a raccontare la storia dall'inizio\ldots{} cominciamo con marigpa, l'ignoranza; cominciamo dicendo come si sperimenta abitualmente la vita''. Noi non proviamo una beatitudine incessante, no? Se intuiamo che c'è una realtà ultima, pura, perfetta e beata, perché non la sperimentiamo di continuo? Che cosa si frappone? Ecco perché il Buddha iniziò con dukkha. Se intuiamo la perfezione e la purezza della nostra stessa natura, è ragionevole voler sapere perché non è sempre in cartellone. Il motivo risiede in avijjā e taṇhā, l'ignoranza e la brama.

Noi andiamo in cerca dell'analisi come fosse una diagnosi medica. Il problema è che `c'è dukkha'; la causa è un attaccamento auto-centrato; la prognosi è che `può essere risolta, la fine di dukkha è possibile'; la medicina è il Nobile Ottuplice Sentiero.

\section*{Le vie di fuga dal ciclo} 

Per molte persone il problema è se questi insegnamenti sull'origine dipendente sono pratici o meno. Però non abbiamo bisogno di andare a cercare molto lontano per vedere questo schema nella nostra vita di tutti i giorni. Possiamo vedere come il ciclo dell'origine dipendente si svolge ripetutamente nel nostro essere un istante dopo l'altro, un'ora dopo l'altra, un giorno dopo l'altro. Rimaniamo intrappolati nelle cose che amiamo, le cose che odiamo, le cose su cui abbiamo un'opinione, nei sentimenti verso noi stessi, nei sentimenti verso il prossimo, in ciò che ci piace, ciò che non ci piace, nella speranza e nella paura. Va avanti all'infinito. La buona notizia è che ci sono vari punti in cui possiamo interrompere questo ciclo e alla fine liberare il cuore.

Si potrebbe fare un seminario di un mese sull'origine dipendente senza esaurire il tema. Per cui qui vorrei darvi giusto alcuni punti chiave. 

Diciamo che il peggio è già avvenuto. Qualcosa di molto doloroso ha già avuto luogo. Siamo circondati dai cocci di vetro. Abbiamo litigato con qualcuno. Ci siamo appropriati di qualcosa che non ci appartiene. Siamo stati egoisti o avidi. Qualcuno ci ha fatto del male. Come ci siamo ficcati in questo pasticcio? Così è la vita. Stiamo sperimentando le sofferenze di dukkha, ma non c'è bisogno di sentirsi una vittima e rifugiarsi nella solita solfa ``perché io?''.

Uno degli insegnamenti più belli del Buddha è che l'esperienza della sofferenza può andare in due direzioni. Da una parte, può generare abbattimento e confusione. Dall'altra, può maturare nella ricerca. Quando tutto va storto, abbiamo una scelta. Ci limitiamo a crogiolarci in questi pensieri oppure diciamo: ``Perché è così? Cosa sto facendo per farne un problema?''. La ricerca entra in gioco per scoprire dove è che ci attacchiamo e perché cerchiamo la felicità dove non c'è (A 6.63).

Anche alla fine del ciclo (nascita, invecchiamento, malattia, morte, sofferenza, pianto, dolore, perdita e disperazione) possiamo usare quel dolore come leva per risvegliarci. Di fatto, il Buddha sottolinea in alcuni insegnamenti che è proprio l'esperienza di dukkha che può far sorgere la fede (S 12.23). Il dolore ci dice: ``Questo fa veramente male. Eppure, in qualche modo so che questa non è la realtà ultima''. Sappiamo anche che ``io \textit{posso }fare qualcosa a questo riguardo; dipende da me''. Così nasce la fede che qualcosa si può fare e la fede è ciò che ci proietta verso il cammino della trascendenza.

Un'altra area di investigazione è il legame fra sensazione e attaccamento, fra \textit{vedanā} e taṇhā. Taṇhā alla lettera significa `sete'. Spesso viene tradotta con `desiderio', ma ci sono desideri sani e desideri non sani. Ecco perché attaccamento è una parola decisamente migliore. Contiene un elemento intrinsecamente agitato, frenetico, `io, io, io'. La sensazione è un mondo di innocenza. Possiamo avere una sensazione di intensa beatitudine, eccitante e piacevole. Possiamo avere una sensazione estremamente dolorosa. Possiamo avere una sensazione vaga e neutra sia nel corpo sia nella mente. La sensazione di per sé è assolutamente innocente, non possiede affatto una qualità intrinsecamente positiva o negativa. Se c'è sufficiente consapevolezza, tutti i fenomeni mentali e sensoriali, così come tutte le sensazioni piacevoli, spiacevoli o dolorose ad essi associate possono essere conosciute, senza attaccamento, come apparenze. Non appena l'ignoranza, marigpa, si affaccia, il cuore comincia a bramare. Se è bello: ``Lo voglio''. Se è brutto: ``Vattene. Fa schifo''. Da qualche parte in mezzo a questi due estremi in genere ci creiamo un'opinione al riguardo. In meditazione possiamo chiaramente spezzare il ciclo in questo punto in cui possiamo impedire di rinascere, in cui possiamo stare con la qualità dell'integrità del Dharma. C'è la sensazione, la vista, l'odore, il gusto e il tatto, e noi riconosciamo le emozioni che vi si accompagnano, ma è solo il mondo delle sensazioni, puro e innocente.

Spesso incoraggio le persone a fare esercizi sull'attaccamento, così che possano veramente arrivare a capire la qualità di questa esperienza in modo diretto. Bramate o attaccatevi a qualcosa deliberatamente, fino a quando arrivate a comprenderne la struttura così chiaramente che non appena si affaccia, anche nelle forme più banali, ne siete consapevoli. È come prendere un oggetto e trattenerlo (\textit{solleva il batacchio della campana e lo tiene saldamente}). Questo è l'attaccamento. C'è l'attaccamento ma, se lo mantenete con consapevolezza, lentamente l'attaccamento cessa (\textit{rilassa la mano}). Non lo gettiamo via, non lo rompiamo. Allentiamo la presa. Facciamo questi piccoli esercizi in modo che ci possiamo rendere conto quando una sensazione si trasforma in `lo voglio', o `devo', o `non dovrebbero' o `di più', `di meno' o `fatemi uscire da qui', o `ahia', la riconosciamo come brama e attaccamento. Anche nei suoi aspetti più sottili siamo in grado di vedere che `c'è l'attaccamento'.

Gli insegnamenti Dzogchen offrono una stupenda analisi dell'anatomia dell'attaccamento. Ne descrivono tutti gli aspetti e le forme più sottili. È molto utile farlo coincidere in maniera così sottile con la nostra esperienza diretta, come fare la mappatura dei minimi particolari dei tessuti di cui siamo fatti fisicamente. Arrivando a conoscere la qualità dell'attaccamento, possiamo riconoscere che, non appena lo lasciamo andare, il problema è risolto. Va tutto perfettamente bene così com'è. Il ciclo della rinascita si interrompe proprio qui.

C'è una storia raccontata da Ajahn Sumedho che fa al caso nostro. È una delle sue storie che mi si è scolpita nella mente. Ebbe luogo molti anni fa quando era un giovane monaco nel monastero di Ajahn Chah, un monastero estremamente rigoroso dove la vita era particolarmente austera. Non accadevano cose emozionanti; c'era sempre un'atmosfera di desolazione e di calma piatta per i sensi. Non solo, al Wat Pah Pong si serviva il cibo peggiore al mondo. Se i cuochi cercavano di renderlo più saporito, Ajahn Chah andava in cucina e li redarguiva. Occasionalmente c'era una tazza di the una o due volte a settimana. Altre volte c'era un succo di frutta bollente o una bevanda fermentata locale chiamata \textit{borupet}. Il borupet è una sorta di vite amara di cui è impossibile descrivere il sapore. Se togliete la corteccia a un vecchio albero e assaggiate la linfa, quello è il sapore. È veramente disgustoso e ti lega la bocca. Pare che faccia molto bene, ma è una roba orribile da bere o mangiare.

Ogni settimana c'era una veglia che durava tutta la notte e i novizi portavano delle grosse caraffe fumanti. Una volta Ajahn Chah chiese al nuovo arrivato Ajahn Sumedho, che aveva una grossa tazza: ``Ne vuoi un po'?''. Ajahn Sumedho, non sapendo cosa contenesse la caraffa, disse: ``Riempila'', o qualcosa di simile. Gli riempirono la tazza di borupet. Ne bevve un sorso e reagì con un sonoro ``Puaaaaaaà!''. Ajahn Chah sorrise e disse: ``Bevila tutta, Sumedho''. Sì, attaccarsi al desiderio è dukkha.

\section*{``Mi piace, ma non lo voglio''}

Ecco un'altra storia che racconta spesso Ajahn Sumedho. A volte Ajahn Chah lo invitava ad accogliere i visitatori che arrivavano al monastero. Un giorno arrivò al Wat Pah Pong un gruppo di giovani donne attraenti. Credo che studiassero da infermiere presso la locale università di Ubon Ratchathani. Alcune decine di loro erano sedute con un atteggiamento molto rispettoso, nelle loro belle divise bianche e turchesi. Ajahn Chah tenne per loro un discorso di Dharma e chiacchierò con i loro insegnanti, con i professori e così via. Ajahn Sumedho rimase seduto al suo fianco durante le svariate ore per cui si prolungò la riunione. Sedere in compagnia di così tante belle ragazze in così stretta intimità non era qualcosa che accadesse di frequente al giovane Bhikkhu Sumedho.

Ad Ajahn Chah piaceva mettere ogni tanto alla prova i suoi discepoli per vedere a che punto fossero e così, dopo che il gruppo di studentesse fu partito, Ajahn Chah si volse e disse: ``Allora, Sumedho, come ti senti? Che effetto ha avuto sulla tua mente?'' Tenete presente che nel Sudest Asiatico il rapporto con la sessualità in generale è molto più diretto; e Ajahn Sumedho disse in thailandese ``\textit{chorp, daer }\textit{my ao}'', che vuol dire ``mi piace, ma non lo voglio''. Ajahn Chah fu molto soddisfatto della risposta. Di fatto, ne fu così colpito che nelle due o tre settimane che seguirono la citava: ``Questa è la pratica essenziale del Dharma. C'è il riconoscimento che qualcosa è attraente, che è bello, ma poi c'è anche l'arbitrio: lo voglio \textit{veramente}? Ho bisogno di possederlo? Ho bisogno di corrergli dietro? No, non ne ho bisogno. Senza timore, repressione o avversione c'è un allontanamento''.

Se Ajahn Sumedho avesse bofonchiato: ``Sono rimasto lì seduto trasformandole tutte in cadaveri'', allora Ajahn Chah avrebbe forse pensato: ``Molto bene. Ma dà l'idea che sia un tipo incline all'avversione, spaventato dall'attrazione sessuale o dalla sfera della sessualità. Come monaco fa il suo dovere cercando di frenare le passioni, ma forse non è consapevole del Dharma più profondo''. Oppure se avesse detto: ``Tutto quello che sono riuscito a fare è inchiodarmi al pavimento'' Ajahn Chah avrebbe pensato: ``Giusta osservazione. È il tipo avido; dovremo convogliare con cura questa tendenza col tempo''. Ma Ajahn Chah vide che Ajahn Sumedho aveva veramente trovato la via di mezzo. ``È così com'è: è molto attraente, bello e delizioso, ma io non desidero possederlo. Non lo sto rigettando, ma nemmeno ne ho bisogno. È così com'è''.

\section*{All'inizio}

L'ultima parte di questo argomento di cui voglio parlare, in realtà sta all'inizio della storia, di fatto è il punto di partenza del ciclo dell'origine: \textit{Avijjā paccayā saṅkhārā}, l'ignoranza condiziona le formazioni. Ho partecipato a ritiri in cui Ajahn Sumedho si soffermava anche per tre settimane su questa singola frase: ``L'ignoranza condiziona le formazioni''. C'erano letteralmente due o tre discorsi di Dharma al giorno su ``L'ignoranza condiziona le formazioni''. Riassumeva questo insegnamento di Dharma in una frase e la ripeteva all'inifinito; ``L'ignoranza complica ogni cosa''. Cosa significa?

Saṅkhārā è un termine dall'accezione ampia che in sostanza significa: ``ciò che è composto'', e viene tradotto in molti modi: formazioni karmiche, miscugli, fabbricazioni, formazioni volontarie, dualismo soggetto/oggetto; c'è un'ampia costellazione di significati.

Ciò che la frase ``L'ignoranza complica ogni cosa'' indica è che non appena cessa vijjā, non appena si perde rigpa, istantaneamente i semi del dualismo cominciano a formarsi e germogliare. C'è un osservatore e un osservato, c'è questo e quello; un qui e un lì; un io e un mondo. Anche a livello più sottile ed embrionale sta parlando di questo. Non appena c'è avijjā, ci sono i presupposti per il saṅkhārā. Poi diventa un vortice; basta un movimento infinitesimale e inizia a crescere in maniera esponenziale. \textit{Saṅkhārā} \textit{paccayā viññāṇaṃ}: il saṅkhārā condiziona la coscienza. La coscienza condiziona la mente-corpo. La mente-corpo condiziona i sei sensi. I sei sensi condizionano la sensazione, la brama e così via.

Quando arriviamo ai sei sensi c'è il corpo qui e il mondo lì, che spe\-ri\-men\-tiamo come realtà apparentemente solide.

Se il processo è appena avviato, si tratta di bloccarlo in fretta. Possiamo fare un passo indietro e vedere dove sono stati creati un osservatore e un osservato. Si dice che ``saṅkhārā fa capolino'' come una tartaruga; intendendo che una qualche forma sta cercando di infilare la testa dentro rigpa. Ma se l'80\%\ di rigpa, il conoscere, è presente, facciamo ancora in tempo ad acchiapparla e tornare a rilassarci nella consapevolezza aperta.

Stiamo parlando dell'area sottile di movimento in cui, non appena c'è un cedimento della presenza mentale o la più tenue colorazione o distorsione di quella consapevolezza, si insinua il dualismo. Questo è il seme di tutta la faccenda. Se lo si vede a questo punto senza assecondarlo, quel seme, quel movimento primordiale, non crescerà ulteriormente, si interromperà proprio lì. Se non lo si vede, il vortice si rafforzerà fino a che c'è un ``io qui, il mondo là fuori''. E allora: ``Lo voglio, non lo sopporto, devo averlo. È meraviglioso, stupendo, sto conquistando qualcosa'': dolore, pianto, sofferenza, perdita e disperazione.

\vspace*{-1.2em}
\section*{Fame insaziabile}

\vspace*{-0.8em}
Cosa accade a quell'estremità del ciclo, quando dukkha non è maturato nella ricerca della verità e abbiamo lasciato che la nostra sofferenza diventasse composta? Ci sentiamo incompleti. Ci sono ``io'' che mi sento infelice, derelitto, insicuro, incompleto, alienato. Poi, non appena c'è un'idea o una sensazione o un'emozione o un oggetto sensoriale che potrebbe farci sentire nuovamente completi, gli saltiamo addosso. ``Questo mi sembra interessante. Forse questo farà al caso mio''.

C'è una sensazione di fame, un vuoto, un desiderio che deriva dall'esperienza della sofferenza. Se non siamo svegli a quello che sta accadendo, pensiamo che ciò che ci manca sia qualche \textit{cosa}, un nuovo lavoro, una nuova auto, un nuovo partner. Oppure ci manca la salute perfetta. Ci manca un'accettabile pratica di meditazione. Non dovremmo frequentare i lama tibetani; dovremmo unirci ai Theravādin ad Abhayagiri. Dovremmo riavvicinarci al cristianesimo. Dovremmo trasferirci alle Hawaii. E così all'infinito. Andiamo appresso a ogni tipo di oggetto esterno o di programma interiore per trovare il pezzo mancante.

Questo è il ciclo della dipendenza, e si tratta di una esperienza molto comune. Sono sicuro che tutti avete provato esperienze simili. Nonostante le nostre migliori intenzioni, ci ritroviamo nei pasticci. Scopriamo di aver tentato di soddisfare un qualche desiderio, di un lavoro, di un partner, di una tecnica di meditazione, un insegnante, un'auto, qualcosa che ci soddisfi. Poi lo otteniamo e pensiamo: ``Adesso sì''. Ma è veramente così?

Un po' di tempo fa, al monastero arrivarono due nuove auto. Una era un furgone scoperto. Una Ford F250 V8 da 5,7 litri con le sponde di legno. Il primo giorno che l'avevamo notai che qualcuno aveva ammaccato la targa posteriore. Avevamo questo furgone solo da un giorno e c'era già l'attaccamento. Ero indispettito e volevo sapere: ``Chi è stato? Chi è andato a sbattere contro il nostro furgone?'' Questo è dukkha.

Era stato così facile cedere alla sensazione che quel nuovo furgone Ford F250 V8 da 5,7 litri bianco ci avrebbe fatto felici e reso le cose molto più semplici. E in un certo senso è vero, non c'è dubbio. Per un po' è gratificante. Quella sensazione c'è davvero, è reale. La mente si contrae intorno alla sensazione e in quel momento di ``sì'' noi ci sentiamo assolutamente gratificati. L'universo si è ridotto a quel punto infinitesimale: ``Io felice. Ho una bella cosa''. Il problema è che l'universo in realtà non è così piccolo. Riusciamo a tenerli insieme solo fin tanto che dura l'eccitazione. Assaporiamo un cibo delizioso, facciamo un ritiro che ci ispira, vediamo un film emozionante, ci gustiamo l'odore di un'auto nuova, e poi finisce. Questi oggetti non ci soddisfano più. Il posto dove mancava quel dato oggetto è di nuovo vacante e c'è di nuovo dukkha. Se non ci rendiamo conto di quello che sta avvenendo, cercheremo un altro oggetto per riempire quel vuoto e il ciclo della rinascita si perpetua all'infinito. Accade migliaia di volte al giorno. Osservate il processo voi stessi, prendete nota. Scoprirete che avviene molto rapidamente. Ajahn Chah era solito dire che seguire l'origine dipendente è come lasciarsi cadere da un albero e tentare di contare i rami durante la caduta. È così veloce. L'intero processo può svolgersi dall'inizio alla fine nell'arco di un secondo e mezzo. Wow! Quasi non riusciamo a vedere che sta avvenendo e, bum, sappiamo quanto fa male quando atterriamo. In ogni momento possiamo vedere l'urgenza di afferrarci. Quando lo vediamo chiaramente; quando ci è diventato veramente familiare, possiamo interrompere il processo e lasciar andare il ciclo di nascita e morte.

Per favorire questa familiarità e il lasciar andare è importante sperimentare e riconoscere gli svantaggi dell'esistenza ciclica. Innanzi tutto fa male. Tanto l'eccitazione è autentica, così è il dolore. Non c'è l'eccitazione senza il dolore. Sarebbe bello, no? Quando il dolore arriva ci rendiamo conto che è vuoto. Quando l'eccitazione arriva, la sperimentiamo come assolutamente reale. Bisogna reagire veramente in fretta per scacciarla. Quello che è certo è che ci sono un sacco di persone che ci provano. Mentre sorge il piacere noi ci sentiamo: ``reali, reali, felici, felici, felici, felici''. Quando le sensazioni piacevoli si affievoliscono cerchiamo di vedere che il dolore e la delusione sono ``vuoti, vuoti, vuoti, vuoti''. Come diciamo noi in California ``Continua a sognare''. La vita non è così. 

