
\begingroup\setlength{\parindent}{0em}\setlength{\parskip}{0.8em}

In genere, si sono adottati i termini pāli quando ci si riferisce agli insegnamenti Theravāda e quelli sanscriti quando ci si riferisce agli insegnamenti Mahāyāna e Vajrāyāna, tranne quando i termini sanscriti (come ad esempio Dharma) sono più diffusi in Occidente.

\textbf{Ahaṃkāra} il senso dell'io, letteralmente, ``fatto di `io sono'\thinspace''

\textbf{Akāliko} senza tempo

\textbf{Amatadhātu }l'elemento del senza-morte; sinonimo di nirvāṇa / nibbāna

\textbf{Amrita} (sanscrito) il nettare dell'immortalità

\textbf{Anāgāmin} letteralmente, ``colui che non ritorna''; colui che ha raggiunto il penultimo stadio dell'illuminazione

\textbf{Anattā} letteralmente, ``non-sé''; una delle tre caratteristiche di tutti i fenomeni

\textbf{Anicca} impermanenza, incertezza; una delle tre caratteristiche di tutti i fenomeni

\textbf{Ananta }infinito, illimitato

\textbf{Anidassana }non manifesto, invisibile, privo di forma

\textbf{Anuloma }letteralmente, ``con la corrente''; si riferisce alla dimensione del sorgere del ciclo dell'origine dipendente

\textbf{Arahant }un essere completamente illuminato

\textbf{Arūpa-jhāna }assorbimento privo di forma; i gradi più raffinati di concentrazione meditativa

\textbf{Āsava }gli ``efflussi''; le abitudini non salutari del cuore: desideri sensoriali, opinioni, divenire e ignoranza

\textbf{Asurā }gli dèi invidiosi, i titani; una delle sei dimensioni della cosmologia buddhista, simbolo dell'indignazione giustificata e della forza combinata con la violenza

\textbf{Atammayatā }letteralmente, ``non fatto di questo''; non identificazione o non fabbricazione 

\textbf{Attavadupādāna }attaccamento al concetto e al senso dell'io

\textbf{Atthi }il verbo ``essere'' implicante lo stato trascendente

\textbf{Avalokiteshvara }(sanscrito) letteralmente, ``Colui che ascolta i suoni del mondo'', il bodhisattva della compassione, conosciuto anche come Chenrezig (tibetano) e Kuan Yin (cinese)

\textbf{Avijjā }ignoranza, nescienza, inconsapevolezza; uno degli anelli della catena dell'origine dipendente; ``marigpa'' in tibetano

\textbf{Bhava }divenire, essere; uno degli anelli della catena dell'origine dipendente

\textbf{Bhikkhu }monaco buddhista; letteralmente ``colui che vede il pericolo del saṃsāra'' o ``colui che vive di questua''

\textbf{Bodhi }(albero della) l'albero ai piedi del quale il Buddha sedette nella notte della sua illuminazione

\textbf{Borupet }un'amara vite medicinale, originaria della Thailandia

\textbf{Brahmā }gli dèi delle dimensioni più raffinate della cosmologia buddhista

\textbf{Brahmavihāra }i quattro stati mentali sublimi o divini, che rappresentano il mondo emotivo al suo grado più raffinato e più salutare; sono mettā (gentilezza amorevole), karuṇā (compassione), muditā (gioia per la fortuna degli altri), upekkhā (equanimità); conosciuti anche come ``dimore piacevoli'', sebbene non siano di per sé trascendenti

\textbf{Buddha-Dharma }(sanscrito) / \textbf{Dhamma} (pāli) gli insegnamenti del Buddha

\textbf{Ch'an }(cinese) letteralmente, ``assorbimento meditativo''; è il termine cinese che traduce il pāli ``jhāna'' e che in giapponese corrisponde a ``zen''

\textbf{Chenrezig }(tibetano) cfr. Avalokiteshvara

\textbf{Citta }cuore o mente

\textbf{Cittaṃ pabhassaraṃ āgantukehi kilesehi }``il cuore è intrinsecamente radioso, le contaminazioni sono solo visitatori''

\textbf{Deva} / \textbf{devatā }esseri celesti delle dimensioni paradisiache

\textbf{Dharma} (sanscrito) / (\textbf{Dhamma} (pāli) la verità delle cose così come sono: gli insegnamenti del Buddha che rivelano la verità e illustrano gli strumenti per farne esperienza diretta

\textbf{Dharmakāya} (sanscrito) letteralmente, ``il corpo del Dharma''; l'elemento immanifesto dei tre corpi del Buddha, secondo gli insegnamenti della tradizione settentrionale

\textbf{Dharmasālā} (sanscrito) rifugio per pellegrini

\textbf{Di}\textbf{ṭṭ}\textbf{hupādāna }attaccamento a opinioni e punti di vista

\textbf{Dukkha }sofferenza, insoddisfazione; l'insicurezza, l'instabilità e l'imperfezione intrinseche nelle cose; una delle tre caratteristiche di tutti i fenomeni

\textbf{Dukkha-nirodha }la cessazione di dukkha; la Terza Nobile Verità

\textbf{Dzogchen} (tibetano) letteralmente ``la grande perfezione naturale'', ``grande picco'', ``grande vetta''; corrisponde al termine sanscrito mahā-ati

\textbf{Essenza della mente }l'attributo trascendente, incondizionato della mente

\textbf{Hoti }il verbo essere implicante la condizione mondana, condizionata

\textbf{Jhāna }assorbimento meditativo

\textbf{Kamupādāna }attaccamento al piacere sensoriale

\textbf{Khandha} (pāli) / \textbf{skandha} (sanscrito) gruppo, insieme o aggregato; di solito si riferisce a uno dei cinque costituenti della sfera mentale e fisica: forma (specialmente il corpo), sensazione, percezione, formazioni mentali e coscienza

\textbf{Kuan Yin} (cinese) cfr. Avalokiteshvara

\textbf{Ku}\textbf{ṭ}\textbf{ī }capanna, residenza monastica

\textbf{Loka }mondo, dimensione, oppure universo

\textbf{Luang Por} (thailandese) forma rispettosa e affettuosa che significa ``venerabile padre''

\textbf{Mahāyāna }il Grande Veicolo, o tradizione settentrionale del buddhismo

\textbf{Mamankāra }me, egoità, letteralmente ``fatto di `me'

\textbf{Mañjushrī} (sanscrito) il bodhisattva della saggezza

\textbf{Marigpa} (tibetano) ignoranza (cfr. avijjā)

\textbf{Mettā }gentilezza amorevole, uno dei brahma-vihāra

\textbf{Nahm lai ning} (thailandese) acqua corrente immobile 

\textbf{Nāma-rūpa }mente-corpo, nome-forma, soggetto-oggetto, uno degli anelli dell'origine dipendente

\textbf{Namo tassa bhagavato arahato sammāsambuddhassa}``Omaggio al Beato, Nobile e Perfettamente Illuminato''; la classica frase usata per introdurre la maggior parte delle cerimonie, recitazioni di insegnamenti e benedizioni in pāli

\textbf{Nandanā} (bosco di) un giardino del piacere nel paradiso dei trentatré dèi

\textbf{Ngondro} (tibetano) pratiche preliminari

\textbf{Niraya }la condizione infernale; una delle sei dimensioni della cosmologia buddhista, simbolo degli stati di rabbia, sofferenza estrema e passione

\textbf{Nirodha }cessazione; sinonimo di nirvāṇa (sanscrito) / nibbāna (pāli)

\textbf{Nirvāṇa} (sanscrito) / \textbf{Nibbāna} (pāli) pace, la meta del sentiero buddhista; letteralmente ``freschezza''

\textbf{Nyingmapa} (tibetano) letteralmente ``gli Antichi''; la più vecchia scuola del buddhismo tibetano, coloro che conservano e trasmettono gli insegnamenti dello Dzogchen

\textbf{Paccaya }condizionare, causare, influenzare

\textbf{Pāramī} / \textbf{pāramitā }perfezioni spirituali

\textbf{Parinibbāna} (pāli) / \textbf{parinirvāṇa} (sanscrito) nirvāṇa definitivo o completo; termine usato solitamente in riferimento al trapasso di un essere illuminato

\textbf{Pa}\textit{\textbf{ṭ}}\textbf{iloma }letteralmente ``contro corrente''; si riferisce alla dimensione di cessazione del ciclo dell'origine dipendente

\textbf{Peta} (pāli) / \textbf{preta} (sanscrito) spiriti affamati; una delle sei dimensioni della cosmologia buddhista, simbolo dello stato di dipendenza insaziabile

\textbf{Poo roo} (thailandese) ``colui che conosce''; la facoltà del conoscere

\textbf{Pūjā }recitazione devozionale delle scritture e della pratica rituale

\textbf{Rigpa} (tibetano) consapevolezza non duale; riconoscimento autentico dell'essenza della mente; noto anche come ``la visione''; l'equivalente pāli è ``vijjā'', quello sanscrito ``vidyā''

\textbf{Rinpoche} (tibetano) letteralmente ``il prezioso'', titolo onorifico di solito conferito ai lama che si ritiene abbiano sviluppato molte perfezioni (pāramitā) nelle vite precedenti

\textbf{Rūpa-khandha }forma o aspetto fisico dell'esistenza; uno dei cinque khandha (cfr.)

\textbf{Sabbato pabha }che si irradia in tutte le direzioni, o accessibile da tutti i lati

\textbf{Samaṇa Gotama }il Buddha; letteralmente ``l'asceta itinerante della famiglia dei Gotama''

\textbf{Saṃsāra }letteralmente ``vagare senza sosta''; la dimensione di nascita e morte

\textbf{Sa}\textbf{ṅ}\textbf{khārā }formazioni mentali; uno dei cinque khandha; uno degli anelli della catena dell'origine dipendente

\textbf{Sīla }virtù; precetti morali

\textbf{Sīlabbatupādana }attaccamento alle regole, alle convenzioni e alle osservanze

\textbf{Skandha} (sanscrito) / \textbf{khandha} (pāli) cfr. khandha

\textbf{Suññata }vacuità

\textbf{Sutta} (pāli) / \textbf{sūtra} (sanscrito) letteralmente ``filo''; insegnamento delle scritture

\textbf{Tan }titolo onorifico che significa ``venerabile amico'', usato in Thailandia per i monaci più giovani e per i novizi

\textbf{Taṇhā }letteralmente ``sete''; brama; uno degli anelli della catena dell'origine dipendente

\textbf{Tārā }(sanscrito) letteralmente ``Colei che porta al di là''; una bodhisattva nata da una delle lacrime di Avalokiteshvara; è l'aspetto della saggezza del Buddha Amoghasiddhi

\textbf{Tathatā }quiddità

\textbf{Tathāgata }l'epiteto adottato dal Buddha per parlare di se stesso; letteralmente ``giunto alla quiddità''

\textbf{Theravāda} letteralmente ``la Via degli Anziani''; la tradizione meridionale del buddhismo

\textbf{Trekcho} (tibetano) letteralmente ``tagliare''; un aspetto della pratica di meditazione del buddhismo tibetano

\textbf{Udāna} ``Versi ispirati del Buddha''; uno dei libri della raccolta dei discorsi canonici

\textbf{Upādāna }attaccamento, aggrapparsi; uno degli anelli della catena dell'origine dipendente

\textbf{Upadesha} (sanscrito) indicazione, istruzione

\textbf{Vajra} (sanscrito) letteralmente ``diamante'', ``indistruttibile'', ``fulmine'', si riferisce di solito all'aspetto supremo o ultimo delle cose

\textbf{Vajrasattva} (sanscrito) letteralmente ``essere indistruttibile''; un membro del pantheon tibetano che rappresenta l'incarnazione della saggezza di tutti i Buddha; è una figura estremamente significativa nella pratica Dzogchen

\textbf{Vajrāyāna} (sanscrito) letteralmente ``il Veicolo del Diamante'' o ``il Supremo Veicolo''; l'aspetto tantrico della tradizione settentrionale del buddhismo

\textbf{Vedanā }sensazione; uno dei cinque khandha; uno degli anelli della catena dell'origine dipendente

\textbf{Vijjā }sapere trascendente, vera conoscenza; cfr. rigpa

\textbf{Viññāṇa }coscienza discriminante; uno dei cinque khandha; uno degli anelli della catena dell'origine dipendente

\textbf{Vipassanā }visione profonda, intuitiva; meditazione di visione profonda

\textbf{Zafu} (giapponese) cuscino per la meditazione

\endgroup
