
\dropcaps{I}{n genere la pratica di vipassan\=a} è insegnata e coltivata con un'attenzione particolare e dettagliata sugli oggetti mentali, gli oggetti dell'esperienza. A prescindere dal fatto che gli oggetti siano sensazioni fisiche, pensieri, impressioni, suoni o emozioni, noi prestiamo una scrupolosa attenzione alla loro natura, osservando attentamente in che modo gli oggetti dell'esperienza vanno, vengono e mutano. Inoltre, incontriamo ciascun oggetto riconoscendone la natura impermanente e insoddisfacente. Pertanto, la finalità principale della pratica è l'attenzione precisa sull'oggetto, il mondo oggettivo.

Il mio addestramento nella vipassanā, sia nei monasteri di Ajahn Chah, sia nella tradizione Thailandese della Foresta in generale, non è stato così incentrato o associato a questa tecnica specifica. L'attenzione è sull'oggetto, ma occorre anche percepire lo sfondo di consapevolezza in cui l'oggetto appare. Da una parte si osserva l'andare e venire di sensazioni, pensieri, percezioni, e così via; ma dall'altra si sostiene anche quell'esperienza all'interno dello spazio di consapevolezza. L'oggetto è percepito nell'ambito di un contesto. Fare queste distinzioni significa generalizzare, ma in questo modo tento di presentare un quadro più ampio della pratica di vipassanā di quanto si faccia in genere. 

\section*{Essere il conoscere}

È inoltre importante passare dalla sfera oggettiva a quella soggettiva e alla qualità del conoscere. Vari maestri thailandesi, come Ajahn Chah, Ajahn Buddhadāsa e Ajahn Brahmamuni, ma anche molti altri stimati insegnanti di meditazione, parlavano spesso di lasciar andare completamente la sfera oggettiva e semplicemente essere il conoscere. C'è un'espressione thailandese, ``\textit{yoo gap roo}'', che alla lettera significa ``lì con il conoscere''.

La pratica di rigpa si riferisce a qualcosa di molto simile. Comporta specificamente distogliere lo sguardo dall'oggetto; scegliere deliberatamente di non prestarvi troppa attenzione, per portare invece la maggior parte dell'attenzione sulla natura del soggetto. C'è un non essere più attratti dalla seduttività dei sensi, mentre ci concentriamo, non identificandoci, sul soggetto.

In ultima analisi, proprio come negli insegnamenti della tradizione della Foresta Thailandese, rigpa vuol dire svuotare sia la sfera soggettiva sia quella oggettiva. Scopo della pratica è la consapevolezza priva di soggetto e oggetto. Il cuore dimora in rigpa, la qualità del conoscere aperto, spazioso, e intanto c'è il riconoscimento della natura intrinseca della mente stessa: è vuota, lucida, sveglia e luminosa. I thailandesi amano le allitterazioni; Ajahn Chah e Ajahn Buddhadāsa erano soliti ripetere la frase ``\textit{sawang sa-aht sangoup}'' per parlare di questa qualità. \textit{Sawang} significa ``radiosità'' o ``luce intensa''; \textit{sa-aht} significa ``puro''; \textit{sangoup} significa ``pacificato''. \textit{Sawang sa-aht sangoup}: radiosità, purezza e pace.

Pertanto, quando usiamo il termine ``vipassanā'', è importante sapere che comprende varie modalità di pratica, come questa ``essere il conoscere''; non si riferisce quindi solo a una specifica tecnica sistematica. Possiamo utilizzare tutta una serie di pratiche per arrivare alla qualità della liberazione, in cui si realizza la natura stessa della mente. Ci sono vari modi per sostenere lo svuotare e il lasciar andare, la disidentificazione da pensieri, sensazioni, corpo, mente e dal mondo che ci circonda. Tutto può esserci di aiuto verso queste realizzazioni. 

Quando ascolto gli insegnamenti Dzogchen spesso mi vengono in mente un paio di versi dai poemi del Terzo Patriarca Zen, in cui si dice: ``Tutto è vuoto, chiaro, auto-illuminante, senza tensione dell'energia mentale''. Questi insegnamenti sono in circolazione già da un po', non è vero? A me sembra che corrispondano esattamente alle istruzioni che riceviamo nella tradizione Tibetana, specialmente l'ultimo verso: ``senza tensione dell'energia mentale''; non c'è nessuno che faccia nessuna \textit{cosa.} Si riferisce alla qualità intrinsecamente pura e libera della mente. Seguiamo pratiche convenzionali, come calmare o illuminare la mente, oppure svegliare la mente, ma in realtà stiamo solo allineando la sfera del condizionato con la realtà fondamentale già esistente. Questa natura intrinseca della mente è già completamente in pace, piena di energia e assolutamente sveglia. Questa è la sua natura inerente.

\section*{Buddha, Dharma e Sangha}

Inoltre, è interessante riflettere sulla presa di rifugio ordinaria e la presa di rifugio straordinaria e sui vari livelli di comprensione che esse rispecchiano. In che modo possiamo considerare il Buddha, il Dharma e il Sangha come aspetti o modi in cui parlare delle qualità di vacuità, conoscere e lucidità? Un modo per indicare la lucidità è vederla come una combinazione o la coesistenza di conoscere e vacuità. Un altro modo è vedere queste due che si manifestano assieme come compassione, o attività compassionevole innata. 

Ritengo che sia un modo molto utile di parlarne: il Buddha è ciò che è sveglio, ciò che conosce, quindi prendere rifugio nel Buddha è prendere rifugio nella consapevolezza della mente. Il Buddha sorge dal Dharma. Il Buddha è un attributo, conoscere è un attributo di quella realtà fondamentale. Il Dharma è l'oggetto supremo, il modo in cui sono le cose. La sua caratteristica è la vacuità. Il Buddha è il soggetto supremo, ciò che conosce, ciò che è sveglio. Quando il soggetto supremo conosce l'oggetto supremo, quando la mente che conosce è consapevole di come sono le cose, ciò che compare è il Sangha, l'azione compassionevole. Il Sangha fluisce intrinsecamente da quella qualità. Quando c'è la consapevolezza di come sono le cose, gli abili mezzi compassionevoli sorgono e fluiscono naturalmente. I tre rifugi, come potete vedere, sono interrelati.

È utile considerarli semplicemente come attributi separati dell'identica qualità essenziale. Per esempio, l'acqua ha l'umidità. Possiamo parlare dell'umidità, ma non possiamo separare l'umidità dall'acqua. Ci sono altre qualità quali la fluidità e la temperatura dell'acqua che ugualmente non possiamo estrarre. Sono qualità distinte; le possiamo distinguere, ma non le possiamo separare. Quando investighiamo questa qualità di rigpa, la natura della mente, è utile vedere come tutti i suoi attributi siano intrinsecamente intrecciati e interrelati tra loro. Di fatto non li possiamo separare; formano un tutt'uno.

Un modo per tenere insieme tutto questo ci viene da una frase che Ajahn Chah usava ripetere spesso: ``Dentro è Dharma. Fuori è Dharma. Tutto è Dharma''. Sia che lo vediamo o meno, tutto è Dharma. È come dire del mare: ``Questa è acqua. Dentro è acqua. Fuori è acqua. Tutto è acqua''. La mente è Dharma. Il conoscere è Dharma. Il mondo fisico che ci circonda è Dharma. Tutti gli essere intorno a noi, ciascuno di essi, fanno tutti parte della natura. ``Natura'', tra l'altro, è un'altra traduzione per Dharma.

\vspace*{-1.2em}
\section*{Il Dharma consapevole della sua natura}

\vspace*{-0.8em}
Quando il cuore dimora in rigpa, possiamo dire che si tratta del Dharma consapevole della sua stessa natura. È facile pensare che questo modo di esprimersi ci possa inorgoglire un po'. Comunque, durante alcune pratiche di visualizzazione, come vedere il nostro corpo, la parola e la mente come il corpo, la parola e la mente del Buddha, creiamo anche un ideale di cui possiamo essere all'altezza. Queste pratiche ed espressioni ci aiutano a risvegliare l'intuizione su ciò che già c'è.

Visualizziamo il Buddha che emana qualità di gentilezza, compassione e saggezza, irradiandole verso di noi, e il nostro corpo, raggiunto da questi raggi, si riempie di luminosità. Diventiamo completamente posseduti da questo essere radioso. Tutto il ``piccolo io'' è stato scacciato, c'è solo questo Buddha che occupa il nostro guscio. È un modo per usare la nostra immaginazione, ma ha anche lo scopo di far scattare un'intuizione profonda, o agnizione. Può darsi che tra la saggezza della nostra mente e la saggezza del Buddha ci sia già solo un pelo di differenza. Solo che la qualità qui è in qualche modo più oscurata di quella che si trova nella mente o nella vita di un Buddha. Può succedere che prendiamo queste cose in maniera molto personale e che cominciamo a chiederci: ``Sto cercando di prendermi in giro? Mi si sta forse dicendo che sono una specie di stupendo \textit{avatar}?''. Cercate di mettere tutto questo da parte e usate queste frasi per aiutare a illuminare quelle restrizioni auto-limitanti e lasciarle andare; usatele come strumenti per aiutare il cuore a risvegliarsi a quella comprensione. 

Dire a noi stessi ``questo è solo il Dharma consapevole della sua stessa natura'' rispecchia la base della fede. Quale aspetto di voi non è Dharma? Quale aspetto di voi non è parte della natura? Potete nominarne uno? I vostri problemi? Le vostre ossessioni? Le vostre unghie incarnite? La vostra mente? L'essenza della mente? Le vostre idee? Di questo, tutto, ogni singola parte, appartiene all'ordine naturale. Non c'è nessun elemento del nostro essere, nulla di quanto ci circonda, fisico o mentale, che non appartenga all'ordine naturale. Quando diciamo ``questo è il Dharma consapevole della sua stessa natura'', anche quella `mente che conosce' è un aspetto del Dharma.

Vedere il Dharma che conosce se stesso può essere usato per pervadere le nostre percezioni auto-centrate e i modi abitudinari di pensare. È come demolirli con la luce della saggezza, li irradiamo, li inondiamo in modo che qualcosa in noi comincia a svegliarsi alla comprensione, ``Oh, forse anch'io\ldots{}'' ``Io? Un cigno?'' $[$\textit{canta}$]$\textit{.} ``E proprio un bel cigno! Guarda nel lago, guarda nel lago''. È esattamente quello che stiamo facendo. Andate a dare un'occhiata.

Questi versi del Primo Tsoknyi Rinpoche contengono parole bellissime: ``Nessun Buddha altrove. Guarda il tuo viso''. ``Guarda il tuo viso'' non significa guardati le rughe, o i brufoli, o i tuoi bellissimi occhi. Significa guarda il tuo viso \textit{originario}. C'è qualcosa oltre le rughe, la bellezza e la mediocrità. Quindi, quando andiamo al lago, cosa significa guardare il nostro viso originario?

Durante questo ritiro abbiamo tutti partecipato alla pratica di visualizzare il Buddha Vajrasattva, l'incarnazione della saggezza purificatrice. Potrebbe sorgere la domanda (così come è accaduto al giovane che chiedeva di Tara): ``Vajrasattva esiste davvero o no?''. Molti studenti che non sono abituati a queste pratiche potrebbero essere stati colti un po' di sorpresa da alcune delle visualizzazioni e delle istruzioni. Devo ammettere di essere rimasto un po' perplesso un paio di volte, quando mi è stato detto: ``Immagina Vajrasattva con la consorte, ricoperto di gioielli, che galleggia su un loto davanti a te con un disco lunare e un \textit{vajra}, e mentre ripeti il mantra dalle cento sillabe, visualizzalo intorno a queste''. Tanto per cominciare si presume che io conosca il sanscrito! E poi mi si chiede di visualizzare \textit{amrita} che scende come pioggia su Vajrasattva e la consorte per poi fluttuare verso di me. Forse sono riuscito ad evocare una vaga sensazione di una specie di luce dorata da qualche parte nelle vicinanze, ma il mio sanscrito non era abbastanza buono.

Naturalmente può darsi che per qualcuno sia tutto meravigliosamente chiaro, beato e liberatorio dall'inizio; ma a molti sicuramente scatena una tipica reazione scettica occidentale: ``Devo crederci, o devo sentirmi inadeguato perché non riesco a crederci? Oppure me ne sto qui seduto e sopporto con consapevolezza le mie reazioni emotive?'' O invece, ``mi do con tutto il cuore alla visualizzazione sperando che il suo significato mi si palesi in seguito?''. Possiamo rimanere incastrati in questo dilemma.

\section*{``Essere o non essere'' \`e la domanda sbagliata}

Una considerevole sezione degli insegnamenti del Buddha esplora proprio questa domanda. Uno dei trattati di Acarya Nāgārjuna, il \textit{Mūla Mādhyamaka Kārikā}, si basa in parte su un passaggio del Canone pāli. Nāgārjuna era un grande filosofo e molti dei suoi insegnamenti divennero dogmi centrali della tradizione Mahāyāna. Sembra però che molti suoi insegnamenti e commenti si basassero sul Canone pāli. Il brano specifico da cui deriva questa esegesi suona più o meno così: ``Quando si vede l'aspetto nascente dell'esperienza, il venire in essere del mondo, con retta saggezza, allora la `non-esistenza' rispetto al mondo non si verifica. E quando si vede con retta saggezza, come è in realtà, la cessazione del mondo, lo svanire delle condizioni, allora la `esistenza' rispetto al mondo non si verifica\ldots{} `Tutto esiste' è un estremo; `Niente esiste' è l'altro estremo. Invece di fare ricorso ad uno degli estremi, il Tathāgata espone il Dharma della Via di Mezzo: ``è con l'ignoranza come condizione che le formazioni vengono in essere\ldots{}'' (S 12.15). Quindi egli continua con tutto lo schema dell'origine dipendente.

È interessante che la parola ``rigpa'' sia una traduzione del temine pāli ``\textit{vijjā}''. Quello che troviamo in questo brano è ciò che generalmente inizia il ciclo dell'origine dipendente. Il Buddha non volle affermare l'esistenza, l'essere, e non volle affermare la non-esistenza. Egli rileva che sia ``l'esistenza è vera'' sia ``io sono'' si schierano con l'eternalismo. Mentre dire ``la non-esistenza è vera'' o ``io non sono'' si schiera con l'annichilimento, con il nichilismo.

Questo faceva impazzire di frustrazione i filosofi suoi contemporanei, perché non riuscivano a ottenere una risposta diretta. Invece, ogni volta, egli indicava che l'insegnamento del Tathāgata è la via di mezzo. Diceva inoltre che tutto il dualismo di esistenza/non-esistenza -- veramente lì/non veramente lì -- sorge a causa dell'ignoranza, perché non si vede chiaramente. Quando vijjā va perso, quando il conoscere va perso, c'è \textit{saṅkhārā}, il dualismo, questo/quello, soggetto/oggetto, qui/lì; e tutto il ciclo di io/altro, io qui e il mondo là fuori, viene in essere a forza di spinte. Da qui nascono tutti questi giudizi dualistici.

Quando le persone cercavano di incastrare il Buddha con la domanda sull'essere o non essere, si sentivano rispondere in questo modo: ``\thinspace'Esiste' non funziona. `Non esiste' non funziona. `Sia esiste che non esiste' non funziona. `Né esiste, né non esiste' non funziona. Il Tathāgata insegna il fatto che la verità è altro da questo. E' con l'ignoranza come condizione che le formazioni\ldots{}'' Alla mente critica questo può apparire come un approccio assolutamente privo di senso. ``Senti, dammi semplicemente una risposta diretta: una volta per tutte, l'io esiste o no?''. Forse è più utile affrontare queste questioni in maniera esperienziale, così da scoprirlo da soli.

\section*{Il potere di purificare}

Dobbiamo chiederci: ``Su cosa vertono questi insegnamenti?''. Si tratta di investigare il significato di ``Vajrasattva come incarnazione di tutti i Buddha'' al di là della valenza apparente. In termini più pratici e realistici, è importante domandarsi: ``Cosa potrebbe essere questa qualità? Qual è quella qualità dell'essere, interna o esterna, in grado di purificare completamente il nostro karma, di purificare completamente il cuore da ogni mancanza, ogni tipo di cattiva azione, ogni tipo di ostacolo negativo? Cos'è che ha questo potere purificatore?

Riflettendo in questo modo, vediamo che può essere solo la visione profonda della vacuità, la non-identificazione, il totale lasciar andare. È il riconoscimento che assolutamente nessuna cosa, materiale, fisica o mentale, \textit{potrebbe} essere me o mia. Qualunque cosa avvenga, qualunque cosa sia sperimentata, non è me, non è mia, non sono io, è vuota di esistenza reale. Quella intuizione stessa, non il concetto, ma proprio quella qualità di vedere significa che a prescindere da ciò che Mara ci butta addosso, a prescindere da quanto queste cose siano feroci, potenti o selvagge, noi le invitiamo a entrare: ``Prego, prego, entrate, siete tutte le benvenute. Chiunque si presenti, siete tutti invitati ad unirvi alla festa''. Non c'è problema, perché c'è questo riconoscimento fondamentale della vacuità.

I versi del Primo Tsoknyi Rinpoce recitano così:

\begin{quote}
\itshape
Mara è la mente che si aggrappa a piacevole e spiacevole. \\
osserva quindi l'essenza di questa magia, \\
libero dalla fissazione dualistica. \\
Realizza che la tua mente è purezza primordiale non costruita. \\
Non c'è un Buddha altrove. Guarda il tuo viso.
\end{quote}

Questi insegnamenti ci stanno dicendo che questa visualizzazione, l'invocazione del principio di Vajrasattva, ha soltanto lo scopo di risvegliarci alla qualità di saggezza trascendente che sta dentro di noi. Prendiamo un oggetto esterno perché in genere ci riesce più facile immaginare qualcosa di esterno piuttosto che immaginare qualcosa che ci riguarda. Quindi onoriamo quell'oggetto risvegliando e coltivando così l'intuizione che quella qualità è veramente già qui, dentro ognuno di noi. Alla fine fa sì che il cuore comprenda questa vacuità vivida, ed è ciò che intendiamo con ``osservare il nostro viso originario''.

\section*{Non c'\`e un Buddha altrove}

Questo cruciale insegnamento mi ricorda una storia che riguarda Ajahn Sumedho quando era un giovane monaco in Thailandia. I primi tempi in cui viveva nel monastero di Ajahn Chah, egli era un monaco molto zelante e super entusiasta. Dopo pochi mesi si era convinto che Ajahn Chah fosse il più grande maestro di Dharma, nonché il maestro più illuminato sulla faccia della terra. Era anche certo che il Wat Pah Pong fosse il miglior monastero del mondo e che il buddhismo Theravāda fosse la risposta a tutti i problemi. Era tutto infiammato; ma naturalmente, come sappiamo tutti, dopo un po' si esaurisce il carburante.

Trascorsero i mesi e gli anni, e Ajahn Sumedho cominciò a notare alcuni difetti nel modo in cui Ajahn Chah gestiva determinate situazioni, o in certe sue abitudini personali, come per esempio il fatto che masticasse la noce di betel. A nessun altro al monastero era permesso masticare il betel. Non che Ajahn Sumedho desiderasse masticare noci di betel, ma molti altri monaci sì. Anche se Ajahn Chah l'aveva proibito, lui poteva fare come gli pareva. Aveva inoltre proibito le sigarette, un'abitudine molto diffusa tra i monaci thailandesi. Era il primo monastero in Thailandia dove le sigarette erano proibite, ma Ajahn Chah continuava a fumare di tanto in tanto. Anche se aveva detto che avrebbe smesso di fumare, un giorno Ajahn Sumedho lo incrociò lungo un sentiero isolato con una sigaretta in bocca. Aveva colto il maestro in flagrante, ma Ajahn Chah si limitò a guardare Ajahn Sumedho facendogli un largo sorriso. Frustrazioni come questa si andavano accumulando in modo lento ma continuo.

Passò altro tempo. Da buon razionalista occidentale, alla fine Ajahn Sumedho decise che ne aveva abbastanza. Ajahn Chah era così acclamato dai thailandesi, i monaci, i laici e tutta la comunità monastica, che nessuno si sarebbe mai azzardato a criticarlo. Non c'era verso che le monache pronunciassero una parola. Anche i monaci, alcuni dei quali erano tipi piuttosto duri e diretti, provavano per Ajahn Chah un tale rispetto che nessuno di loro avrebbe mai detto niente. Ajahn Sumedho ci pensò su e decise: ``Bene, so che sono solo un giovane monaco, ma devo fare il mio dovere. Sarà meglio che mi prepari''.

Stilò un elenco dettagliato di tutte le mancanze di Ajahn Chah. Voleva essere preparato e avere chiari e pronti tutti i fatti da presentare al suo insegnante. Prese allora la lista, scelse il momento giusto e chiese ad Ajahn Chah: ``Sarebbe possibile parlare quando hai tempo? Avrei alcune cose da dirti''.

La vita di Ajahn Chah era piuttosto aperta e priva di complicazioni. In verità non aveva una vita privata. Sì e no dormiva nella sua capanna per circa quattro ore a notte. Tutto qui, era questa la sua vita privata. Il resto del tempo era un facile bersaglio.

Ajahn Chah accettò di parlare con il suo studente. E siccome Ajahn Sumedho non voleva mettere in imbarazzo il suo insegnante davanti a tutti, scelse un momento in cui non c'erano troppe persone in giro. Molto delicato da parte sua. Potete immaginare la lama che incombeva su di lui, pronta ad abbattersi. Ajahn Sumedho raccolse il coraggio e finalmente avvicinò Ajahn Chah. Aveva diligentemente imparato a memoria l'elenco di tutti i punti che sentiva il bisogno di sollevare. Cominciò a recitare il suo elenco dettagliato all'insegnante: ``Ti stai ingrassando, pesi davvero un po' troppo. Stai tanto tempo a parlare con la gente invece di meditare con noi e spesso quello che dici non è proprio buon Dharma. Sono solo chiacchiere e aria fritta, come quando parli del raccolto di mango di quest'anno o di come stanno i polli, oppure i consigli che dai a qualcuno su come prendersi cura dei bufali. A che serve parlare tanto della vita del nord-est della Thailandia? E poi per quanto riguarda le noci di betel e le sigarette mi sembra che tu usi due pesi e due misure, quando invece dovresti essere d'esempio per i monaci''.

Sia detto per inciso, sto un po' improvvisando qui, mi permetto una licenza poetica; ma vi prego di ricordare che Ajahn Sumedho stesso ha raccontato questa storia infinite volte, non si tratta di informazioni private. 

Alla fine, terminato il lungo elenco dettagliato, se ne sta lì, aspettando un secco diniego, o un rimprovero. In circostanze normali è ragionevole aspettarsi una reazione di questo tipo. Invece Ajahn Chah lo guardò con affetto e disse: ``Ti sono molto grato, Sumedho, per avermi fatto presente queste cose. Terrò a mente quello che mi hai detto e vedrò cosa si può fare. Ma anche tu non dovresti dimenticare che forse è un bene che io non sia perfetto. Altrimenti potresti cercare il Buddha fuori dalla tua mente''. Seguì un lungo silenzio commosso. Quindi il giovane Sumedho si congedò umilmente, rincuorato e castigato a un tempo.

\section*{Entrare nella stanza 101}

Ho già accennato alle intuizioni che ebbe Ajahn Chah durante i pochi giorni in cui ebbe occasione di studiare con Ajahn Mun: c'è la mente e ci sono i suoi oggetti, che sono intrinsecamente separati l'una dagli altri. Nel lessico Theravāda si parla di mente con la ``m'' maiuscola, Mente, e di oggetti mentali. Nella tradizione Dzogchen si usano espressioni simili per riferirsi a questa intuizione: c'è la mente (``m'' minuscola) e c'è l'essenza della mente. La parola ``mente'' qui è usata con il significato di mente condizionata, la mente dualistica, mentre il termine ``essenza della mente'' è usato per la mente incondizionata. C'è il condizionato e l'incondizionato. Come potete vedere, c'è una grande risonanza tra le due pratiche, anche se usano la stessa parola con significati diversi. 

A questo riguardo, Ajahn Mun ha scritto dei versi sull'illuminazione dal titolo ``La ballata della liberazione dai khandha'' degni di un applauso a scena aperta:

\vspace*{-1.2em}
\begin{quote}
\itshape
Il Dhamma sta come il Dhamma, \\
i khandha stanno come i khandha. \\
È tutto qui.
\end{quote}

Il termine sanscrito è \textit{skandha}: corpo, sensazioni, percezioni, formazioni mentali e coscienza. Quindi il Dharma è il Dharma e gli skandha sono gli skandha. C'è il condizionato; c'è l'incondizionato. Tutto qui. Questo è tutto ciò che abbiamo bisogno di sapere. 

Ajahn Chah aveva sentito questo da Ajahn Mun e ne era rimasto profondamente colpito. Però era anche solito parlare della circostanza in cui questo principio gli si era concretizzato. In Thailandia c'è una forte cultura dei fantasmi. Anche se molti di noi hanno smesso di avere paura degli spiriti da bambini, nella cultura thailandese gli spiriti sono molto presenti. Da bambino mi avevano raccontato storie abbastanza spaventose, ma quelle che i thailandesi raccontano ai loro figli sono veramente agghiaccianti, disgustose. Sono piene di sangue e viscere, di cattiveria e malvagità. Tutti crescono con questo tipo di immagini, quindi con una cultura del terrore dei fantasmi.

Questa paura aveva turbato anche Ajahn Chah. Monaco già da un po', era perfettamente consapevole di aver sempre evitato questa paura. Era qualcosa che non aveva mai veramente risolto. Da giovane era conosciuto come un ragazzo forte e pieno di fiducia in se stesso. Anche da adulto era un tipo piuttosto duro. Eppure dopo svariati anni di vita monastica i fantasmi ancora lo terrorizzavano e aveva una gran paura dei cadaveri. Ogni volta che soggiornava solo nella foresta era solito recitare versi protettivi per tenere lontani demoni e fantasmi.

Essendo uno cui piaceva andare fino in fondo, decise di prendere di petto la sua paura. Era così che affrontava i problemi ogni volta che c'era qualcosa da imparare. Decise che era ora di smettere di evitare la sua paura dei fantasmi; l'avrebbe affrontata una volta per tutte.

Ajahn Chah decise di montare la zanzariera e accamparsi sul luogo delle cremazioni fuori il villaggio in cui si trovava. Probabilmente ci risulta difficile immaginare la situazione, ma il monaco che ha scritto la sua biografia la paragona a quello che avviene nel romanzo \textit{1984}. Era la Stanza 101 dove uno si trova di fronte a tutte le sue paure più profonde, il proprio terrore più indescrivibile e primordiale. Andare nel campo delle cremazioni per Ajahn Chah era come entrare nella Stanza 101 per il protagonista di \textit{1984}. Diceva che gli ci era voluta tutta la sua forza di volontà per muovere un passo dietro l'altro. 

Via via che scendevano le tenebre, la sua mente urlava: ``Non essere ridicolo. Non farlo. Non fa bene al tuo \textit{samadhi}. Sii ragionevole. Potresti farlo più in là, l'anno prossimo, quando avrai un pratica più solida''. Ma si impose di rimanere e si accampò lì. Dopo aver montato la zanzariera, vi entrò e si sedette.

Durante il giorno c'era stata la cremazione di un bambino. Durante la cerimonia funebre Ajahn Chah era stato bene fin tanto che c'erano persone in giro. Poi tutti se n'erano andati e adesso lui era lì da solo. Nella sua biografia c'è una lunga descrizione di quella prima notte piena di immagini violente e dell'estremo sforzo di volontà cui dovette fare ricorso per arrivare al giorno seguente. Ajahn Chah era talmente terrorizzato che rimase immobile tutta la notte. All'alba si disse: ``Stupendo, ce l'ho fatta. Ho fatto questa cosa della cremazione. Me ne vado''. Conoscete questo tipo di reazione: ``Ho fatto il mio dovere. Bene, ho superato la prova. Ho sofferto abbastanza. Non ho più bisogno di farlo. Ho il permesso di andare adesso?''.

Invece Ajahn Chah aveva compreso: ``No, no, no. Questo non è trascendere la paura; è solo sopportarla. Non l'ho affatto superata. Sono ancora terrorizzato. Posso anche giustificarmi dicendo che non sono obbligato a farlo, ma il terrore è ancora qui davanti a me e io sono determinato a superarlo''.

Era proprio contento di aver deciso di rimanere. Quindi pensò: ``Per lo meno non ci sarà un'altra cremazione''. Ma, com'era prevedibile, quel giorno morì un adulto. Quindi Ajahn Chah rimase durante la cremazione e poi, di nuovo, tutti se ne andarono. Tutto ciò che aveva per proteggersi erano la sua pratica e la sua zanzariera. Potreste pensare che una zanzariera non offra una gran protezione. Ma chiunque si sia mai accampato sull'Himalaya, a Yosemite, o in qualunque altro posto dove ci sono bestie selvagge, sa che anche lo strato di plastica più sottile o una rete possono farti sentire al sicuro. ``Orsi grizzly? Qual è il problema? Non li vedo nemmeno. Facile''.

Ci sono passato anch'io l'anno scorso quando mi ero accampato nel cuore della foresta ad Abhayagiri, dove ci sono orsi e leoni di montagna. Mi ero sistemato vicino a un torrente a tre chilometri dal resto del mondo, digiunando e dissetandomi con l'acqua del ruscello. I primi giorni ogni foglia o ramoscello che cadeva dagli alberi equivaleva ad almeno tre orsi e un leone di montagna. Ogni volta. Quando scendeva la notte poi, il numero degli animali triplicava. Dopo qualche giorno mi ero abituato, ma, credetemi, stare dentro la zanzariera con una candela mi faceva sentire come a Fort Knox. Senza problemi. Basta che spegniate la candela e i rischi aumentano. Sollevate la zanzariera e\ldots{} avrete una visuale a quasi 360\textdegree.

Ajahn Chah diceva: ``La mia zanzariera era come una fortezza circondata da sette mura concentriche. Addirittura la presenza della mia ciotola della questua era rassicurante''. Prese la risoluzione di rimanere lì seduto con le proprie sensazioni, sapendo che era il modo migliore per superare le sue paure. La notte precedente, mentre sedeva immobile al suo posto, c'erano stati i soliti rumori di animali, grilli che cantavano tutta la notte, foglie e rami che cadevano dagli alberi. Non c'era stato niente di speciale, i rumori familiari di sempre. La seconda notte le cose cambiarono. Era seduto lì quando, a mezzanotte circa, credette di sentire dei passi. Quando vivi nella foresta impari a riconoscere i vari rumori prodotti dagli animali. Conosci la differenza tra il rumore dei cervi e quello degli gli orsi. Lucertole e serpenti non fanno lo stesso rumore. 

Ajahn Chah era lì seduto e pensava: ``Sento dei passi. Non si tratta di un animale. È una creatura che cammina su due gambe e proviene dal fuoco''. Allora si disse: ``Non essere ridicolo. Forse è qualcuno del villaggio che viene a vedere se sto bene. Forse sono venuti a offrirmi qualcosa e, se è così, si avvicineranno e mi saluteranno''. Ciò nonostante era ben determinato a rimanere lì seduto con gli occhi chiusi. Poi sentì il rumore di passi, tump, tump, tump, che si avvicinavano sempre di più. Cominciò a irrigidirsi. Sudava copiosamente e si disse: ``Non farti prendere dal panico. È solo uno del villaggio col passo pesante''.

Nella sua immaginazione poteva vedere il corpo carbonizzato. Poteva vedere uno scheletro con le viscere di fuori, brandelli di carne bruciata che penzolavano, pelle e occhi che gli rotolavano lungo le guance e una bocca mezzo carbonizzata. Mentre sentiva questo fetido ammasso di carne avanzare verso di lui, si disse: ``Non crederci. È solo la tua immaginazione. Smettila; stai fermo, concentrati e lascia andare la paura''. Nel frattempo i passi si facevano sempre più vicini. Poi sentì che gli giravano intorno. Tump, tump, tump; e giravano, giravano. A quel punto era in uno stato di puro terrore. Era andato oltre l'ansia. Il corpo era paralizzato e sudava a dirotto; era completamente irrigidito.

Poi la presenza si avvicinò e si fermò proprio davanti a lui. Ajahn Chah era ancora determinato a tenere gli occhi chiusi, a non dare neanche una sbirciata. A quel punto era talmente sopraffatto dalla paura che scoppiò. La paura aveva raggiunto il culmine quando all'improvviso pensò: ``Per tutti questi anni ho recitato `Il corpo è impermanente, la sensazione è impermanente, le percezioni sono impermanenti. Il corpo è non io, le sensazioni sono non io, le percezioni sono non io, le formazioni mentali sono non io, la coscienza è non io'\thinspace''. Quindi non era solo spaventato, era anche molto concentrato e attento. L'intuizione gli balenò nella coscienza: ``Anche se questo è un terribile fantasma mostruoso che sta per aggredirmi, tutto ciò che può aggredire è ciò che \textit{non sono io}. Tutto ciò cui può fare del male è il corpo, le sensazioni, le percezioni, le formazioni mentali e la coscienza. Questo è tutto ciò che può essere danneggiato, e non sono io, non è me. Ciò che sa tutto questo non può essere toccato''.

Immediatamente la sensazione di terrore svanì. Era come accendere una luce. Scomparve completamente ed egli entrò in uno stato di incredibile beatitudine. Era passato direttamente da dukkha assoluto, dolore e paura agghiacciante, a una beatitudine straordinaria. La sua mente era sveglia e in quel momento udì i passi, tump, tump, tump, che si allontanavano. Alla fine scomparvero.

Non ne scoprì mai la causa.

Ajahn Chah rimase lì seduto immobile fino all'alba. Durante la notte piovve a dirotto. Lacrime di estasi gli scorrevano lungo le guance mescolandosi alla pioggia. Nulla al mondo avrebbe potuto farlo muovere.

Paragonare quell'esperienza di libertà dalla paura con il sordido terrore della prima metà della notte gli fece comprendere il fatto che il Buddha è il nostro vero rifugio. La mente di Buddha è il nostro rifugio. È quello il posto sicuro. O, come disse il Primo Tsoknyi Rinpoche: ``Non c'è nient'altro da cercare. Dimora nel tuo luogo''. Egli aveva compreso che: ``È tutto qui. A prescindere dalle circostanze, da ciò che mi trovo di fronte, è solo questione di fare questo, di ricordare questo''.

\section*{Paura di fallire}

È un racconto molto impressionante. Potremmo pensare: ``Luoghi dove si fanno cremazioni, fantasmi e tutto il resto, che c'entro io?''. Vediamo se riesco a spiegarvi il senso di questo insegnamento. Spesso mi piace citare un piccolo episodio che ritengo estremamente significativo. Un po' di anni fa fu fatto un sondaggio tra migliaia di persone. Credo che fosse stato organizzato dalla Facoltà di Psicologia di Harvard. Scopo del sondaggio era di scoprire cos'è che spaventa le persone. Tra le prime dieci cose, credo che morire di cancro fosse al quarto posto. Non ricordo cosa fosse al terzo posto; al secondo la guerra nucleare. Al primo posto c'era la paura di parlare in pubblico. 

In sostanza, significa che siamo più pronti a sopportare la distruzione di tutto il pianeta che a convivere con una brutta figura in pubblico (anche conosciuta in alcuni ambienti come morte dell'io). Non è interessante? Tra le dieci cose c'era anche perdere tutti i propri averi o essere aggrediti fisicamente. Non sono cose da poco, ma vale la pena notare che la paura numero uno era parlare in pubblico. Abbiamo più paura di morire sul palcoscenico che della morte stessa.

Molto di tutto questo ha a che vedere con il fatto che ai nostri giorni la morte fisica è rimossa e non è ammessa nella società americana. Anche quando sei nella bara si pretende che tu abbia l'aspetto di uno che sta andando a ballare. Ti fanno indossare un bell'abito elegante, ti mettono un garofano all'occhiello e sei tutto truccato pronto per uscire.

Anche invecchiare non è di moda. Se un paziente muore, il medico ha fallito. Non si riconosce il fatto che la causa della morte è la nascita. Culturalmente mettiamo in atto una negazione collettiva della morte fisica, così molti di noi non la vedono, è una pura astrazione. La maggior parte di noi non entrano mai in contatto con la morte fisica. Se non fai volontariato in un ospizio, o se non lavori in un ospedale, puoi diventare adulto senza aver mai visto un cadavere. Quando mio padre morì all'età di 80 anni, mia madre disse che era la prima volta che vedeva un cadavere.

Forse questa negazione è collegata alla paura di fallire. Pensate a quante volte vi siete detti o avete sentito qualcuno dire: ``Non mi importa di morire; vorrei solo non dover soffrire troppo. Veramente non ho paura della morte''. Però se chiedete a qualcuno: ``Cosa provi rispetto all'idea di fallire? Come ti sentiresti a fare una brutta figura in pubblico? Come ti senti quando ti impegni in un progetto e questo fallisce? Che effetto ti fa essere rifiutato dalla persona che ami? Come ti senti se ti si dice che ciò che hai fatto o ciò che rappresenti è assolutamente inutile o anche semplicemente noioso?''. Sapete di cosa sto parlando, no? Non c'è bisogno che continui. Ognuno di noi ha le sue fantasie preferite, tipo ``non sono più attraente come una volta; ero proprio un bell'uomo (una bella donna), ma adesso non c'è rimasto più niente''. Possiamo vedere che c'è più identificazione con il nostro aspetto, la nostra personalità, di quanta ce ne sia con la vita stessa del corpo. Almeno per la maggior parte di noi.

Sarebbe quindi più saggio e utile lavorare semplicemente con quelle situazioni in cui il nostro io è messo alla prova, piuttosto che andarsi a cercare drammatici rischi fisici o sport estremi. Personalmente, ho compreso che per tutta la vita avevo avuto paura del fallimento. Ho scoperto che sapevo fare un sacco di cose, ma sceglievo di fare solo quelle in cui ero sicuro di riuscire. Non accettavo di partecipare dove non mi sentivo sicuro. Dopo svariati anni di vita monastica, riuscivo a vedere chiaramente quanto fosse alto l'investimento per apparire sempre bravo. Gli insegnanti di Dharma sanno che una delle circostanze in cui si è più esposti è durante i discorsi di Dharma. All'inizio, notai che dopo aver tenuto un discorso, anche se sentivo che era andato tutto bene, avevo bisogno di ulteriori conferme. A volte, soprattutto nei monasteri, le persone sanno essere davvero poco reattive. Le luci sono basse, tu stai lì seduto sul trono e non riesci a vedere se i monaci e le monache stanno sonnecchiando. La posizione monastica di chi dorme è perfettamente bilanciata, ma totalmente involontaria. Ad ogni modo, dopo i discorsi me ne stavo nel piccolo spogliatoio che c'è ad Amaravatī a perdere tempo vicino alla porta. Tutti i monaci erano obbligati a passare per quella stanza per uscire dall'edificio. Mi aggiravo in uno stato confuso di ansia e paura. ``Mio dio, come sarà andata? Gli sarà piaciuto? Che avranno pensato di me? Sono bravo?''.

Mi bastava che qualcuno dicesse semplicemente: ``Ottimo discorso!''. E allora, ah! che bello! A volte però non succedeva niente fino al giorno seguente. Le persone mi passavano accanto e io ero ancora teso, incerto sul discorso. ``È stato un fiasco? Pensano che sia un idiota?'' fino a quando qualcuno mi si avvicinava per dire: ``Quello che hai detto ieri sera è stata la cosa più utile che abbia mai sentito. Davvero preziosa''. Ah! La beatitudine di nuovo. ``Che monaca saggia. Lei veramente capisce il Dharma''.

In realtà intendevo dire: ``Grazie per aver lusingato il mio io''. Cominciai a notare questo aspetto e mi dissi: ``Questo è un male che richiede un po' di attenzione''.

Ogni giorno, mentre facevamo colazione, eravamo soliti tenere una riunione durante la quale Ajahn Sumedho dava insegnamenti e si organizzava il lavoro della giornata. Ajahn Sumedho si trovava al centro, mentre il resto della comunità sedeva tutto intorno alla sala; io ero al suo fianco. Di quando in quando si creava l'occasione per un commento o una battuta e io ne approfittavo per dire qualcosa. Cominciai a notare che quando tutti ridevano alle mie battute provavo un piacere simile a quello di un gatto quando gli si dà la panna. ``Che bella sensazione!''. E cominciai a capire che c'era qualcosa di veramente malsano. Ero così dipendente da quella sensazione di piacere. ``Non è stato proprio bravo Amaro? Ha segnato un bel punto''. Altre volte, quando intervenivo per dire qualcosa di arguto e invece non rideva nessuno, mi frantumavo in mille pezzi. ``Mio dio, è terribile. Che disastro! che orrore!''. Non riuscivo nemmeno a esprimerlo a parole, mi sentivo distrutto. Cominciai a capire: ``Guarda, una sensazione la inseguo con tanta passione, mentre rifuggo dall'altra. Interessante''.

Presi quindi la decisione cosciente di concedermi deliberatamente di fallire. Questo approccio mi costò un bel po' di lavoro. Certo, nulla in confronto ad Ajahn Chah sul luogo delle cremazioni, ma per me fu altrettanto potente. Cominciai a correre maggiori rischi mettendomi in situazioni dove sapevo di non essere particolarmente bravo e lasciando che gli altri vedessero il mio egoismo, imparando a stare con quella sensazione di andare in pezzi, dell'io che viene frantumato (come per esempio quando non otteneva quello che voleva o non riceveva complimenti). In realtà mi ci volle un sacco di tempo solo per imparare a stare anche solo per cinque o dieci minuti con quelle sensazioni che mi gorgogliavano e blateravano dentro. Alla fine compresi: ``È solo una sensazione''. Quindi, quando mi volgevo verso le sensazioni e mi permettevo di rischiare un po' di più, cominciai a vedere quanta energia avevo investito per evitarle. C'era stata una corrente sommersa di paura e terrore.

Di fatto volgermi verso la sensazione, permetterne la presenza e andare a vedere il bisogno di evitare, proprio come Ajahn Chah aveva fatto nel luogo delle cremazioni, significava aver risolto solo metà del problema. Dopo un po' cominciai a vedere con chiarezza come questa fosse un'ottima opportunità per assistere alla caduta dei progetti dell'io. Con mia grande sorpresa, dopo un paio di anni, scoprii che mi faceva piacere essere frainteso dalle persone o che si facessero di me un'idea sbagliata, che non mi trovassero simpatico o che mi criticassero. Era sorprendente. Non lo dico come se si trattasse di un successo, ma perché ne ero sconcertato. Il cuore di fatto si rallegra più nel vedere la realtà delle cose così come sono, che delle lusinghe all'io.

Per quanto riguarda imparare ad avere fiducia nella qualità del conoscere, gettare le basi di rigpa e coltivarla, uno degli aspetti più difficili è il carico dell'io. Tutta l'area di ``io'' che faccio delle scelte, ``io'' che riesco, ``io'' che fallisco, tutto ciò che viene caricato di io, me e mio, è un campo di investigazione molto interessante. Quando facciamo una scelta che possiamo etichettare come un successo, proviamo una sensazione bellissima. C'è una luminosità morbida, calda e dorata che ci fa sentire proprio bene. Quando invece scegliamo o decidiamo qualcosa che non funziona e cade a pezzi, prendiamo il fallimento come qualcosa di personale. ``\textit{Io} ho fallito''. Ciò nonostante non escludiamo nessun elemento di tutto questo dalla comprensione profonda. Ogni singolo ambito in cui sorge l'io deve essere oggetto di consapevolezza, in modo che non vada ad ostruire la spaziosità e la luminosità innate. A prescindere da quanto densa e spessa possa sembrare la consistenza del senso dell'io, di fatto è trasparente. Non è un gran sollievo scoprire che è così?

