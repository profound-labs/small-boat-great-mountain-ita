
\dropcaps{T}{utti i libri sono il frutto} dell'impegno fisico, spirituale e intellettuale di molte persone. Innanzitutto vorrei esprimere la mia gratitudine al Ven. Tsoknyi Rinpoche per avermi dato l'opportunità di studiare sotto la sua guida, di insegnare insieme a lui e per essere stato così gentile da scrivere la prefazione di questo libro. Poi vorrei ringraziare Guy Amstrong, che per primo ha avuto l'idea di pubblicare questi discorsi, per il suo incoraggiamento e per la sua generosa prefazione.

I discorsi e i dialoghi sono stati trascritti da un gruppo di persone pazienti e scrupolose: Laura Collins, Kondañña, Joyce Radelet, Toby Gidal e Joan Andrai. La prima bozza editoriale è stata completata con grande perizia e lungimiranza da Ronna Kabatznick, assistita da Rachel Markowitz. A Joseph Curran si deve l'armoniosa sistemazione del testo. Marianne Dresser ha gentilmente contribuito alla redazione dell'indice, mentre il progetto editoriale complessivo è stato curato con grande maestria e sensibilità da Margery Cantor e Dennis Crean. Dee Cuthbert-Cope ha offerto la sua meticolosa e competente revisione. Anche le doti artistiche di Ajahn Jitindriyā sono state una benedizione per il progetto: ha eseguito sia il bellissimo disegno di copertina sia molti elementi grafici.

Altrettanto pregevole è stata la collaborazione offerta da Madhu Cannon, segretario di Tsoknyi Rinpoche a Kathmandu, e da Erik Pema Kunsang, interprete e consulente in materia di lingua tipetana.

Infine, vorrei esprimere il mio apprezzamento per l'ispirazione post facto di René Daumal e del suo capolavoro spirituale incompiuto \textit{Il monte analogo}. La storia narra il viaggio di un gruppo di cercatori spirituali su una piccola barca chiamata ``L'impossibile'', verso un'isola nascosta dove svetta imponente il Monte Analogo, che loro aspirano a scalare. È stato soltanto dopo la trascrizione dei discorsi per \textit{Piccola barca, grande montagna} e la scelta del titolo, che ho letto l'eccellente racconto di Daumal.

