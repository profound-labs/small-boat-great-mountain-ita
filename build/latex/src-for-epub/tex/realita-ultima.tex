
\dropcaps{L}{'incontro fra tradizioni spirituali}, compresi gli insegnamenti Theravāda sulla saggezza e lo Dzogchen, due grandi espressioni del Buddha-Dharma, è uno dei principali benefici dei nostri tempi. La rivoluzione tecnologica ci permette di viaggiare, di comunicare e ha enormemente facilitato gli scambi fra tradizioni diverse. La maggior parte dei grandi testi spirituali sono reperibili su internet e inoltre, una grande quantità di conferenze e ritiri fanno sì che praticanti, insegnanti e maestri spirituali possano praticare insieme e discutere dei rispettivi lignaggi, di realizzazioni e conoscenze. Attualmente stiamo assistendo a una storica e memorabile apertura fra tradizioni spirituali diverse. Per la prima volta possiamo conoscere tutte queste tradizioni, capire dove coincidono e dove invece si scontrano.

L'altra sera, proprio all'inizio di questo ritiro, qualcosa mi ha fatto ricordare questa sorprendente confluenza di tradizioni. Erano da poco passate le sette e io ero seduto nella mia stanza. C'era calma e silenzio, quando all'improvviso ho sentito un forte rumore provenire da fuori. Siccome in quel momento al monastero si stavano facendo degli scavi, pensai che stessero portando qualche macchinario pesante per i lavori. Immaginai un'enorme macchina gialla che veniva trasportata fino al centro ritiri. Però poi udii quelli che sembrava fossero scoppi di un grosso motore, bang! Bang! Bang! Alla fine si interruppero e furono sostituiti da uno strepitio di trombe. Mi dissi: ``Sarà uno dei soliti festeggiamenti Dzogchen e avranno pensato che non era il caso di invitare i bhikkhu''. Poi però mi resi conto che quel tipo di feste, i festeggiamenti di Dharma, si celebrano alla fine e non all'inizio dei ritiri. Per cui continuai a chiedermi: ``Che significa? Che sarà mai tutto questo tafferuglio?''. Pensai che prima o poi l'avrei saputo. 

Finalmente mi ricordai che era l'inizio del nuovo anno ebraico e che una tradizione ebraica aveva qualcosa a che fare con suono di corni e rulli di tamburo. Mi ricordai anche che nella tradizione tibetana si dice che quando il Buddha fu invitato a insegnare dagli dèi brahmā, gli dèi arrivarono per fare la loro richiesta con un corno di conchiglia e una ruota del Dharma. Pensai: ``Forse dopo tutto non si tratta di una tradizione ebraica. Forse sono gli dèi brahmā che discendono con i loro corni di conchiglia e le ruote del Dharma per richiedere gli insegnamenti''.

Infatti, i suoni che avevo sentito facevano parte del rituale del nuovo anno ebraico e Wes Nisker era quello che soffiava nello shofar, la tromba di corno d'ariete. Più tardi appresi che il fischio acuto dello shofar simboleggia il richiamo a risvegliarsi dall'inconsapevolezza. Il suono dello shofar ci fa ricordare la nostra intima vocazione, il nostro vero scopo, risvegliarci ed essere liberi.

Questa è un'epoca meravigliosa per essere vivi e assistere a una tale vicinanza fra tradizioni spirituali diverse, sia in ambito buddhista sia fra altre religioni. Queste interconnessioni non solo ci spingono a guardare oltre l'aspetto esteriore di una tradizione spirituale, ma gettano anche una luce sugli enigmi con i quali conviviamo. Da una parte, ci sono gli insegnamenti orali, le tradizioni e le strutture che permettono di tramandare nel tempo e nello spazio, attraverso tutto il pianeta, conoscenze e valori. Dall'altra però, proprio quelle strutture possono diventare gli ostacoli che inibiscono e bloccano le stesse verità che stanno cercando di comunicare. 

Siamo davvero fortunati che il buddhismo sia così nuovo in Occidente. Molte persone hanno riflettuto sul fatto che ``questi sono i bei giorni andati''. Fra cento anni, avremo un presidente buddhista, ci arriveranno generose donazioni da filantropi e il buddhismo sarà diventato un'istituzione. Gli arrampicatori sociali troveranno conveniente diventare buddhisti e i tempi gloriosi saranno finiti. Per questo siamo fortunati a praticare prima che il buddhismo diventi parte della norma sociale. Essere buddhisti in questo momento storico significa stare ai margini. Dopo tutto, in termini convenzionali, non è socialmente molto profittevole essere buddhisti. Secondo me, uno dei grossi inconvenienti dell'essere monaco in Asia è che le persone ci attribuiscono automaticamente importanza solo perché abbiamo la testa rasata e gli abiti monastici. Le persone in Asia ci credono speciali, mentre in Occidente ci considerano solo degli eccentrici. Per la strada ci gridano dietro i commenti più disparati. In Inghilterra in genere sono epiteti del tipo ``Skinhead!'', ``Hari Krishna!'' o ``Salve Ari!''.

Questo incontro di espressioni spirituali distinte, in cui sono presenti sia una comprensione delle forme religiose sia un impegno nei loro confronti, è veramente prezioso. Ma anche in un contesto così solido c'è la sfida ad andare oltre, cioè a usare la forma e, allo stesso tempo, a vedere cosa c'è al di là. Dobbiamo essere capaci di accettare la convenzione e usarla per quello che è. Interiormente però, dobbiamo essere completamente liberi, senza confini; dobbiamo lasciar andare qualunque cosa. Esteriormente, dobbiamo essere molto rigorosi e precisi, seguire la routine e fare tutto secondo le regole. La mia esperienza mi insegna che ci vuole un po' di tempo per apprezzare il vero senso di tutto questo.

\section*{La ricerca della libert\`a}

Come probabilmente molti altri, anch'io da adolescente mi sono posto il problema della libertà. Essendo nato nel 1956, ho fatto appena in tempo a essere uno degli ultimi figli dei fiori. Per quasi tutta la mia giovinezza ho creduto negli ideali di libertà e desideravo farne davvero esperienza. Però, invece di diventare un anarchico bombarolo, mi convertii in un anarchico filosofo che sventolava fiori. Eppure avevo preso molto sul serio questa mia aspirazione alla libertà e intuivo nel profondo che la libertà è possibile, che tutti noi esseri umani abbiamo il potenziale di essere completamente liberi, che c'è qualcosa di infinitamente puro, non inibito e non inibibile dentro di noi. Tuttavia, nella mia esperienza mi trovai a scontrarmi con una serie infinita di restrizioni e frustrazioni. Prima era andar via di casa; poi era la legge; poi il fatto di non avere abbastanza soldi. Vedevo sempre qualche ostacolo davanti a me e che, se solo non ci fosse stato, sarei stato libero.

Ero completamente disorientato. Per quanto mi sforzassi di liberarmi da convenzioni, forme e strutture (in genere sfidando queste cose), sembrava ci fosse sempre un nuovo ostacolo. Continuavo a scontrarmi con limitazioni e di conseguenza mi sentivo costantemente frustrato. Stavo soffrendo e non ne capivo il motivo.

Lasciai l'Inghilterra e presi a viaggiare sperando di trovare la libertà da qualche parte, ovunque fosse. Andai nel Sudest asiatico e mi diedi a uno stile di vita dionisiaco, mangiare, bere, divertirsi; sesso, droga e rock \&\ roll; ballare sulla spiaggia sotto la luna inneggiando alla libertà. Eppure dentro di me sentivo che i nodi stavano arrivando al pettine; intuivo che questa strada decadente non conduceva verso la libertà. Allora continuai a cercare.

Me ne andai nel Nordest della Thailandia, dove difficilmente si incontrano turisti occidentali, e mi trovai per caso in un monastero della foresta. Era uno dei monasteri di Ajahn Chah in cui vivevano monaci occidentali. É importante sapere che la tradizione della foresta thailandese rappresenta la corrente più austera di un'ortodossia già di per sé piuttosto rigida in cui si pratica l'osservanza rigorosa di una tradizione già di per sé molto conservatrice. Quello che però mi colpì immediatamente fu che quelle persone, che pure conducevano un'insolita vita estremamente austera, fossero i tipi più allegri che avessi mai incontrato. Si alzavano alle tre del mattino, mangiavano un pasto al giorno, bevevano una tazza di tè due volte la settimana, dormivano su sottili stuoie di paglia, non praticavano sesso (assolutamente niente sesso), né alcol, droghe o rock \&\ roll. Eppure si trovavano perfettamente a loro agio, erano persone semplici e socievoli. Mi chiesi ``Cosa avranno da essere tanto allegri? Come fanno a essere così felici quando la loro vita è piena di rinunce?''. Poi incontrai Ajahn Chah, l'insegnante. Se avevo avuto l'impressione che i monaci fossero soddisfatti della loro sorte, incontrarlo fu uno shock ancora più grande. Ajahn Chah appariva come l'uomo più felice della terra. Da quarant'anni viveva da monaco nella foresta, senza sesso, musica o alcol. Si potrebbe ritenere che a quel punto chiunque dovrebbe essere bello che esaurito. Invece avevo davanti un uomo perfettamente a suo agio con la vita. Di fatto se la godeva completamente, era assolutamente soddisfatto.

La vita al monastero era estremamente sobria. Lo scopo era di ridurre al minimo tutti i fattori esterni così da poter concentrare tutta l'attenzione direttamente, in maniera univoca, nell'unico posto in cui è possibile trovare la libertà: il mondo interiore. Infatti lo stile di vita monastica invece di essere la negazione del mondo sensoriale, o una critica, o l'odio o la paura per esso, si basava sulla semplicità della vita. Il compito dei monaci era di portare l'attenzione sulla dimensione interiore, dove si può essere veramente liberi. Rimasi così affascinato da questo modo di essere che, con mio grande stupore, decisi di fermarmi. Quando ero arrivato, non pensavo che sarei rimasto più di tre giorni.

Ben presto mi resi conto che fino a quel momento avevo cercato la libertà nel posto sbagliato. Ricordo che in tutta sincerità e ridacchiando dissi a me stesso ``Come ho potuto essere così stupido?''. Non mi era mai venuto in mente che la libertà può venire solo dal di dentro. Avevo cercato la libertà in ciò che è inerentemente limitato. Il mio modo scriteriato di cercare la libertà consisteva nello sfidare le convenzioni, nel tentare di non farmi inibire dalle regole della società o dai dettami della mia personalità, o dai condizionamenti del mio corpo. Esteriormente apparivo libero, ma dentro ero prigioniero delle mie credenze e dei miei comportamenti. Solo volgendo l'attenzione all'interno avrei potuto scoprire la libertà che era già lì. Compresi che le forme esterne che adottiamo e usiamo (ad esempio le regole e gli orari da seguire durante i ritiri, il linguaggio e il vocabolario del buddhismo, le diverse tecniche di meditazione) sono studiati per aiutarci a portare l'attenzione là dove siamo già completamente liberi. Non è che si debba \textit{diventare} liberi, si tratta di scoprire quella qualità dell'essere che è inerentemente senza impedimenti e senza limiti.

\section*{Verit\`a convenzionale e verit\`a ultima}

Col passare del tempo, cominciai a prestare maggior attenzione all'importanza che Ajahn Chah sovente attribuiva al rapporto fra convenzione e liberazione, fra realtà convenzionale e realtà ultima. Le cose di questo mondo sono mere convenzioni create da noi stessi. Prima le creiamo e poi ci perdiamo in esse, o ne restiamo accecati. Questo è fonte di confusione, difficoltà e conflitto. Una delle grandi sfide della pratica spirituale è di creare le convenzioni, accettarle e usarle senza fare confusione. Possiamo recitare il nome del Buddha, prostrarci, cantare, seguire tecniche e pratiche, definirci buddhisti e poi, senza nessuna ipocrisia, riconoscere contemporaneamente che ogni cosa è completamente vuota. Non c'è nessun buddhista! Si tratta di un punto che Ajahn Chah ribadiva spesso: ``Se credi veramente di essere buddhista, allora sei completamente fuori strada''. Spesso gli capitava di sedersi sul seggio del Dharma, da dove teneva un discorso di fronte all'intera assemblea di monaci e laici e dire: ``Qui non ci sono monaci o monache, non ci sono laici, né uomini o donne, queste sono solo vuote convenzioni che creiamo noi''.

La nostra capacità di votarci onestamente a qualcosa e, contemporaneamente di vedere oltre, è un esercizio difficile per noi occidentali. Noi tendiamo a essere estremisti. O ci aggrappiamo a qualcosa e ci identifichiamo con essa, oppure pensiamo che sia priva di senso e la rifiutiamo; tanto non è reale in ogni caso. Per noi la Via di Mezzo non è sicuramente la più comoda. La Via di Mezzo significa far convivere la verità convenzionale con la verità ultima, e fare in modo che una non contraddica o smentisca l'altra.

Mi viene in mente un episodio accaduto durante una conferenza buddhista in Europa tenuta da un lama tibetano; tra il pubblico c'era un serissimo praticante tedesco. Il rinpoche aveva insegnato le visualizzazioni di Tārā e la \textit{pūjā} alle ventuno Tārā. Durante gli insegnamenti il praticante, con grande sincerità, congiunse le mani e chiese: ``Rinpoche, Rinpoche, io ho questo grosso dubbio. Vedi, tutto il giorno facciamo la pūjā alle ventuno Tārā e io mi sono molto impegnato in questa pratica. Voglio fare le cose per bene. Però ho questo dubbio: Tara, esiste veramente o no? Davvero Rinpoche, lei è lì o no? Se lei è lì, allora posso darmi con tutto il cuore; ma se lei non è lì, allora io non voglio fare la pūjā. Per cui ti prego, Rinpoche, una volta per tutte dicci, esiste o non esiste?''. Il lama rimase un momento con gli occhi chiusi, poi sorrise e disse: ``Lei lo sa di non essere reale''. Non ci è dato di sapere quale fu la reazione dello studente.

\section*{Che cos'\`e un essere vivente?}

Un aspetto sostanziale della maturità spirituale si impernia nella comprensione di quale sia la natura della realtà convenzionale. Una grossa parte del nostro condizionamento si basa sul presupposto che ci sia qualcosa che è un essere vivente `reale'. Vediamo noi stessi in termini di limitazioni di corpo e personalità e ci autodefiniamo all'interno di questi confini. Presumiamo che anche gli altri esseri siano limitate sacche di esistenza che fluttuano nel cosmo. Invece, molta della nostra pratica consiste nello smontare, nel de-costruire questo modello. Invece di prendere il corpo e la personalità come caratteristica che definisce ciò che siamo, prendiamo il Dharma come punto di riferimento fondamentale di ciò che siamo (o, per usare il linguaggio Vajrāyāna, prendiamo il \textit{Dharmakāya} come punto di riferimento fondamentale). Allora vediamo che il corpo e la personalità non sono altro che minuscoli sottoinsiemi di ciò e, di conseguenza, ci rapportiamo alla nostra vera natura in modo molto diverso. Il corpo e la personalità sono riconosciuti come finestrelle da cui filtra la natura di Dharma. Attraverso la matrice del corpo, la personalità e le nostre facoltà mentali, è possibile comprendere la natura della realtà; non è un qualcosa di insignificante che aggiungiamo a margine. In tutte le tradizioni buddhiste, comprendere cosa è un essere vivente significa rivedere tutta la struttura, l'immagine abituale di ciò che siamo.

Nell'ambito del buddhismo Mahāyāna (ad esempio nel \textit{Vajra Sūtra}), è abbastanza normale che negli insegnamenti si dica che ``Gli esseri viventi sono infiniti. Faccio voto di salvarli tutti'. E come fate a salvare tutti gli esseri viventi? Rendendovi conto che non ci sono esseri viventi. É così che salvate gli esseri viventi''. Dire però che non ci sono esseri viventi significa forse che non esistono? Non possiamo dire neanche questo. Un'autentica comprensione di questa espressione significa che riusciamo a vedere oltre le normali limitazioni dei sensi.

\section*{Dove siamo?}

Potete praticare la comprensione dell'esperienza della limitazione. Provate a escludere l'elemento fisico di ciò che siete e osservatevi semplicemente in termini di mente. Scoprirete che non solo si sfalda tutto il concetto di confine, ma anche l'idea di `dove sono io' e `dove sono gli altri'. Vi accorgerete che parlare del corpo, del posto che occupa e dello spazio tridimensionale ha senso solo in termini di \textit{rūpa-khandha}, cioè solo in relazione alla forma materiale del mondo. Infatti, `dentro' e `fuori', `qui' e `lì', `spazio' e `relazioni spaziali' hanno senso solo riferiti alla forma, non alla mente. La mente non esiste nello spazio. Lo spazio tridimensionale esiste solo in relazione al mondo della forma fisica.

Ecco perché meditare con gli occhi aperti è una buona verifica. Apparentemente là fuori ci sono dei corpi separati. Ce n'è uno qui, ce n'è uno lì. Con gli occhi chiusi, invece, è più facile provare una sensazione di unità. La forma materiale ci suggerisce la separatezza, ma la separatezza è completamente dipendente dal mondo materiale. In termini di mente, non è corretto dire `luogo'. La mente non è in qualche\textit{ luogo. }Noi siamo qui, eppure non siamo qui. Quelle limitazioni di un'individualità separata sono convenzioni che hanno una valenza relativa, ma non assoluta.

Creiamo l'illusione della separatezza e dell'individualità attraverso la fede nel mondo sensoriale. Quando cominciamo a lasciar andare il mondo sensoriale, in particolare il modo in cui ci rapportiamo alla forma fisica, allora cominciamo a essere in grado di espandere la visione di ciò che siamo come esseri. Non si tratta neppure di vedere come ci sovrapponiamo agli altri esseri; si tratta di comprendere che noi siamo tutt'uno con gli altri esseri.

\section*{La Via di Mezzo}

La meditazione è una danza speciale in cui ci impegniamo con tutto il cuore nella pratica di de-costruire la visione materialistica della realtà. La sfida è di trattenere e contemporaneamente lasciar andare; è di vedere chiaramente cosa stiamo facendo e allo stesso tempo vedere oltre. Per fare questo, è importante coltivare una \textit{sensibilità} alla Via di Mezzo. Questo è il punto di equilibrio. La Via di Mezzo non è semplicemente il punto intermedio fra due estremi, non è una sorta di mezzo e mezzo. È più come dire $[$\textit{regge il batacchio della campana verticalmente e sposta l'estremità inferiore verso sinistra}$]$ l'esistenza è qui e la non-esistenza è qui $[$\textit{sposta l'estremità inferiore verso destra}$]$. La Via di Mezzo è il perno centrale, intorno al suo vertice ruotano i due estremi; non è l'estremità inferiore del batacchio a metà oscillazione. Di fatto è la fonte da cui scaturiscono entrambe. Questo è solo uno dei modi per descriverla.

Può darsi che alcune persone abbiano più familiarità con la pratica tibetana, altre invece con le pratiche Theravāda e vipassanā. Spesso ci domandiamo ``É possibile renderle compatibili? É opportuno farlo?''. Quando cerchiamo di allineare metodi diversi rischiamo di confonderci, di andarci a ficcare in un vicolo cieco, perché un metodo ci dice di fare così e l'altro di fare colà. Per questo vi invito a riconoscere che qualunque tecnica, qualunque forma di espressione, non è altro che una convenzione che noi accettiamo di usare per raggiungere un unico obiettivo: trascendere la sofferenza ed essere liberati. Ecco a cosa ci indica \textit{qualunque} tecnica. 

Il modo per sapere se ciò che stiamo facendo è proficuo è chiederci: ``Mi sta conducendo verso la fine della sofferenza o no?''. Se sì, continuiamo. Altrimenti dobbiamo spostare la nostra attenzione su ciò che invece ci è utile. Basta semplicemente che ci chiediamo: ``Sto sperimentando \textit{dukkha}? C'è un senso di alienazione o di difficoltà?''. Se c'è, significa che ci stiamo aggrappando o attaccando a qualcosa. Dobbiamo riconoscere che il cuore è attaccato a qualcosa e compiere un gesto per allentare la presa, per lasciar andare. A volte non vediamo dov'è che si genera la sofferenza. Siamo così abituati a fare le cose in un determinato modo che lo diamo per scontato. Ma nella meditazione, mettiamo in discussione lo status quo. Investighiamo dov'è il senso di `mal-essere' e cerchiamo di scoprirne la causa. Facendo un passo indietro ed esaminando attentamente la sfera interiore, è possibile scoprire dov'è l'attaccamento e qual è la causa. Ajahn Chah soleva dire: ``Se ti prude la gamba, mica ti gratti l'orecchio''. In altre parole, vai dov'è dukkha, a prescindere da quanto possa essere sottile, riconoscilo e lascialo andare. È in questo modo che lasciamo che dukkha si dissolva. È così che sappiamo se le pratiche che stiamo facendo sono efficaci o meno.

I miei consigli e suggerimenti su come comprendere la realtà convenzionale e la realtà ultima non sono qualcosa in cui dovete credere. Gli insegnamenti buddhisti vengono sempre esposti come temi da contemplare. Dovete scoprire per conto vostro se quello che sto dicendo sembra ragionevole e autentico. Non vi preoccupate se vi si danno istruzioni contraddittorie. Cercate di non sprecare troppe energie o troppa attenzione per fare in modo che tutto corrisponda, altrimenti resterete confusi. Non riuscirete a far combaciare tutti i pezzi, però potete recarvi nel luogo da cui provengono.

