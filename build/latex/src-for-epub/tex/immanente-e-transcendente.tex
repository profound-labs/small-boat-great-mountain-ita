
\dropcaps{P}{erch\`e la nostra pratica} determini una libertà autentica e inamovibile, abbiamo bisogno di osservare con attenzione le motivazioni che guidano il nostro cuore.

È facile che lavorando con la mente si diventi inconsapevoli di un atteggiamento feroce e ossessivo. Il termine tibetano \textit{trekcho} significa `recidere', e recidere la corda dell'attaccamento è un aspetto molto importante della pratica della saggezza. Ci sono bellissime immagini di Mañjushrī che con la sua spada fiammante recide le illusioni e questo è un simbolo religioso molto incisivo. È di questo tipo di chiarezza di cui abbiamo bisogno per aprirci un varco nel groviglio dell'ignoranza, per scuotercela di dosso, per venire fuori da ciò che ci blocca. Ma è anche facile che questo atto di recidere diventi una mera abitudine che pervade tutti i nostri sforzi in meditazione, anche quando miriamo a praticare in un atteggiamento mentale di \textit{non} meditazione. 

Il nostro modo di operare può diventare piuttosto brutale e squilibrato. Sto parlando proprio della mia esperienza personale, perché spesso ho notato in me questa tendenza.

Quando iniziai a seguire gli insegnamenti del Buddha sulla saggezza, il mio atteggiamento, sebbene io sia un tipo alla mano, una persona socievole e gentile, era: ``Datemi l'essenza degli insegnamenti del vajra. Andiamo al sodo e recidiamo tutte le contaminazioni. Compassione e gentilezza amorevole è roba da ragazzini, datemi l'assoluto!''. La maggior parte di noi vuole il meglio, il massimo, il più assoluto, il più puro, il più del più, l'essenza segreta dell'essenza, ``la quintessenza essenziale più preziosa e raffinata di tutti i possibili insegnamenti''. Non vogliamo perdere tempo con quello che ci sembra inferiore o superficiale. Vogliamo roba buona, consistente. Le persone ascoltano gli insegnamenti o leggono libri e pensano: ``Sì, ecco; la vacuità assoluta, voglio questo. Dai, facciamolo''. E anche se questo è un elemento cruciale, è significativo il fatto che i canti che recitiamo qui in questo ritiro includono, in ogni insegnamento, la dedica del merito. In sostanza, anche questa pratica Dzogchen volta alla realizzazione di queste qualità di saggezza suprema, anche qui, come in tutta la tradizione Mahāyāna, c'è un richiamo costante a dedicare la nostra pratica al beneficio di tutti gli esseri.

Alcuni dei canti che mi piace fare durante i ritiri sono sui ``quattro \textit{brahma-vihāra}''\textit{, }il ``discorso sulla gentilezza amorevole'', le ``riflessioni sul benessere\textit{ }universale\textit{''} e la ``condivisione delle benedizioni''. Questi richiami `soft' sono l'espressione dolce, gentile e amorevole del Dharma. A mio avviso questi canti hanno una grande rilevanza. Probabilmente molte persone, specialmente coloro che praticano da molto tempo la vipassanā in Occidente, non sono abituate a fare molti canti durante i ritiri. In questi ambienti l'elemento devozionale o cerimoniale non è molto presente. Personalmente invito le persone a considerare questi canti non come un abbellimento per l'umore, come i fiori sull'altare, ma come una parte sostanziale della pratica. I canti non hanno il semplice scopo di rallegrare l'ambiente. I canti riflettono il nostro impegno collettivo a mettere in pratica la via del Buddha. Dopo tutto, le parole sono sue, ma le voci sono le nostre. Ripeto, è un elemento religioso. Per anni e anni il mio atteggiamento nei confronti dei canti del mattino e della sera è stato di altezzosità e derisione. Recitavo tutte le parole ed eseguivo il rituale prescritto, però nell'intimo mi dicevo: ``Andiamo al sodo. Tagliamo corto e andiamo al dunque. Smettiamola di tergiversare e perdere tempo. Gentilezza amorevole, devozione, sono tutte storie!''.

\section*{Chiarire l'intenzione compassionevole}

La prima volta che ho avuto una rivelazione a questo riguardo è stato quasi vent'anni fa. Ero un giovane monaco molto zelante e, sebbene la mia mente fosse spesso super attiva e deconcentrata, dopo tre o quattro anni di addestramento monastico, scoprii che la meditazione mi riusciva abbastanza facilmente e che ero in grado di ottenere alti stati di concentrazione. Quelli erano i primi anni in Inghilterra per la nostra comunità, quando Ajahn Sumedho era solito fare due o tre discorsi di Dharma al giorno e sembrava che ci fosse un flusso continuo di profonda saggezza. Era un periodo di grande ispirazione. La sensazione era che l'illuminazione fosse a un passo, che fosse una realtà ovvia. Era solo questione di superare le ultime contaminazioni e, bam! il gioco era fatto.

Instaurammo la tradizione di fare un ritiro invernale durante i freddi e bui mesi di gennaio e febbraio. Una volta, il ritiro era cominciato da circa tre settimane e io mi stavo impegnando con molta diligenza ed ero estremamente concentrato nella meditazione; non parlavo con nessuno e non guardavo nulla. A ogni quarto di luna facevamo una veglia di meditazione per tutta la notte. Era la luna piena di gennaio. Io ero molto caricato e mi dissi con convinzione: ``Questa è la notte decisiva''. Era una notte limpida di un inverno inglese. In cielo brillavano le stelle e la luna piena era di uno splendore fiammeggiante. Ero elettrizzato. Ci trovammo per la meditazione seduta, facemmo i canti, ascoltammo il discorso di Dharma e così via, poi, finito tutto questo, per il resto della notte eravamo liberi di fare meditazione seduta e camminata a nostra discrezione.

Insomma, io stavo lì seduto con la mente molto chiara e luminosa e questo pensiero continuava a frullarmi in testa: ``Fra qualche minuto, fra qualche secondo''. Lo conosciamo tutti: ``Appena un po' a sinistra, appena un po' a destra, ok, adesso rilassati un po', raddrizzati un po', va bene, vai così, non fare nulla, va bene, va bene''. Sono sicuro che sapete tutti di cosa sto parlando.

Questo andò avanti per ore. La mia mente era sempre più energizzata, più chiara e luminosa, spazzando via le contaminazioni e gli oscuramenti a destra e a manca. Gli indizi si moltiplicavano, del tipo: ``Sta per succedere qualcosa di grosso''. Alle due del mattino circa, dei rumori cominciarono a penetrare nella mia coscienza: tump, tump, e poi boati, porte che si aprivano e si chiudevano, passi pesanti per il corridoio. Pensai: ``Scarpe nel corridoio? Chi è che cammina con le scarpe nell'atrio?''. Come potete immaginare, c'era una qualche interferenza con il mio progetto di illuminazione. Decisi però di ignorarla e mi dissi: ``È solo un rumore $[$\textit{ronzio}$]$. Soltanto io e la luna che ci affrettiamo verso il nibbāna''. Sebbene facessi del mio meglio per ignorare il rumore, notai a un certo punto una presenza davanti a me. Aprii gli occhi. Uno dei monaci era chino verso di me e diceva: ``Puoi venire fuori un momento?''. E il mio primo pensiero fu: ``Cosa intendi con `vieni fuori'? Questa è la mia notte. Ho da fare''. Resistetti all'impulso di esternare i miei pensieri, uscii dalla stanza e trovai la polizia nell'atrio. ``Polizia? Che sta succedendo qui?''.

Risultò che uno dei novizi, un giovanotto piuttosto stravagante di nome Robert, si era ficcato in qualche pasticcio. Tutta la meditazione durante il ritiro invernale, associata al fatto che non si fosse mai praticato quel tipo di concentrazione, poteva mandare in tilt diverse persone. Non solo il giovane Robert aveva oltrepassato i limiti, ma era anche andato molto lontano. Inoltre, aveva svuotato la cassetta per le piccole spese prima di andarsene. Al pub locale Robert aveva offerto da bere e aveva tenuto un discorso a tutti i presenti. Dato che si trovava in uno stato di mente lievemente alterato, ma molto lucido, si accorse di essere in grado di poter leggere nella mente delle persone. Fissava le persone nel pub dritto negli occhi e diceva: ``Tu fai questo e pensi quest'altro; lo so io cosa stai preparando''. E le persone andavano nei matti. Non vi scordate che siamo in Inghilterra, e gli abitanti di un villaggio inglese non sono certo abituati a vedere arrivare un giovanotto con la testa rasata e vestito di bianco che si intrufola nell'intimità del pub locale, fa regali a tutti e svela i loro segreti più intimi. In realtà gli inglesi non sono inclini a rivelare i loro segreti più intimi nemmeno nei momenti migliori. Che ci fosse poi qualcuno che si comportasse in maniera così bizzarra e divulgasse i loro pensieri era decisamente inaccettabile. Così chiamarono la polizia. I poliziotti, con buon senso e con compassione altrettanto tipicamente inglesi, si resero conto che il tipo non era perfettamente in sé, e lo ricondussero al monastero. Però a quel punto Robert era davvero fuori di sé. Cominciò a delirare e a parlare con veemenza, dicendo che voleva uccidersi.

Il monaco in piedi accanto a me disse: ``Robert è in un bel pasticcio. È estremamente alterato e si vuole buttare nel lago. Puoi andare ad aiutarlo? Sei il solo che possa farlo''. Era vero. Come lui, ero uno dei membri più giovani del Sangha e mi sentivo molto vicino a quel novizio, inoltre ero una delle poche persone nella comunità in grado di rapportarsi con lui.

A quell'epoca Robert viveva in un \textit{kuṭī }nella foresta. La maggior parte dei membri della comunità abitava nell'edificio principale o nel cottage delle monache, e il \textit{kuṭī }nella foresta si trovava a circa mezz'ora di cammino. Un parte della mia mente diceva: ``Ma insomma, questa è la mia grande notte dell'illuminazione''. Perciò il mio primo impulso fu di dire: ``Non questa notte''. Ma poi qualcosa dentro di me disse: ``Non essere stupido, vai, non hai scelta''. Quindi mi caricarono di thermos pieni di cioccolata bollente, dolci e altri generi di conforto ammessi a quell'ora nel monastero e mi precipitai su per il bosco. Per farla breve, trascorsi le tre ore seguenti in sua compagnia, bevendo the e cioccolata e tentando di calmarlo. Lo lasciai parlare, parlare e parlare. Alla fine, esausto, crollò e verso l'alba aveva voglia di dormire. Capii che stava bene e che non avrebbe commesso sciocchezze. Quindi lo lasciai e ritornai al monastero.

Mi stavo precipitando giù per la collina quando all'improvviso pensai: ``Che fretta c'è? Perché sto correndo?''. Rallentai il passo sempre di più e alla fine mi fermai e guardai in alto. La luna piena stava tramontando sull'altra sponda del lago. E allora riemersero tutte le voci che avevano affollato la mia mente durante la prima parte della notte: ``In qualunque momento, ormai. Questa è la \textit{mia} notte. Ci sto arrivando davvero''. Inoltre mi resi conto che in tutto questo neanche per un attimo avevo pensato a qualcun altro che non fossi io; io e il mio programma di illuminazione, io che mi risvegliavo, io che ottenevo la liberazione. Mi accorsi che non mi ero minimamente preoccupato di praticare per il beneficio di nessun altro. Mi sentii piccolo così $[$\textit{indica con il pollice e l'indice una distanza di pochi centimetri}$]$. Come potevo essere stato così incredibilmente stupido?

Il solo fatto di essere stato assieme a un essere sofferente mi permetteva di vedere quanto la mia attenzione durante la meditazione si era ridotta al punto che tutti gli altri esseri ne erano stati completamente estromessi. Quello che era iniziato come una buona intenzione, cioè il desiderio di svilupparmi spiritualmente ed essere liberato, che sembrava la cosa migliore che chiunque potesse fare con la propria vita, si era ridotto, ridotto e ridotto ancora fino a diventare io che conquistavo il primo premio. La motivazione incredibilmente superficiale della mia pratica fu palese. E io mi chiesi: ``A che serviva veramente tutta quella fatica?''.

Allora mi colpì profondamente quanto sia importante il principio altruistico. Perché anche se uno fa un grosso lavoro interiore e sviluppa ottime qualità e abili mezzi, ignorare il prossimo in questo modo inficia il vero scopo della nostra pratica. Gli altri esseri non sono un riferimento puramente nominale. Nella nostra comunità si è soliti cantare la `condivisione dei meriti' ogni giorno, ma fu solo dopo questo incidente che compresi: ``Le persone reali soffrono veramente. Giusto, le persone \textit{reali}\ldots{} ah''.

Essere stato così vicino a Robert in un momento in cui la mia mente era in uno stato di grande lucidità e sensibilità, permise al concetto di pratica per il beneficio di tutti gli esseri di scendere veramente nel profondo. Da quella volta ho cominciato a prestare più attenzione all'elemento altruismo nel suo insieme e a risvegliare consapevolmente un interesse per gli altri esseri. Non si tratta semplicemente di un'idea. Lo interiorizzai veramente.

Da quella volta ho cominciato a capire meglio molti degli insegnamenti Mahāyāna. Ho visto che quell'enfasi ristretta sull'illuminazione individuale era diventata la motivazione che guidava ciò che facevo. Attraverso quella prospettiva di `illuminazione personale', la mente naturalmente devia, trascurando il quadro generale.

\section*{Chiarire la pratica della saggezza} 

Durante i primi anni della nostra comunità in Inghilterra, il nostro atteggiamento nei confronti della vita era in qualche modo tutto rivolto verso l'alto e verso l'esterno. Era come se dicessimo: ``Finalmente il Dharma arriva in Occidente!''. Ajahn Sumedho era il leader glorioso. Tutto era dorato, fertile ed espansivo, c'erano energia ed entusiasmo incredibili. Si sentiva nell'aria una carica di Dharma.

Il nostro primo monastero, giù nel West Sussex, aveva un'ampia foresta, ma era limitato in quanto a edifici; non sarebbe mai stato adatto a ospitare una grossa comunità residente, né ai ritiri di gruppo. Eravamo cresciuti molto rapidamente e così la comunità aveva acquistato una vecchia scuola nell'Hertfordshire e la maggior parte del Sangha si era trasferita là. Era il monastero di Amaravati. Vi risiedevano circa trenta, trentacinque monaci e monache e circa venti laici.

Nel 1986 l'aspetto del ``verso l'alto e verso l'esterno'' si era andato sviluppando e raggiunse l'apice durante il ritiro invernale di quell'anno. Optammo per l'approccio ``niente prigionieri; o morte o gloria''. Il programma di pratica quotidiana andava dalle tre del mattino fino alle undici di sera; senza pause. Alcuni dei soggetti più robusti spaccavano il ghiaccio che ricopriva il laghetto e si tuffavano alle tre di notte per rinfrescarsi prima della seduta del mattino. (Siccome circolava testosterone in abbondanza, quest'abitudine era molto diffusa tra i maschi). La comunità si era lanciata in questo programma con grande allegria.

Sebbene Ajahn Sumedho non fosse molto loquace all'epoca, l'anno seguente, mentre si discuteva di come avremmo condotto il ritiro invernale, disse: ``Beh, veramente non sono molto soddisfatto dei risultati di ciò che abbiamo fatto l'anno scorso; sicuramente c'era del fuoco nell'aria, ma non ha avuto un buon effetto sulle persone''. Dopo aver valutato come andavano le cose per il Sangha, egli concluse dicendo: ``Questo spirito proprio non mi piace; va nella direzione sbagliata''.

Così per le prime due o tre settimane del ritiro invernale del 1987, Ajahn Sumedho continuava a dire alle persone di \textit{non} meditare: ``Siate semplicemente svegli''.

Soleva ripeterci infinite volte: ``Smettila; smetti di meditare!''. Continuava a ripetercelo e due o tre volte al giorno dava discorsi di Dharma sul non meditare. Diceva alle persone di aprire gli occhi e smettere di cercare di concentrarsi. A volte c'era una rimostranza lamentosa: ``Ma cosa dovremmo \textit{fare}?''. E la persona che l'aveva rivolta riceveva una risposta tonante: ``FARE? Non \textit{fare} nulla. Lo sei già. Non \textit{fare} niente''. Il metodo era identico alla `non-meditazione non-distratta' utilizzata nella pratica Dzogchen.

Cercava di indicare quella dimensione del fare, dell'affaccendarsi, quella qualità del \textit{divenire} che molto facilmente prende il sopravvento sulla meditazione. Essa può permeare l'intero sforzo della pratica spirituale. La tendenza al divenire prende il sopravvento e trova legittimazione nell'essere chiamata meditazione, o ``io che divento illuminato''. Nel frattempo, ci sfugge il fatto che stiamo perdendo di vista la questione principale e che quello che stiamo facendo si è trasformato in un programma basato sull'io. Rimaniamo intrappolati nell'illusione, nel tentativo di trasformare l'io in qualcos'altro, con il risultato che perdiamo di vista la vera essenza della pratica. Compiere lo sforzo di vedere come avviene tutto questo ha fatto sì che quel ritiro fosse molto ricco di frutti. Dopo due o tre settimane cominciammo a cogliere il senso di cosa significhi essere presenti: ``Non fare qualcosa adesso per diventare illuminato in futuro. Semplicemente, sii sveglio adesso''.

\section*{Essere Buddha}

Ajahn Sumedho cominciò anche a parlare molto su `essere Buddha'. Io arrivai ad apprezzare molto questo insegnamento perché, sebbene la mia pratica a quel punto fosse molto più equilibrata in termini di motivazione altruistica, continuavo ancora a lanciarmi al massimo della velocità perché pensavo di dover andare da qualche parte. Quella fu la prima volta che mi balenò il pensiero che forse qualunque atteggiamento di `andare da qualche parte' non era necessariamente una buona idea. 

Quando mi imbattei negli insegnamenti Dzogchen un po' di tempo fa, questi mi rammentarono molto quella qualità di `essere Buddha'. Come facciamo a creare un equilibrio, nella non-meditazione non-distratta, fra l'attenzione sottile, cioè una consapevolezza chiara ed esattamente focalizzata, e il non \textit{fare} nulla? Un'espressione che mi piace usare a questo proposito è ``diligente assenza di sforzo''. Si fa uso di energia. C'è un impegno. C'è un'unità o integrità di scopo. Eppure, c'è anche un'assenza di sforzo. Non spingiamo, non tiriamo, non cerchiamo di ottenere qualcosa, semplicemente lasciamo che l'energia naturale del cuore funzioni in maniera attenta e libera. ``Stai fermo e procederai sull'onda dello spirito'', come si dice nella tradizione taoista.

\section*{Compassione e saggezza: immanente e trascendente}

Questo `essere Buddha' in sostanza comporta l'integrazione di compassione e saggezza e acquisire un senso di come operino congiuntamente. 

Mañjushrī che brandisce la spada fiammeggiante, simboleggiando l'elemento della saggezza, è un ritratto molto virile dell'archetipo della luce penetrante e dell'energia che vi passa attraverso. Avalokiteshvara, anche conosciuto come Kuan Yin in cinese e Chenrezig in tibetano, impersona la compassione, che è un'energia ricettiva, spesso simboleggiata da una soave figura femminile. Il nome significa `Colei che ascolta i suoni del mondo'. Mentre Mañjushrī possiede l'energia maschile, estroversa, penetrante e raffigura la visione, Avalokiteshvara raffigura l'ascolto, la ricettività e l'accettazione.

Inoltre è interessante il fatto che Avalokiteshvara, in tutte le scritture dell'India, della Cina e del Tibet era in origine una figura maschile. In Cina, col passare dei secoli, Avalokiteshvara si è trasformato in una figura femminile. Attualmente le immagini lo rappresentano spesso come una figura femminile. Lui è diventato una lei. È quindi giusto e comprensibile che lei si manifesti come femmina, perché questa immagine rappresenta quella qualità che è intrinsecamente più ricettiva e femminile.

Se la nostra pratica non include entrambe queste qualità, la saggezza e la compassione, Mañjushrī e Kuan Yin, se spingiamo troppo verso una direzione o l'altra tendiamo a essere gravemente squilibrati. Il punto è mantenere sempre quest'ambivalenza. La sfida è di svuotare tutto e contemporaneamente apprezzare l'interezza delle cose.

Un'altra cosa sulla quale rifletto è la parola che il Buddha usava per riferirsi a se stesso, `Tathāgata'. Questa parola, composta di due parti, l'ha coniata il Buddha. La prima parte, `\textit{tatha}', significa `tale' o `così'; la seconda parte, `\textit{agata}', significa `venuto'; mentre la parola `\textit{gata}' significa `andato'. Di conseguenza c'è stato un lungo dibattito: è `Tath-agata' o `Tatha-gata'? Il Buddha è `così venuto' o `così andato'? Egli è completamente qui o completamente andato? È totalmente immanente o totalmente trascendente? Gli studiosi si sono arrovellati per secoli su questo dilemma.

Il Buddha amava i giochi di parole e l'ironia. Spesso usava i doppi sensi, quindi io ritengo che abbia usato intenzionalmente questo termine ambiguo. Significa \textit{sia} `completamente andato', \textit{sia }`completamente qui'. L'aspetto di `andato' è quello della saggezza trascendente: andato, vuoto, niente, completamente trasparente. `Così venuto', `venuto al così com'è', `venuto alla quiddità'. Sono gli aspetti dell'essere interamente qui, totalmente immanente, completamente in sintonia con tutte le cose, completamente attento e incarnato in tutte le cose. L'elemento di compassione è ciò che rappresenta il significato di `così venuto', dove ogni cosa è io. Nell'elemento di saggezza, nulla è io. 

Nel \textit{Sutta Nipata} il Buddha dice: ``Il saggio non considera nulla al mondo come se gli appartenesse, né considera nulla al mondo come se non gli appartenesse'' (SN 858). Questo illustra magnificamente cosa significa mantenere quell'ambivalenza.

\section*{Saggia gentilezza: amare non \`e piacere}

Come dovremmo usare e comprendere la gentilezza e la compassione a questo proposito? Per quanto mi riguarda, non mi piace insegnare la meditazione sulla gentilezza amorevole come un ambito separato della pratica spirituale. Trovo che sia molto più saggio coltivare la gentilezza amorevole come un tema di fondo, come una presenza gentile e amorevole che informa e infonde ogni sforzo compiuto nel nostro addestramento spirituale. Il modo in cui affrontiamo qualunque ambito dell'addestramento deve contenere questa qualità di gentilezza amorevole. Come presupposto di tutto questo, è importante capire che \textit{amare} ogni cosa non significa che ci deve \textit{piacere} ogni cosa. A volte si commette l'errore di credere che affinché ci sia la gentilezza amorevole dobbiamo cercare di fare in modo che ci piaccia ogni cosa. Ad esempio, potremmo tentare di convincerci che ci piace il dolore, il lutto, l'amore non corrisposto, lo scoperto in banca, il decadimento delle nostre facoltà mentali o la ex moglie che ci perseguita. Questo è un modo fuorviante di praticare la gentilezza amorevole.

\textit{Mettā} si comprende meglio come ``il cuore che non indugia nell'avversione''. Non indugiare nell'avversione verso nulla, neppure i nostri nemici. Qualcuno mi ha citato un passaggio da uno degli ultimi libri del Dalai Lama in cui parla dei cinesi e si riferisce ad essi come ``i miei amici, il nemico''. Pertanto, la gentilezza amorevole è quella qualità per cui siamo in grado di trattenerci dall'accumulare avversione, anche nei confronti di ciò che è amaro, doloroso, brutto, crudele o nocivo. Il punto è realizzare quel luogo nel nostro cuore dove sappiamo che anche questo ha un suo ruolo in natura. Sì, accogliere tutto lo spettro di ciò che è apparentemente sgradevole, repellente e decisamente spregevole.

La gentilezza amorevole è la qualità del permettere e dell'accettare queste cose come parte dell'intera immagine. Non si tratta di dire che approviamo qualunque cosa o che crediamo che cose come la tortura, il tradimento e la malignità siano buone. Si tratta di accettare che queste esistono e di riconoscere pienamente che fanno parte del paesaggio della vita. Eccole. Quando fondiamo la pratica sull'accettazione amorevole, qualunque cosa con cui abbiamo a che fare, in termini di nostra mente e nostro mondo, ha la qualità fondamentale della sintonia. Per quanto mi riguarda, trovo che questa qualità deve esserci sia se sto facendo pratica di concentrazione, pratica di visione profonda o la pratica Dzogchen sul non-dualismo (lasciare andare completamente ogni cosa nella sfera soggettiva e oggettiva). Dobbiamo riconoscere che non c'è un nemico. C'è il Dharma. Non c'è né loro, né quello né esso. Tutto è pertinente. Fondamentalmente tutto è pertinente e ha un suo posto in natura. 

I canti sui brahma-vihāra che facciamo compaiono nella `La parabola della sega' in cui il Buddha insegna che: ``Se qualcuno fosse catturato dai banditi ed essi gli stessero facendo a pezzi il corpo, membro a membro con una sega a due manici, chiunque lasciasse sorgere un pensiero di avversione nei loro confronti a causa di questo, non starebbe praticando i miei insegnamenti''. (M 21.20)

Mi rendo conto che qualcuno possa considerarlo un insegnamento incredibilmente scoraggiante e improbabile, ma io lo trovo di grande aiuto e pieno di saggezza. Si dice che l'odio non può andare d'accordo con il Dharma e che quindi non è mai giustificato.

Il Buddha ha usato un esempio estremo, quasi assurdo, là dove sarebbe assolutamente ragionevole provare una \textit{certa} avversione verso coloro che ci stanno facendo a pezzi con una sega. Si potrebbe pensare che una lieve irritazione, un'ombra di negatività qua e là, sarebbero accettabili. Eppure il Buddha non disse questo, è vero? Egli disse: ``Neanche la punta di un capello di avversione è accettabile''. Non appena il cuore vacilla verso ``no, questo non è pertinente, questo non dovrebbe essere, sei malvagio, perché io?'' allora il Dharma è stato offuscato, è andato perso. È così. Qualcosa dentro di noi potrebbe ribellarsi, ma il cuore sa che è vero. Qualunque indugiare nell'avversione lo indica molto chiaramente e a causa di questo l'avversione è un segno incontrovertibile che `il Dharma è andato perso'.

Non appena ci scopriamo a giudicare la nostra mente, o le persone che ci circondano con asprezza, coltivando un odio giustificabile verso il governo o la nostra mente discorsiva o le nostre emozioni errabonde, o le nostre vite distrutte, non c'è una visione della realtà; è offuscata. Questo atteggiamento non è in accordo con la verità. Quindi, l'odio, l'avversione, diventa un segno che ci dice che abbiamo perso la strada.

Questo livello di addestramento descritto dal Buddha può apparire assolutamente impraticabile, invece è fattibile. Credo che sia molto utile riconoscerlo, perché ciò di cui crediamo di essere capaci è molto diverso da ciò di cui siamo veramente capaci. Potremmo pensare: ``Non potrei mai farlo. È impossibile''. Invece, vi assicuro, è possibile. Quel potenziale è a disposizione di tutti noi. E quando scopriamo quella qualità di accettazione totale e di assoluta non-avversione, dove c'è gentilezza e compassione, lì c'è un'incredibile qualità di agio e rilassatezza, un'autentica non-discriminazione. Che tipo di saggezza stiamo sviluppando se questa fa i bagagli e se ne va non appena il gioco si fa duro, non appena fa troppo caldo, o la persona `sbagliata' assume il comando, o il nostro organismo diventa debole e dolorante?

Uno spirito di sincera gentilezza amorevole è la cosa più difficile da creare di fronte all'amarezza e al dolore, perché per farlo c'è bisogno di trovare la spaziosità che circonda queste esperienze. È qui che il cuore tende più facilmente a contrarsi e a scontrarsi. Eppure possiamo raccogliere questa qualità e dire: ``Sì, anche questo fa parte della natura. Anche questo è soltanto così com'è''. Allora, in quel momento, vi è un'espansione intorno all'esperienza. Percepiamo lo spazio della vacuità che la circonda e la pervade e vediamo che è trasparente. Per quanto densa e reale sia la sensazione di `io, me e mio' in quel trattenere, vediamo in questa spaziosità che non solo c'è uno spazio che la circonda, ma c'è anche una luce che la attraversa.

\section*{Quando accade il peggio}

Una storia che mi piace raccontare a questo proposito riguarda l'insegnante del Venerabile Maestro Hsüan Hua. Il Maestro Hua era l'abate della Città dei Diecimila Buddha ed è stato colui che ci ha dato la terra dove si trova ora il nostro monastero. Lui e Ajahn Sumedho erano ottimi amici. L'insegnante del Maestro Hua, il Venerabile Maestro Hsü Yün, era il patriarca di tutti e cinque i lignaggi del buddhismo in Cina ed era molto rispettato. Era a capo del lignaggio Ch'an, del lignaggio dei sūtra, del lignaggio dei mantra, del lignaggio del Vinaya e del lignaggio esoterico. Non è un segreto che sètte diverse tendono a discutere fra di loro, eppure egli era così incontrovertibilmente puro e dotato che tutti lo volevano come capo. Quando l'Armata Rossa prese il potere, si cercò di fare piazza pulita di qualunque religione e ovviamente lui venne preso di mira. L'esercito cinese attaccò il suo monastero quando lui aveva circa centodieci anni. Lo picchiarono con mazze di legno fino a lasciarlo esanime a terra in una pozza di sangue, ritenendolo morto. Sebbene gli avessero spezzato le ossa e leso organi interni, egli si riprese. La notizia che fosse sopravvissuto si sparse nella regione. Un po' di tempo dopo, i soldati dell'Armata Rossa ritornarono e questa volta usarono mazze di ferro per picchiarlo fino a ridurlo un disastro. Il fragile vecchio era letteralmente a pezzi e gravemente ferito, eppure non morì.

I suoi discepoli si prendevano cura di lui e cercavano di aiutarlo a guarire dalle gravissime lesioni. Tutti loro erano esterrefatti che fosse ancora vivo. Inutile dire che egli era dotato di incredibili poteri meditativi, pertanto i suoi discepoli erano convinti che cercasse di conservare la propria energia vitale per loro. Credevano che il maestro comprendesse la sensazione di lutto che avrebbero provato se fosse morto e gli erano tutti molto devoti. E quindi lo imploravano: ``Per favore, non rimanere in vita solo per il nostro bene. Siamo profondamente commossi dal fatto che stai sopportando settimane e settimane di dolore e agonia solo perché non vuoi lasciarci abbattuti dalla perdita. Ma è ora che tu muoia, preferiremmo che ti concedessi di andare in pace piuttosto che sopportare tutta questa sofferenza''. Ed egli disse: ``Quello che faccio non è per voi. È vero che sto rimanendo in vita, ma non per il vostro bene, è per i soldati. Se morissi in conseguenza delle loro percosse, la retribuzione karmica per coloro che mi hanno aggredito sarebbe immensa, e non potrei sopportare di esserne il responsabile''. Dopo di ciò, i soldati lo lasciarono in pace. Egli sopravvisse e addirittura tenne ancora ritiri. I libri \textit{Ch'an and Zen Training}, tradotti da Charles Luk, sono tratti dai discorsi di Dharma che egli tenne in un ritiro quattro anni dopo.\footnote{Traduzione italiana: Lu K'uan-Yu (Charles Luk), \textit{Ch'an e Zen}, Edizioni Mediterranee, Roma 1977 (N.d.T.).} Morì all'età di centoventi anni. Aveva fatto voto di essere monaco per cento anni.

Quindi, non indugiare nell'avversione verso nulla è sicuramente fattibile.

C'è un'altra storia a questo proposito che riguarda un essere leggermente meno eminente, uno dei nostri monaci che stava compiendo un pellegrinaggio in India. Stava visitando i luoghi santi durante una marcia di mille miglia insieme a un laico. Erano sei mesi che viaggiavano per l'India, vivendo quasi esclusivamente di elemosine. I villaggi indiani possono essere piuttosto pericolosi e dovunque andassero, la gente continuava a ripetere loro: ``State attenti, state attenti, girano dei banditi; potrebbero derubarvi''. Ma loro continuavano ad andare avanti, nonostante gli avvertimenti e pensavano: ``Oh no, non a noi, noi stiamo compiendo questo santo pellegrinaggio, nulla può toccarci''. Facevano canti di protezione ed erano stati benedetti da svariati grandi maestri prima di partire. E, dato che avevano già attraversato alcune regioni pericolose senza alcun inconveniente, si erano fatti un po' arroganti: ``Ce la caviamo piuttosto bene, qui''.

Prima che questo monaco partisse per l'India, il Maestro Hua aveva visitato il monastero di Amaravati. Una volta, mentre dava un discorso informale a un gruppo di noi, il monaco che stava per partire gli fece una domanda. Il Maestro Hua, che non sapeva che sarebbe andato in India, rispose alla sua domanda dicendo: ``Quando vai a praticare nei luoghi del Buddha, non dovresti trovare difetti in nessuno e per nessuna ragione''. Così, quando andò in India, il monaco lo prese per un mantra e lo scolpì nella coscienza.

Il monaco e il laico che lo accompagnava stavano viaggiando attraverso un territorio ricco di foreste fra Nālandā e Rajgir, quando, all'improvviso, si imbatterono in un gruppo di uomini dall'aspetto arcigno che stavano abbattendo alberi nella foresta. Tutti avevano asce e randelli di legno. Era un posto molto isolato e gli uomini gli si fecero immediatamente intorno. Volevano prendergli tutto. A quel punto il laico, cercando di proteggere il monaco, cominciò a battersi con questi. Dopo averle prese il laico si diede alla fuga, inseguito da un paio di banditi. Altri quattro erano rimasti soli con il monaco. Era chiaro che l'avrebbero ucciso. Dato che il monaco parlava un po' di hindi, era in grado di capire quello che stavano dicendo. Non solo, il capo bandito brandiva un'ascia sopra la sua testa. La situazione non lasciava adito a dubbi.

A quel punto il pensiero gli balenò improvvisamente in testa: ``Quando vai a praticare nel luoghi del Buddha, non dovresti trovare difetti in nessuno e per nessuna ragione''. Egli ragionò così: ``Se è questo quello che sta succedendo, non posso fuggire. Non mi batterò con queste persone, e se anche lo facessi, vincerebbero comunque. Semplicemente mi arrenderò a loro''. Allora chinò il capo, giunse le mani e cominciò a cantare: ``Namo tassa\ldots{}''. Rimase calmo in attesa di essere colpito dall'ascia, ma non accadde nulla. Guardò in su e vide che l'uomo che teneva l'ascia non riusciva a colpire. Allora il monaco si fece un po' ardito e fece questo gesto $[$\textit{Con un dito} \textit{t}\textit{raccia una linea in mezzo alla testa}$]$, ma di nuovo il bandito non poteva decidersi a colpire.

A quel punto il laico, che si era nascosto pensò: ``Aspetta un attimo. Dovrei proteggere il monaco e non sto compiendo il mio dovere''. Ritornò di corsa per cercare di aiutarlo. Si azzuffarono di nuovo. Il laico si rese conto di essere nuovamente in pericolo e corse via un'altra volta. Si nascose fra i cespugli in fondo a un dirupo. Morale della storia, i banditi gli rubarono tutto. Al monaco vennero lasciati i sandali e la parte inferiore dell'abito. Tutto il resto se l'erano portato via.

Però, in tutto questo, il monaco non si fece nemmeno un graffio. Il laico che si era battuto era stato un po' sbatacchiato ed era un po' malconcio a causa delle spine dei rovi e della caduta dal dirupo. Più tardi, mentre parlavano dell'accaduto, il monaco comprese: ``Se fossi morto, sarei morto con la mente concentrata sul Triplice Gioiello''. Il laico capì: ``Se fossi morto, sarei morto con la mente di un animale braccato''. 

Queste sono immagini volutamente forti. Eppure, rappresentano molto chiaramente una determinata preziosa qualità. Queste storie ci incoraggiano a rivolgerci verso ciò che ci spaventa o ci repelle di più. Quando l'uomo con l'ascia ci minaccia, possiamo rivolgerci a lui e dire: ``Prego, sono pronto''. Anche quando l'ascia è dentro di noi, quelle intense ondate di avidità, o di paura e di ansia, le ondate di nostalgia e rimpianto, è quel movimento di rivolgersi verso queste esperienze e accettarle così come sono che permette al cuore di essere libero. La vera saggezza, lontano dall'essere oltre la pratica della gentilezza, di fatto dipende da questa accettazione non discriminante indifferenziata del bello e del brutto. Quando smettiamo di scappare di corsa dalle cose che sono apparentemente dolorose, addirittura insopportabili, e ci impegniamo nell'accettazione e la resa, si ha una trasformazione magica. Trasformiamo la cosiddetta difficoltà e ci spostiamo verso uno stato completamente diverso.

Quando riandiamo con la mente a una situazione in cui qualcuno ci ha aggredito con rabbia e noi semplicemente abbiamo ricevuto quell'energia senza reazioni, siamo diventati uno specchio, giusto? O se noi ci avventiamo su qualcuno e questa persona si limita a dire ``non è proprio la tua giornata, non è così?'' ci ritorna diritto addosso e diciamo: ``Hai ragione, mi dispiace''. L'intensità è trasformata dalla purezza della riflessione. E quando abbiamo a che fare con la nostra vita emotiva, lo stesso tipo di ricettività aperta e chiara ha il potere di trasmutare lo stato emotivo. Non lo sopprime. L'emozione avvizzisce; la sua energia si trasforma in qualcosa che di fatto ravviva e illumina la mente, il calore è trasformato in luce.

\section*{Traffico a doppio senso}

Negli insegnamenti del Buddha ci sono molte strutture con cui ci imbattiamo, come ad esempio i brahma-vihāra (gentilezza amorevole, compassione, gioia simpatetica ed equanimità). Queste vengono illustrate come pratiche specifiche che possiamo sviluppare. Lo stesso vale per i sette fattori dell'illuminazione. Con lo sforzo, questi stati del cuore e della mente possono essere coltivati, ma è necessario comprendere il quadro generale. Quando il cuore è completamente illuminato e liberato, quando c'è rigpa, la consapevolezza non dualistica, la disposizione naturale del cuore è gentilezza amorevole, compassione, gioia ed equanimità. Queste qualità si irradiano naturalmente quando il cuore è completamente libero. Non c'è qualche `cosa' che `io faccio'. Questa è la disposizione innata del cuore puro. Lo stesso vale per i fattori dell'illuminazione (consapevolezza, contemplazione della realtà, energia, gioia, tranquillità, concentrazione ed equanimità). Si tratta di qualità intrinseche della mente liberata, del cuore risvegliato e illuminato. Sono manifestazioni immanenti di questa realtà trascendente.

Oppure prendete i cinque precetti: quando il cuore è completamente liberato è impossibile ferire deliberatamente un altro essere. È impossibile agire mossi da avidità. È impossibile abusare sessualmente di un altro essere o usare il proprio mondo sensoriale con indulgenza. È semplicemente impossibile. Non potete mentire o usare la parola in modo da ferire o ingannare. È come se la forza di gravità spirituale non ve lo permettesse. Non c'è nulla che possa farvi rinunciare alla verità. 

Quando diciamo ``praticherò la gentilezza amorevole'' o ``svilupperò la compassione'', o ``osserverò i cinque precetti'', palesemente prendiamo quella specifica qualità come una pratica. Di fatto, quello che stiamo facendo veramente è allineare le condizioni della nostra mente dualistica con la realtà della nostra stessa natura. Stiamo aiutando il condizionato a essere in sintonia, in accordo con l'incondizionato e, grazie a questa risonanza, a questa sincronia, c'è una spaziosità che si apre fra le condizioni e attraverso di essa si realizza l'incondizionato. Praticando i sette fattori dell'illuminazione o i brahma-vihāra, stabiliamo le condizioni per questo spazio. Ciò che sta `fuori' in termini di condizionato è completamente intonato con ciò che sta `dentro'. È una pratica, un processo, che funziona nelle due direzioni. Quando pratichiamo la gentilezza amorevole, il nostro cuore automaticamente si accorda con la realtà e ci sentiamo bene. E quando il nostro cuore è risvegliato alla realtà, funziona automaticamente con la gentilezza amorevole o con un altro dei brahma-vihāra. È come il traffico nelle due direzioni su un'autostrada fra il condizionato e l'incondizionato.

Ci sono le qualità intrinseche che emergono introdotte dalle pratiche. Intoniamo le corde in modo da allineare i nostri comportamenti e gli atteggiamenti `esteriori' con ciò che è già `interiormente'. La bontà suona bene perché l'atteggiamento risuona con la realtà. Mentire e nuocere suonano male perché stonano con la realtà di ciò che siamo. È tutto qui. Il Buddha disse che i brahma-vihāra non sono qualità trascendenti; sono una dimora calma e meravigliosa. Facendo queste pratiche creiamo un allineamento in cui le cose coincidono. Le condizioni sono poste in modo che lo spazio è visibile e molto vicino. Quindi, non appena lo spazio si espande, bum! È proprio lì, allineato, e in quel momento il cuore è libero.

\bigskip

\textit{(Poscritto alla storia del monaco e del suo aiutante derubati: dopo circa tre giorni, non solo tutte le loro cose erano state rimpiazzate, ma molti degli oggetti che erano stati loro donati erano molto migliori di quelli che avevano prima.)}

